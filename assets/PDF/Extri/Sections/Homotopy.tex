\section{同伦范畴}

\subsection{局部化}

局部化与分式计算的一般理论见\cite{gabrielCalculusFractionsHomotopy1967}.

\begin{definition}
    (局部化). 给定范畴 $\mathcal{C}$ 与态射类 $S$. 对任意范畴 $\mathcal{D}$, 定义 $\mathrm{Funct}_S(\mathcal{C}, \mathcal{D})$ 为函子范畴 $\mathrm{Funct}(\mathcal{C}, \mathcal{D})$ 的全子范畴, 其对象为将 $S$ 中的态射映为 $\mathcal{D}$ 中同构的函子.
    \begin{enumerate}
        \item 称函子 $Q_1: \mathcal{C} \to \mathcal{C}_1$ 是 $\mathcal{C}$ 关于 $S$ 的弱局部化, 若以下是函子范畴间的等价
        \begin{equation}
            Q_1^* : \mathrm{Funct}(\mathcal{C}_1, \mathcal{D}) \xrightarrow{\sim} \mathrm{Funct}_S(\mathcal{C}, \mathcal{D}), \quad F \mapsto F \circ Q_1.
        \end{equation}
        \item 称函子 $Q_2: \mathcal{C} \to \mathcal{C}_2$ 是 $\mathcal{C}$ 关于 $S$ 的严格局部化, 若以下是函子范畴间的同构
        \begin{equation}
            Q_2^* : \mathrm{Funct}(\mathcal{C}_2, \mathcal{D}) \xrightarrow{\cong} \mathrm{Funct}_S(\mathcal{C}, \mathcal{D}), \quad F \mapsto F \circ Q_2.
        \end{equation}
    \end{enumerate}
\end{definition}

\begin{example}\label{ex: Gabriel-Zisman localization}
    给定范畴 $\mathcal{C}$ 与态射类 $S$, 其 Gabriel-Zisman 局部化 (\cite{gabrielCalculusFractionsHomotopy1967}) 的构造如下. 
    \begin{enumerate}
        \item 我们保持范畴 $\mathcal{C}$ 对象类, 形式地加入 $S$ 中态射的逆元, 得范畴 $\mathcal{C}_S$. 具体地, 子范畴 $\mathcal{C} \subseteq \mathcal{C}_S$ 具有相同对象类. 对任意 $X$ 与 $Y$,
        \begin{equation}
            (X,Y)_{\mathcal{C}_S} = (X, Y)_\mathcal{C} \sqcup ((Y,X)_\mathcal{C} \cap \mathcal{S}).
        \end{equation}
        依照无交并的定义, 我们将 $(X,Y)$ 中态射表示作二元组 $(n, f)$:
        \begin{equation}
            (X,Y)_{\mathcal{C}_S} = \{(0, f) \mid f \in (X,Y)_\mathcal{C}\} \cup \{(1, s) \mid s \in (Y, X)_\mathcal{C}\}.
        \end{equation}
        这是类 $\{0,1\} \times \mathsf{Mor}(\mathcal{C})$ 的一个子集. 依照 $\mathcal{C}$ 中态射复合关系约定
        \begin{enumerate}
            \item $(0,1_X)$ 是 $X \in \mathcal{C}_S$ 的恒等态射;
            \item $(0, g) \circ (0, f) \sim (0, g \circ f)$, 若 $g$ 与 $f$ 可复合.
        \end{enumerate}
        此时, $\mathcal{C}_S$ 是一个范畴, 但未必是局部小的. 存在子范畴
        \begin{equation}
            \iota : \mathcal{C} \to \mathcal{C} \to \mathcal{C}_S,\quad X \mapsto X,\quad f \mapsto (0,f).
        \end{equation}
        \item 定义以下规则生成的等价关系 $\sim$:
        \begin{itemize}
            \item[$\dagger$] $(1, s) \circ (0, s) \sim (0, 1_X)$ 与 $(0, s) \circ (1, s) \sim (0, 1_Y)$, 若 $s: X \to Y$ 属于 $S$.
        \end{itemize}
        得商函子 $\pi: \mathcal{C}_S \to (\mathcal{C}_S)/\sim$.
        \item 将复合函子 $\mathcal{C} \xrightarrow \iota \mathcal{C}_S \xrightarrow \pi (\mathcal{C}_S)/\sim$ 定义作 Gabriel-Zisman 局部化.
    \end{enumerate}
    商函子 $\pi$ 是以类为指标范畴的``滤过余极限''在 NBGC (von Neumann–Bernays–Gödel 与类的选择公理) 公理体系下合理的; 但是, 我们无法在集合视角中检验 $\mathcal{C}_S$ 中两个态射在局部化范畴中相同与否.
\end{example}

实际上, Gabriel-Zisman 局部化是严格的.

\begin{lemma}\label{lem: Gabriel-Zisman localization is strict}
    记 $Q:= \pi \circ \iota : \mathcal{C} \to (\mathcal{C}_S)/\sim$ 是\Cref{ex: Gabriel-Zisman localization} 定义的函子, 则有函子范畴的同构
    \begin{equation}\label{eq: Gabriel-Zisman localization is strict}
        Q^* : \mathrm{Funct}((\mathcal{C}_S)/\sim, \mathcal{D}) \xrightarrow{\cong} \mathrm{Funct}_S(\mathcal{C}, \mathcal{D}), \quad F \mapsto F \circ Q.
    \end{equation}
    \begin{proof}
        (对象 (函子) 映满). $Q$ 已将 $S$ 映作同构, $Q^\ast : \mathrm{Funct}((\mathcal{C}_S)/\sim, \mathcal{D}) \to \mathrm{Funct}(\mathcal{C}, \mathcal{D})$ 的像必然在全子范畴 $\mathrm{Funct}_S(\mathcal{C}, \mathcal{D})$ 中. 任取 $F : \mathcal{C} \to \mathcal{D}$ 使得 $F(S)$ 是同构, 取 $F$ 关于 $\iota$ 的分解
        \begin{equation}
            F_1 : \mathcal{C}_S \to \mathcal{D},\quad X \mapsto FX, \quad (0,f) \mapsto Ff, \quad (1,s) \mapsto (Fs)^{-1}.
        \end{equation}
        函子 $F_1$ 将 $\dagger$ 中等价关系映作恒等. 故存在唯一函子 $F_2 : (\mathcal{C}_S)/\sim \to \mathcal{D}$ 使得 $F_1 = F_2 \circ \pi$. 此时, $F = F_2 \circ Q$, 即 $Q^\ast$ 在对象层面上是满的.
        \\
                (态射 (自然变换) 全). 取 $\eta : F \to G$ 属于 $\mathrm{Funct}_S(\mathcal{C}, \mathcal{D})$. 自然变换是一族由 $\mathcal{C}$ 中对象标记的 $\mathcal{D}$ 中的态射, 故只需检验 $\eta : F_1 \to G_1$ 与 $F_2 \to G_2$ 是自然变换即可. 对前者, 任取 $s: X \to Y$ 属于 $S$, 以下是交换图:
        \begin{equation}
            % https://q.uiver.app/#q=WzAsNixbMSwwLCJYIl0sWzEsMSwiWSJdLFsyLDAsIkZfMVgiXSxbMiwxLCJHXzFYIl0sWzAsMCwiRl8xWSJdLFswLDEsIkdfMVkiXSxbMSwwLCIoMSxzKSJdLFs1LDQsIlxcZXRhX1kiXSxbMywyLCJcXGV0YV9YIl0sWzQsMiwiKEZfMXMpXnstMX0iLDAseyJjdXJ2ZSI6LTJ9XSxbNSwzLCIoR18xcyleey0xfSIsMix7ImN1cnZlIjoyfV1d
\begin{tikzcd}[ampersand replacement=\&]
	{F_1Y} \& X \& {F_1X} \\
	{G_1Y} \& Y \& {G_1X}
	\arrow["{(F_1s)^{-1}}", curve={height=-12pt}, from=1-1, to=1-3]
	\arrow["{\eta_Y}", from=2-1, to=1-1]
	\arrow["{(G_1s)^{-1}}"', curve={height=12pt}, from=2-1, to=2-3]
	\arrow["{(1,s)}", from=2-2, to=1-2]
	\arrow["{\eta_X}", from=2-3, to=1-3]
\end{tikzcd}.
        \end{equation}
        对后者, 交换方块在``商''的意义下必定也是交换的.
        \\
        (态射 (自然变换) 忠实). 证明 $Q^\ast$ 在态射 (自然变换) 层面全时, 我们注意到 $\{\eta_X\}_{X \in \mathsf{Ob}(\mathcal{C})} : F \circ Q\to F' \circ Q$ 是 $\mathcal{D}$ 中的一族态射. 由于 $Q$ 不改变对象类, 自然无法改变一族 $\mathsf{Ob}(\mathcal{C})$-指标的 $\mathcal{D}$ 中态射, 从而 $\eta$ 在 $Q^\ast$ 下的原像只能是自身. 
        \\
        (对象 (函子) 单地映上). 若 $F \circ Q$ 与 $F' \circ Q$ 是相同的函子, 则 $F$ 与 $F'$ 在对象层面相同. 假定 $\eta : F \circ Q \to F' \circ Q$ 与 $\theta : F \circ Q \to F \circ Q$ 是互逆的自然变换, 则存在唯一的原像 $\overline \eta : F \to F'$ 与 $\overline \theta : F' \to F$. 显然 $\overline \theta \circ \overline \eta$ 与 $\overline \eta \circ \overline \theta$ 可以取作恒等自然变换, 从而只能取作恒等变换.
    \end{proof}
\end{lemma}

\begin{remark}\label{rmk: universal property of Gabriel-Zisman localization}
    \Cref{eq: Gabriel-Zisman localization is strict} 是如下泛性质的决定式.
    \begin{itemize}
        \item 对任意函子 $F: \mathcal{C} \to \mathcal{D}$ 使得 $F(S)$ 是同构, 存在唯一函子 $\overline F : (\mathcal{C}_S)/\sim \to \mathcal{D}$ 使得 $F = \overline F \circ Q$.
    \end{itemize}
    此处谈及的``泛性质''不宜定义作顿范畴中的初/终对象, 应理解作普适函子问题的解 (见 solution d'un problème d'application universelle, \cite{EGA}).
\end{remark}

\begin{corollary}\label{cor: localization functor is epic}
    $FQ = GQ$ 当且仅当 $F = G$. 函子 $Q$ 类似``满态射''.
\end{corollary}

\begin{remark}
    (关于类的滤过). 假定 $\mathcal{I}$ 是一个图, 其对象与态射构成集合. $\mathcal{C}$ 存在任意余极限. 函子 $F: \mathcal{I} \to \mathcal{C}$ 的余极限是 $\coprod_{i \in I} F(i)$ 的商, 其等价关系由 $F(I)$ 中交换图生成. 通常地, 唯有求出具体的余极限 $\varinjlim_I F$, 我们才能检验 $\coprod_{i \in I} F(i)$ 中两个元素是否等价.
    \\
    若 $\mathcal{I}$ 是滤过的, 则无需求解余极限即可判断 $\coprod_{i \in I} F(i)$ 中两个元素是否等价. 事实上, $a_i \in X_i$ 与 $a_j \sim X_j$ 在滤过余极限对象中等价, 当且仅当存在一个包含 $i$ 与 $j$ 的有限子图 $I_0$, 使得 $a_i$ 与 $a_j$ 在 $\varinjlim_{I_0} F$ 中等价.
    \\
    在 Gabriel-Zisman 局部化中, $I$ 标记了范畴 $\mathcal{C}_S$ 的点与边, 以及条件 $\dagger$ 蕴含的交换图. $I$ 通常是真类. 若 $I$ 是滤过的 ($\kappa$-滤过的), 则 $\mathcal{C}_S$ 中两个态射在局部化范畴中态射的等价性可以在某个有限子图 (基数小于 $\kappa$ 的子图) 中检验. 通常来说, Gabriel-Zisman 局部化并非滤过的.
\end{remark}

即便局部化可以被``滤过地''定义, 局部化范畴的 $\mathrm{Hom}$-类未必是集合.

\begin{example}
    定义 $\mathcal{C}$ 为如下范畴: 对象类是 $\{A, B\} \sqcup \mathcal{X}$, 其中 $\mathcal{X}$ 是真类. 态射有且仅有以下几类:
    \begin{enumerate}
        \item 所有恒等态射, \qquad 2. 对任意 $X$, $(A,X)_\mathcal{C} = \{f_X\}$, \qquad 3. 对任意 $X$, $(B,X)_\mathcal{C} = \{g_X\}$.
    \end{enumerate}
    记 $\mathcal{S} = \{i_X\}_{X \in \mathcal{X}}$. 此时, $(A,B)_{\mathcal{C}_S/\sim}$ 是一个真类.
\end{example}

\begin{example}
    (分式计算). 第一手资料见\cite{gabrielCalculusFractionsHomotopy1967}. 其思想是将局部化范畴中``锯齿状''的态射化简作分式, 且两个分式的等价关系可以在有限步骤内检验. 局部化范畴中的一个态射即一个分式所在的等价类, 而这一等价类又是一个滤过系统, 该态射可表示为``分子部分''的滤过余极限. 由分式定义导出范畴 (Deligne 方法) 的例子见\cite{kellerDerivedCategoriesTheir1996a}.
\end{example}

\begin{definition}
    (局部化的记号). 以下规定几类局部化记号.
    \begin{enumerate}
        \item (GZ 局部化). 通常记作 $Q: \mathcal{C} \to \mathcal{C}[S^{-1}]$, 即\Cref{ex: Gabriel-Zisman localization} 定义的 $\pi \circ \iota : \mathcal{C} \to \mathcal{C}_S/ \sim$. 泛性质表述见\Cref{rmk: universal property of Gabriel-Zisman localization}.
        \item (分式). 如果 $S$ 是乘法系, 则由左 (右) 分式构造的局部化范畴记作 $S^{-1}\mathcal{C}$ ($LS^{-1}\mathcal{C}$).
        \item (加法商). 加法商通常用 $\mathcal{C} / \mathcal{B}$ 表示 (\Cref{thm:extri-quotient}), 之后将证明这是局部化.
        \item (同伦范畴). 给定模型范畴 $\mathcal{C}$, 其同伦范畴记作 $\mathsf{Ho}(\mathcal{C})$, 定义为 $\mathcal{C}$ 关于弱等价类的局部化.
    \end{enumerate}
\end{definition}

\subsection{加法局部化}

一个棘手的问题是, 加法范畴的 GZ 局部化范畴未必是加法范畴.

\begin{example}
    记 $\mathcal{C}$ 是域 $k$ 中的有限维向量空间范畴, 取 $S = \{0 \to k\}$. $\mathcal{C}[S^{-1}]$. 容易验证, $f \in \mathcal{C}$ 是局部化范畴中的零态射, 当且仅当 $\mathrm{rank}(f) \leq 1$. $\mathcal{C}[S^{-1}]$ 显然不是加法范畴.
\end{example}

为给出 $Q$ 是加法函子的充分条件, 先作如下准备.

\begin{definition}
    (积范畴). 将如下范畴定义作范畴 $\mathcal{A}$ 与 $\mathcal{B}$ 的积.
    \begin{enumerate}
        \item 对象类是 $\mathsf{Ob}(\mathcal{A}) \times \mathsf{Ob}(\mathcal{B})$ (类的 Catersian 积).
        \item 态射类是 $\mathsf{Mor}(\mathcal{A}) \times \mathsf{Mor}(\mathcal{B})$ (类的 Catersian 积).
        \item 恒等态射与态射复合按分量定义.
    \end{enumerate}
\end{definition}

\begin{lemma}\label{lem: bifunctor is funct in product}
    $\mathrm{Funct}(\mathcal{A} \times \mathcal{B}, \mathcal{C})$ 恰好包含由 $\mathcal{A}$ 与 $\mathcal{B}$ 至 $\mathcal{C}$ 的双函子.
    \begin{proof}
        给定 $F : \mathcal{A} \times \mathcal{B} \to \mathcal{C}$, 下检验 $F(A, -) : \mathcal{B} \to \mathcal{C}$ 是函子.
        \begin{enumerate}
            \item 对象的对应是 $B \mapsto F(A,B)$, \quad 态射的对应是 $f: B \to B' \mapsto F(1_A, f)$.
            \item 复合律通过积范畴的态射复合 $(1_A, g) \circ (1_A, f) = (1_A, g \circ f)$ 检验, 单位律类似可验证.
        \end{enumerate}
        对 $f : A \to A'$ 与 $g : B \to B'$, 有等式
        \begin{equation}
            F(f,1_{B'}) \circ F(1_A, g) = F(f,g) = F(1_{A'},g) \circ F(f, 1_B).
        \end{equation}
    \end{proof}
\end{lemma}

\begin{remark}
    由以上引理, 我们似乎更愿意将 $A$ 写作态射 $1_A$. 依照经验, 若某问题涉及双函子, 则``应当''将对象提升作态射.
\end{remark}

\begin{lemma}\label{lem: currying for categories}
    (范畴的 Curry 化). 以下是函子范畴的同构
    \begin{equation}\label{eq: currying for categories}
        \mathrm{Funct}(\mathcal{A} \times \mathcal{B}, \mathcal{C}) \cong \mathrm{Funct}(\mathcal{A}, \mathrm{Funct}(\mathcal{B}, \mathcal{C})), \quad F \mapsto (A \mapsto (B \mapsto F(A,B))).
    \end{equation}
    \begin{proof}
        给定 $F : (A \times B) \to C$, 记对应所得的函子为
        \begin{equation}
            \mathfrak F: \mathcal{A} \to \mathrm{Funct}(\mathcal{B}, \mathcal{C}), \quad A \mapsto F(A,-).
        \end{equation}
        下证明 $F \mapsto \mathfrak F$ 是函子.
        \begin{enumerate}
            \item (对象). \Cref{lem: bifunctor is funct in product} 说明 $F(A,-) : \mathcal{B} \to \mathcal{C}$ 是一个函子. 
            \item (态射). 给定自然变换 $\theta_{(?,-)} : F \Rightarrow G$, 则 $\theta_{(A, -)} : \mathfrak F(A) \to \mathfrak G(A)$ 也是自然变换 (这无非将 $\mathcal{C}$ 中态射族的指标集由 $\{(?,-)\}$ 限制为 $\{(A,-)\}$).
            \item (复合律与恒等律). 容易验证. 证明的关键步骤是把双函子遗忘成单函子.
        \end{enumerate}
        反之, 我们由 $\mathfrak F$ 对应定义双函子 $F$. 困难之处是检验 $F$ 的双函子性. 任取 $(f,g) : (A,B) \to (A', B')$,
        \begin{align}
\theta_{(A',B')} \circ F(f,g) & = \theta_{A'}(B')\circ (\mathfrak F(f))(g)  = \theta_{A'}(B') \circ (\mathfrak F(f)) (\mathrm{id}_{B'}) \circ (\mathfrak F(\mathrm{id}_{A}))(g)\\
& = (\mathfrak G(f)) (\mathrm{id}_{B'}) \circ \theta_{A}(B') \circ (\mathfrak F(\mathrm{id}_{A}))(g) = (\mathfrak G(f)) (\mathrm{id}_{B'}) \circ (\mathfrak G(\mathrm{id}_{A}))(g) \circ \theta_{A}(B)\\
& = (\mathfrak G(f))(g) \circ \theta_A(B)  \qquad = G(f, g) \circ \theta_{(A,B)}.
\end{align}
复合律与恒等律验证从略.
    \end{proof}
\end{lemma}

\begin{example}
    记 $I$ 为图, $k$ 为域 (视作单点范畴). 积范畴 $k \times I$ 即路代数 $kI$. 例如, 对 $v \in \mathsf{Ob}(I)$ 与 $e \in \mathsf{Ob}(I)$, 则 $(1_k, 1_v)$ 与 $(1_k, e)$ 分别是路代数基底中的点与边. 由\Cref{eq: currying for categories}, 得函子范畴的同构:
    \begin{equation}
        \mathrm{Funct}(kI, \mathbf{Ab}) \cong \mathrm{Funct}(I, \mathrm{Funct}(k, \mathbf{Ab})).
    \end{equation}
    左侧等价于左 $kI$-模范畴; 右侧等价于 $I$ 的左 $k$-(模)表示范畴. 通常要求 $I$ 是有限图, 从而路代数 $kI$ 有单位元.
\end{example}

\begin{theorem}\label{thm: localization commutes with product}
    (积范畴的局部化). 给定范畴 $\mathcal{C}_1$ 与 $\mathcal{C}_2$, 分别取包含所有同构的态射类 $S_1$ 与 $S_2$, 则 $S_1 \times S_2$ 是 $\mathcal{C}_1 \times \mathcal{C}_2$ 的态射类, 且包含所有同构. 设 $Q_i : \mathcal{C}_i \to \mathcal{C}_i[S_i^{-1}]$ 是 $\mathcal{C}_i$ 关于 $S_i$ 的 GZ 局部化, 则
    \begin{equation}
        \mathcal{C}_1 \times \mathcal{C}_2 \to \mathcal{C}_1[S_1^{-1}] \times \mathcal{C}_2[S_2^{-1}],\quad (?, -) \mapsto (Q_1(?), Q_2(-))
    \end{equation}
    映 $S_1 \times S_2$ 为同构. 今断言, 局部化诱导的函子
    \begin{equation}
        (\mathcal{C}_1 \times \mathcal{C}_2)[(S_1 \times S_2)^{-1}] \to \mathcal{C}_1[S_1^{-1}] \times \mathcal{C}_2[S_2^{-1}]
    \end{equation}
    是范畴的同构.
    \begin{proof}
        对任意范畴 $\mathcal{D}$, 由\Cref{lem: Gabriel-Zisman localization is strict} 得
        \begin{equation}
            \mathrm{Funct}((\mathcal{C}_1 \times \mathcal{C}_2)[(S_1 \times S_2)^{-1}], \mathcal{D}) \cong \mathrm{Funct}_{S_1 \times S_2}(\mathcal{C}_1 \times \mathcal{C}_2, \mathcal{D}),\quad F \mapsto F \circ Q_{1 \times 2}.
        \end{equation}
        引入以下引理.
        \begin{quoting}
        \begin{lemma}
            函子 $G : \mathcal{C}_1 \times \mathcal{C}_2 \to \mathcal{D}$ 将 $S_1 \times S_2$ 映作同构, 当且仅当以下条件满足:
        \begin{enumerate}
            \item 对任意 $X_1 \in \mathcal{C}_1$, $G(X_1, -)$ 将 $S_2$ 映至 $\mathcal{D}$ 中同构;
            \item 对任意 $s_1 \in S_1$, $G(s_1, -)$ 是 $\mathrm{Funct}(\mathcal{C}_2, \mathcal{D})$ 中的自然同构.
        \end{enumerate}
        \begin{proof}
            ($\downarrow$). 对任意 $s_2 \in S_2$, $G(X_1, - )(s_2) = G(1_{X_1},s_2)$ 是同构. 对任意 $s_1 \in S_1$, 自然变换 $\{G(s_1, 1_{X_2})\}_{X_2 \in \mathsf{Ob}(\mathcal{C}_2)}$ 是同构. ($\uparrow$). 反之, $G$ 将 $(S_1 \times 1)$ 与 $(1 \times S_2)$ 映作同构, 因此将复合得到的 $S_1 \times S_2$ 映作同构.
        \end{proof}
        \end{lemma}
        \end{quoting}
        由这一引理,
        \begin{align}
            &\mathrm{Funct}_{S_1 \times S_2}(\mathcal{C}_1 \times \mathcal{C}_2, \mathcal{D}) \cong \mathrm{Funct}_{S_1}(\mathcal{C}_1, \mathrm{Funct}_{S_2}(\mathcal{C}_2, \mathcal{D}))\\
            \cong \ &\mathrm{Funct}(\mathcal{C}_1[S_1^{-1}], \mathrm{Funct}(\mathcal{C}_2[S_2^{-1}], \mathcal{D})) \cong \mathrm{Funct}(\mathcal{C}_1[S_1^{-1}] \times \mathcal{C}_2[S_2^{-1}], \mathcal{D}).
        \end{align}
        这说明 $(\mathcal{C}_1 \times \mathcal{C}_2)[(S_1 \times S_2)^{-1}]$ 与 $\mathcal{C}_1[S_1^{-1}] \times \mathcal{C}_2[S_2^{-1}]$ 满足同一泛性质, 从而它们是同构的.
    \end{proof}
\end{theorem}

\begin{remark}
    以上证明过程与可表函子的米田引理有相似之处, 两者都是说明``泛性质''决定的对象唯一. 对后者, $\mathrm{Hom}$ 的``泛性质''由集合论语言描述.
\end{remark}

\begin{lemma}
    (伴随与局部化). 给定伴随函子的一组资料
    \begin{equation}
(\mathcal{D} \xrightarrow F\mathcal{C}) \dashv (\mathcal{C}  \xrightarrow G \mathcal{D}) ;\quad ( 1_{\mathcal{D}} \xrightarrow \eta GF, FG \xrightarrow \varepsilon 1_{\mathcal{C}}).
\end{equation}
假定存在 $\mathcal{C}$ 态射类的 $S$ 与 $\mathcal{D}$ 态射类的 $T$, 使得 $F(T) \subseteq S$ 且 $G(S) \subseteq T$. 记 $Q_\mathcal{C} : \mathcal{C} \to \mathcal{C}[S^{-1}]$ 与 $Q_\mathcal{D} : \mathcal{D} \to \mathcal{D}[T^{-1}]$ 是相应的 GZ 局部化, 则存在诱导的伴随函子
\begin{equation}
(\mathcal{D}[T^{-1}] \xrightarrow{\overline F} \mathcal{C}[S^{-1}]) \dashv (\mathcal{C}[S^{-1}] \xrightarrow{\overline G} \mathcal{D}[T^{-1}]);\quad ( 1_{\mathcal{D}[T^{-1}]} \xrightarrow{\overline \eta} \overline G \ \overline F, \ \overline F \ \overline G \xrightarrow{\overline \varepsilon} 1_{\mathcal{C}[S^{-1}]}).
\end{equation}

\begin{proof}
    函子 $\overline F$ 与 $\overline G$ 由泛性质诱导. 自然变换 $\overline \eta$ 定义如下:
    \begin{equation}
        % https://q.uiver.app/#q=WzAsNCxbMCwyLCJcXG1hdGhjYWwgRFtUXnstMX1dIl0sWzIsMiwiXFxtYXRoY2FsIERbVF57LTF9XSJdLFswLDAsIlxcbWF0aGNhbCBEIl0sWzIsMCwiXFxtYXRoY2FsIEQiXSxbMCwxLCIxX3tcXG1hdGhjYWwgRFtUXnstMX1dfSIsMCx7Im9mZnNldCI6LTN9XSxbMCwxLCJcXG92ZXJsaW5lIEcgXFwgXFxvdmVybGluZSBGIiwyLHsib2Zmc2V0IjozfV0sWzIsMCwiUV97XFxtYXRoY2FsIER9Il0sWzIsMywiMV97XFxtYXRoY2FsIER9IiwwLHsib2Zmc2V0IjotM31dLFsyLDMsIkdGIiwyLHsib2Zmc2V0IjozfV0sWzMsMSwiUV97XFxtYXRoY2FsIER9Il0sWzQsNSwiXFxvdmVybGluZSBcXGV0YSIsMCx7InNob3J0ZW4iOnsic291cmNlIjoyMCwidGFyZ2V0IjoyMH19XSxbNyw4LCJcXGV0YSIsMCx7InNob3J0ZW4iOnsic291cmNlIjoyMCwidGFyZ2V0IjoyMH19XV0=
\begin{tikzcd}[ampersand replacement=\&]
	{\mathcal D} \&\& {\mathcal D} \\
	\\
	{\mathcal D[T^{-1}]} \&\& {\mathcal D[T^{-1}]}
	\arrow[""{name=0, anchor=center, inner sep=0}, "{1_{\mathcal D}}", shift left=3, from=1-1, to=1-3]
	\arrow[""{name=1, anchor=center, inner sep=0}, "GF"', shift right=3, from=1-1, to=1-3]
	\arrow["{Q_{\mathcal D}}", from=1-1, to=3-1]
	\arrow["{Q_{\mathcal D}}", from=1-3, to=3-3]
	\arrow[""{name=2, anchor=center, inner sep=0}, "{1_{\mathcal D[T^{-1}]}}", shift left=3, from=3-1, to=3-3]
	\arrow[""{name=3, anchor=center, inner sep=0}, "{\overline G \ \overline F}"', shift right=3, from=3-1, to=3-3]
	\arrow["\eta", between={0.2}{0.8}, Rightarrow, from=0, to=1]
	\arrow["{\overline \eta}", between={0.2}{0.8}, Rightarrow, from=2, to=3]
\end{tikzcd}.
    \end{equation}
    其中, 选取 $\overline \eta$ 如下:
    \begin{equation}
        % https://q.uiver.app/#q=WzAsOCxbMSwwLCJcXG1hdGhybXtGdW5jdH0oMV97XFxtYXRoY2Fse0R9W1Reey0xfV19LCBcXG92ZXJsaW5lIEcgXFwgXFxvdmVybGluZSBGKSJdLFsxLDEsIlxcbWF0aHJte0Z1bmN0fShRX1xcbWF0aGNhbHtEfSwgXFxvdmVybGluZSBHIFxcIFxcb3ZlcmxpbmUgRiBcXCBRX1xcbWF0aGNhbHtEfSkgIl0sWzIsMSwiXFxtYXRocm17RnVuY3R9KFFfXFxtYXRoY2Fse0R9LCBRX1xcbWF0aGNhbHtEfSBcXCBHIEYpICJdLFsyLDAsIlxcbWF0aHJte0Z1bmN0fSgxX1xcbWF0aGNhbHtEfSwgRyBGKSJdLFswLDAsIlxcb3ZlcmxpbmUgXFxldGEiXSxbMywwLCJcXGV0YSJdLFszLDEsIlFfe1xcbWF0aGNhbCBEfVxcZXRhIl0sWzAsMSwiXFxvdmVybGluZSBcXGV0YSBRX3tcXG1hdGhjYWwgRH0iXSxbMCwxLCJcXGNvbmciLDJdLFswLDEsIihRX3tcXG1hdGhjYWwgRH0pXlxcYXN0IiwwLHsic3R5bGUiOnsiYm9keSI6eyJuYW1lIjoibm9uZSJ9LCJoZWFkIjp7Im5hbWUiOiJub25lIn19fV0sWzEsMiwiIiwwLHsibGV2ZWwiOjIsInN0eWxlIjp7ImhlYWQiOnsibmFtZSI6Im5vbmUifX19XSxbMywyLCIoUV97XFxtYXRoY2FsIER9KV9cXGFzdCIsMl0sWzMsNSwiXFxuaSIsMSx7InN0eWxlIjp7ImJvZHkiOnsibmFtZSI6Im5vbmUifSwiaGVhZCI6eyJuYW1lIjoibm9uZSJ9fX1dLFs0LDAsIlxcaW4iLDEseyJzdHlsZSI6eyJib2R5Ijp7Im5hbWUiOiJub25lIn0sImhlYWQiOnsibmFtZSI6Im5vbmUifX19XSxbNCw3LCIiLDEseyJzdHlsZSI6eyJ0YWlsIjp7Im5hbWUiOiJtYXBzIHRvIn19fV0sWzUsNiwiIiwxLHsic3R5bGUiOnsidGFpbCI6eyJuYW1lIjoibWFwcyB0byJ9fX1dXQ==
\begin{tikzcd}[ampersand replacement=\&]
	{\overline \eta} \& {\mathrm{Funct}(1_{\mathcal{D}[T^{-1}]}, \overline G \ \overline F)} \& {\mathrm{Funct}(1_\mathcal{D}, G F)} \& \eta \\
	{\overline \eta Q_{\mathcal D}} \& {\mathrm{Funct}(Q_\mathcal{D}, \overline G \ \overline F \ Q_\mathcal{D}) } \& {\mathrm{Funct}(Q_\mathcal{D}, Q_\mathcal{D} \ G F) } \& {Q_{\mathcal D}\eta}
	\arrow["\in"{description}, draw=none, from=1-1, to=1-2]
	\arrow[maps to, from=1-1, to=2-1]
	\arrow["\cong"', from=1-2, to=2-2]
	\arrow["{(Q_{\mathcal D})^\ast}", draw=none, from=1-2, to=2-2]
	\arrow["\ni"{description}, draw=none, from=1-3, to=1-4]
	\arrow["{(Q_{\mathcal D})_\ast}"', from=1-3, to=2-3]
	\arrow[maps to, from=1-4, to=2-4]
	\arrow[equals, from=2-2, to=2-3]
\end{tikzcd}.
    \end{equation}
    类似地定义 $\overline \varepsilon$. 这类构造具有统一格式 $\overline ? Q = Q ?$.
    \\
    下证明伴随中的三角恒等式. 以下第一行复合为恒等自然变换:
    \begin{equation}
        % https://q.uiver.app/#q=WzAsMTEsWzAsMCwiXFxvdmVybGluZSBGIl0sWzIsMCwiXFxvdmVybGluZSBGXFwgXFxvdmVybGluZSBHXFwgXFxvdmVybGluZSBGIl0sWzQsMCwiXFxvdmVybGluZSAgRiJdLFswLDEsIlxcb3ZlcmxpbmUgRlFfe1xcbWF0aGNhbCBEfSJdLFsyLDEsIlxcb3ZlcmxpbmUgRlxcIFxcb3ZlcmxpbmUgR1xcIFxcb3ZlcmxpbmUgRiBRX3tcXG1hdGhjYWwgRH0iXSxbNCwxLCJcXG92ZXJsaW5lIEZRX3tcXG1hdGhjYWwgRH0iXSxbMCwyLCJRX3tcXG1hdGhjYWwgQ30gRiJdLFs0LDIsIlFfe1xcbWF0aGNhbCBDfSBGIl0sWzIsMiwiUV97XFxtYXRoY2FsIEN9IEZHRiJdLFs1LDAsIlxcbWF0aHJte0Z1bmN0fShcXG1hdGhjYWwgRFtUXnstMX1dLCBcXG1hdGhjYWwgQ1tTXnstMX1dKSJdLFs1LDEsIlxcbWF0aHJte0Z1bmN0fShcXG1hdGhjYWwgRCwgXFxtYXRoY2FsIENbU157LTF9XSkiXSxbMCwxLCJcXG92ZXJsaW5lICBGIFxcb3ZlcmxpbmUgXFxldGEiXSxbMSwyLCJcXG92ZXJsaW5lICBcXHZhcmVwc2lsb24gXFxvdmVybGluZSAgRiJdLFszLDQsIlxcb3ZlcmxpbmUgIEYgXFxvdmVybGluZSBcXGV0YSBRX3tcXG1hdGhjYWwgRH0iXSxbNCw1LCJcXG92ZXJsaW5lICBcXHZhcmVwc2lsb24gXFxvdmVybGluZSAgRiBRX3tcXG1hdGhjYWwgRH0iXSxbNiw4LCJRX3tcXG1hdGhjYWwgRH0gRlxcZXRhIl0sWzgsNywiUV97XFxtYXRoY2FsIEN9IFxcdmFyZXBzaWxvbiBGIl0sWzMsNiwiIiwxLHsibGV2ZWwiOjIsInN0eWxlIjp7ImhlYWQiOnsibmFtZSI6Im5vbmUifX19XSxbNCw4LCIiLDEseyJsZXZlbCI6Miwic3R5bGUiOnsiaGVhZCI6eyJuYW1lIjoibm9uZSJ9fX1dLFs1LDcsIiIsMSx7ImxldmVsIjoyLCJzdHlsZSI6eyJoZWFkIjp7Im5hbWUiOiJub25lIn19fV0sWzksMTAsIihRX3tcXG1hdGhjYWwgRH0pXlxcYXN0Il1d
\begin{tikzcd}[ampersand replacement=\&]
	{\overline F} \&\& {\overline F\ \overline G\ \overline F} \&\& {\overline  F} \& {\mathrm{Funct}(\mathcal D[T^{-1}], \mathcal C[S^{-1}])} \\
	{\overline FQ_{\mathcal D}} \&\& {\overline F\ \overline G\ \overline F Q_{\mathcal D}} \&\& {\overline FQ_{\mathcal D}} \& {\mathrm{Funct}(\mathcal D, \mathcal C[S^{-1}])} \\
	{Q_{\mathcal C} F} \&\& {Q_{\mathcal C} FGF} \&\& {Q_{\mathcal C} F}
	\arrow["{\overline  F \overline \eta}", from=1-1, to=1-3]
	\arrow["{\overline  \varepsilon \overline  F}", from=1-3, to=1-5]
	\arrow["{(Q_{\mathcal D})^\ast}", from=1-6, to=2-6]
	\arrow["{\overline  F \overline \eta Q_{\mathcal D}}", from=2-1, to=2-3]
	\arrow[equals, from=2-1, to=3-1]
	\arrow["{\overline  \varepsilon \overline  F Q_{\mathcal D}}", from=2-3, to=2-5]
	\arrow[equals, from=2-3, to=3-3]
	\arrow[equals, from=2-5, to=3-5]
	\arrow["{Q_{\mathcal D} F\eta}", from=3-1, to=3-3]
	\arrow["{Q_{\mathcal C} \varepsilon F}", from=3-3, to=3-5]
\end{tikzcd}.
    \end{equation} 
    依照原伴随的三角恒等式, 第三行复合为 $Q_{\mathcal{D}}$, 故第二行复合也为 $Q_{\mathcal{D}}$. 由 $(Q_\mathcal{D})^\ast$ 全忠实, 第一行复合为 $1_{\overline F}$. 另一三角恒等式的证明类似.
\end{proof}

\end{lemma}

\begin{theorem}\label{thm: when localization is additive}
    (加法局部化). 给定加法范畴的 GZ 局部化 $Q: \mathcal{C} \to \mathcal{C}[S^{-1}]$. 若 $S$ 对直和封闭, 则 $Q$ 是加法函子.
\begin{itemize}
    \item 称 $S$ 对直和封闭, 若对任意 $f , g \in S$, 总有 $\binom{f \ \ 0}{0 \ \ g} \in S$.
\end{itemize}
\begin{proof}
    记 $I = \{\cdot , \cdot \}$ 是有两个点的离散图. 则有伴随函子
    \begin{equation}
        (\mathcal{C} \xrightarrow{\Delta} \mathcal{C}^I) \dashv (\mathcal{C}^I \xrightarrow {\oplus} \mathcal{C}); \quad (1_{\mathcal{C}} \xrightarrow{\eta} \oplus \Delta, \Delta \oplus \xrightarrow{\varepsilon} 1_{\mathcal{C}^I}).
    \end{equation}
    将 $C^I$ 视同 $\mathcal{C} \times \mathcal{C}$, 则有 $\Delta : ? \mapsto (?, ?)$ 与 $\oplus : (?, !) \mapsto ? \oplus !$. 由 $S$ 对直和封闭, 则局部化保持伴随函子. 结合\Cref{thm: localization commutes with product}, 得
    \begin{equation}
        % https://q.uiver.app/#q=WzAsNSxbMCwwLCJcXG1hdGhjYWwgQyJdLFsyLDAsIlxcbWF0aGNhbCBDIFxcdGltZXMgXFxtYXRoY2FsIEMiXSxbMCwyLCJcXG1hdGhjYWwgQ1tTXnstMX1dIl0sWzIsMiwiKFxcbWF0aGNhbCBDIFxcdGltZXMgXFxtYXRoY2FsIEMpWyhTIFxcdGltZXMgUyleey0xfV0iXSxbMywyLCJcXG1hdGhjYWwgQ1tTXnstMX1dIFxcdGltZXMgXFxtYXRoY2FsIENbU157LTF9XSJdLFswLDEsIlxcRGVsdGEgIiwwLHsib2Zmc2V0IjotMSwiY3VydmUiOi0xfV0sWzEsMCwiXFxvcGx1cyAiLDAseyJvZmZzZXQiOi0xLCJjdXJ2ZSI6LTF9XSxbMSwzLCJRX3tcXG1hdGhjYWwgQyBcXHRpbWVzIFxcbWF0aGNhbCBDfSJdLFszLDQsIlxcY29uZyJdLFswLDIsIlFfe1xcbWF0aGNhbCBDfSIsMl0sWzEsNCwiUV97XFxtYXRoY2FsIEN9IFxcdGltZXMgUV97XFxtYXRoY2FsIEN9IiwwLHsiY3VydmUiOi0zfV0sWzIsMywiXFxvdmVybGluZSBcXERlbHRhIiwwLHsib2Zmc2V0IjotMiwiY3VydmUiOi0xLCJzdHlsZSI6eyJib2R5Ijp7Im5hbWUiOiJkYXNoZWQifX19XSxbMywyLCJcXG92ZXJsaW5lIFxcb3BsdXMgIiwwLHsib2Zmc2V0IjotMiwiY3VydmUiOi0xLCJzdHlsZSI6eyJib2R5Ijp7Im5hbWUiOiJkYXNoZWQifX19XSxbNSw2LCJcXGJvdCIsMSx7InNob3J0ZW4iOnsic291cmNlIjoyMCwidGFyZ2V0IjoyMH0sInN0eWxlIjp7ImJvZHkiOnsibmFtZSI6Im5vbmUifSwiaGVhZCI6eyJuYW1lIjoibm9uZSJ9fX1dLFsxMSwxMiwiXFxib3QiLDEseyJzaG9ydGVuIjp7InNvdXJjZSI6MjAsInRhcmdldCI6MjB9LCJzdHlsZSI6eyJib2R5Ijp7Im5hbWUiOiJub25lIn0sImhlYWQiOnsibmFtZSI6Im5vbmUifX19XV0=
\begin{tikzcd}[ampersand replacement=\&]
	{\mathcal C} \&\& {\mathcal C \times \mathcal C} \\
	\\
	{\mathcal C[S^{-1}]} \&\& {(\mathcal C \times \mathcal C)[(S \times S)^{-1}]} \& {\mathcal C[S^{-1}] \times \mathcal C[S^{-1}]}
	\arrow[""{name=0, anchor=center, inner sep=0}, "{\Delta }", shift left, curve={height=-6pt}, from=1-1, to=1-3]
	\arrow["{Q_{\mathcal C}}"', from=1-1, to=3-1]
	\arrow[""{name=1, anchor=center, inner sep=0}, "{\oplus }", shift left, curve={height=-6pt}, from=1-3, to=1-1]
	\arrow["{Q_{\mathcal C \times \mathcal C}}", from=1-3, to=3-3]
	\arrow["{Q_{\mathcal C} \times Q_{\mathcal C}}", curve={height=-18pt}, from=1-3, to=3-4]
	\arrow[""{name=2, anchor=center, inner sep=0}, "{\overline \Delta}", shift left=2, curve={height=-6pt}, dashed, from=3-1, to=3-3]
	\arrow[""{name=3, anchor=center, inner sep=0}, "{\overline \oplus }", shift left=2, curve={height=-6pt}, dashed, from=3-3, to=3-1]
	\arrow["\cong", from=3-3, to=3-4]
	\arrow["\bot"{description}, draw=none, from=0, to=1]
	\arrow["\bot"{description}, draw=none, from=2, to=3]
\end{tikzcd}.
    \end{equation}
    记复合函子 $\widetilde \Delta : \mathcal C[S^{-1}] \xrightarrow {\overline \Delta} (\mathcal C \times \mathcal C)[(S \times S)^{-1}] \cong \mathcal C[S^{-1}] \times \mathcal C[S^{-1}]$. 交换图说明 $\widetilde \Delta Q_{\mathcal{C}} = (Q_\mathcal{C} \times Q_\mathcal{C}) \Delta$, 该等式中的 $\widetilde \Delta$ 可以替换作 $\Delta$, 泛性质 ($Q_{\mathcal{C}}$ 右可消, \Cref{cor: localization functor is epic}) 说明 $\widetilde \Delta = \Delta$. 因此, $Q_\mathcal{C}$ 保持 $\Delta$, 故保持直和 $\oplus$. 保持直和的函子是一个加法函子 (\cite{crewHomologicalAlgebraLecture2021}).
\end{proof}
\end{theorem}

以下是一类特殊的加法局部化.

\begin{theorem}\label{thm: quotient category is localization}
    给定加法范畴 $\mathcal{A}$ 与加法全子范畴 $\mathcal{B}$, 下定义两种商.
    \begin{enumerate}
        \item (加法商). 定义 $\mathcal{A} / \mathcal{B}$ 为如下范畴: 对象同 $\mathcal{A}$; 对任意 $X$ 与 $Y$, 态射群为原始态射群的商群:
        \begin{equation}
            (X,Y)_{\mathcal{A} / \mathcal{B}} = (X,Y)_{\mathcal{A}} / \{\text{被 $\mathcal{B}$ 中对象分解的态射}\}.
        \end{equation}
        记函子 $R : \mathcal{A} \to A / \mathcal{B}, \quad X \mapsto RX = X, \quad f \mapsto Rf = [f]$.
        \item (GZ 局部化). 称 $(f;y)$ 是一对好态射, 若 $f : X \to Y$, $g : Y \to X$, 且 $1_Y - fg$ 与 $1_X - gf$ 都能经 $\mathcal{B}$ 中对象分解. 记 $S$ 是 $\mathcal{A}$ 中所有好态射构成的类. 记 GZ 局部化函子 $Q : \mathcal{A} \to \mathcal{A}[S_\mathcal{B}^{-1}]$.
    \end{enumerate}
    则存在范畴的同构 $\Phi$, 使得 $R = \Phi \circ Q$.
    \begin{proof}
        先说明 $Q$ 是加法函子. 依照\Cref{thm: when localization is additive}, 只需说明 $S$ 对直和封闭. 给定好态射 $(f;y) : X \to Y$ 与 $(f';y') : X' \to Y'$, 则显然 $(\binom{f \ \ 0}{0 \ \ f'}; \binom{y \ \ 0}{0 \ \ y'}) : X \oplus X' \to Y \oplus Y'$ 也是好态射.
        \\
        ($Q$ 被 $R$ 分解). 对任意 $X \in\mathcal{B}$, $QX$ 是零对象. 因此, 群同态 $(M,N)_\mathcal{A} \xrightarrow Q (M,N)_{\mathcal{A}[S^{-1}]}$ 经商群 $(M,N)_{\mathcal{A} / \mathcal{B}}$ 分解. 商群的泛性质决定了一族对应
        \begin{equation}
            \overline Q : \mathcal{A}/\mathcal{B} \to \mathcal{A}[S^{-1}],\quad X \mapsto X, \quad [f] \mapsto Qf.
        \end{equation}
        为了说明这是函子, 只需验证恒等律与复合律. 注意到 $[g \circ f] = [g] \circ [f]$, 且 $Q$ 是函子, 故
        \begin{equation}
            \overline Q([g] \circ [f]) = \overline Q([g \circ f]) = Q(g \circ f) = Qg \circ Qf = \overline Q([g]) \circ \overline Q([f]).
        \end{equation}
        \\
        ($R$ 被 $Q$ 分解). 给定一组好态射 $(f;g)$, 则 $[f] \circ [g] = [f \circ g]$ 与 $[g] \circ [f] = [g \circ f]$ 都是恒等态射. 由泛性质, $R$ 经 $Q$ 唯一地分解.
        \\
        以上两种分解都是唯一的 (由商群的泛性质和局部化的泛性质), 从而两函子在去向处相差一个同构. 
    \end{proof}
\end{theorem}

\subsection{Quillen 的同伦范畴}

此部分介绍左右同伦关系与 Quillen 的同伦范畴, 第一手资料是\cite{quillenHomotopicalAlgebra1967} 与\cite{quillenRationalHomotopyTheory1969}, 文献导读可参考\cite{hoveyModelCategories2007}.

\begin{definition}
    (闭区间). 闭区间即单纯形 $\Delta[1]$ 的某种``实现''. 常见的``实现''包含以下两种.
    \begin{enumerate}
        \item $|\Delta[1]| = [0,1]$ 是几何实现, 函子 $|\cdot| : \Delta \to k\mathbf{Top}$ 关于 $\Delta \rightarrowtail \mathbf{PSh}(\Delta)$ 延拓定义了``实现-单纯集化''伴随 (\cite{lurieKerodona}). 详见介绍单纯集方法的书籍.
        \item $\mathrm h (\Delta) = \text{``图范畴 $\{\cdot \to \cdot\}$''}$ 是同伦实现, 函子 $\mathrm h : \Delta \to \mathbf{Cat}$ 关于 $\Delta \rightarrowtail \mathbf{PSh}(\Delta)$ 延拓定义了``同伦-脉''伴随 (\cite{lurieKerodon}). 详见介绍无穷范畴的书籍.
    \end{enumerate}
    特别地, ``实现''保持单纯形的典范态射 $d^k : \Delta[0] \to \Delta[1]$ ($k \in \{0,1\}$) 与 $s^0 : \Delta[1] \to \Delta[0]$. 我们将五元组 $(\Delta[0], \Delta[1], s^0, d^0, d^1)$ 的实现定义作一个区间基本资料.
    \begin{enumerate}
        \item 区间 $I$ 是 $\Delta[1]$ 的实现, 终对象 $\top$ 是 $\Delta[0]$ 的实现 (左伴随保持终对象).
        \item $i_k : \top \to I$ 是 $d^k$ 的实现, 表示端点 $k \in \{0,1\}$ 的包含;
        \item $p : I \to \top$ 是 $s^0$ 的实现, 表示收缩 $I$ 到一个点.
    \end{enumerate}
\end{definition}

\begin{remark}
    以上两种``实现''分别对应拓扑语言与范畴语言. 与拓扑学相仿, 可以定义范畴中的柱对象与路对象, 也可以定义映射锥与映射柱等, 其元语言均取自单纯形.
\end{remark}

\begin{definition}
    (路对象, 积). 给定对象 $X$. $\mathbf{Cat}$ (或 $\mathbf{Top}$) 中的对角态射为下图态射链 $(\star)$:
    \begin{equation}
% https://q.uiver.app/#q=WzAsOCxbMCwxLCJYIFxccHJvZCAgWCJdLFsyLDEsIlheSSJdLFs0LDAsIlxcYnVsbGV0Il0sWzAsMCwiXFxidWxsZXQgXFxzcWN1cCBcXGJ1bGxldCJdLFsyLDAsIkkiXSxbNCwxLCJYIl0sWzUsMCwiKFxcc3RhcikiXSxbNSwxLCJYXnsoXFxzdGFyKX0iXSxbMyw0LCIoaV8wLCBcXCBpXzEpIl0sWzEsMCwiWF57KGlfMCwgXFwgaV8xKX0iLDJdLFs1LDEsIlhecCIsMl0sWzQsMiwicCJdLFs1LDAsIlxcRGVsdGFfWCIsMCx7Im9mZnNldCI6LTIsImN1cnZlIjotMn1dLFs2LDcsIiIsMCx7ImxldmVsIjoyfV1d
\begin{tikzcd}[ampersand replacement=\&]
	{\bullet \sqcup \bullet} \&\& I \&\& \bullet \& {(\star)} \\[-6pt]
	{X \prod  X} \&\& {X^I} \&\& X \& {X^{(\star)}}
	\arrow["{(i_0, \ i_1)}", from=1-1, to=1-3]
	\arrow["p", from=1-3, to=1-5]
	\arrow[Rightarrow, from=1-6, to=2-6]
	\arrow["{X^{(i_0, \ i_1)}}"', from=2-3, to=2-1]
	\arrow["{\Delta_X}", shift left=2, curve={height=-12pt}, from=2-5, to=2-1]
	\arrow["{X^p}"', from=2-5, to=2-3]
\end{tikzcd}.
    \end{equation}
    对 $(\star)$ 作用 $X^{\cdot } = \mathrm{Funct}(\cdot , X)$, 得相应的函子与自然变换. 特别地, $\Delta_X$ 是自然变换. 以上操作是在 $\mathrm{Top}$ (或 $\mathrm{Cat}$) 中进行的, 所有函子与自然变换都视作 $\mathrm{Top}$ (或 $\mathrm{Cat}$) 中对象与态射.
\end{definition}

\begin{definition}
    (柱对象, 余积). 给定对象 $X$. $\mathbf{Cat}$ (或 $\mathbf{Top}$) 中的余对角态射为下图态射链 $X \prod (\star)$:
    \begin{equation}\label{eq: cylinder object}
        % https://q.uiver.app/#q=WzAsOCxbMCwxLCJYIFxcY29wcm9kICBYIl0sWzIsMSwiWCBcXHByb2QgSSJdLFs0LDAsIlxcYnVsbGV0Il0sWzAsMCwiXFxidWxsZXQgXFxzcWN1cCBcXGJ1bGxldCJdLFsyLDAsIkkiXSxbNCwxLCJYIl0sWzUsMCwiKFxcc3RhcikiXSxbNSwxLCJYIFxccHJvZCB7KFxcc3Rhcil9Il0sWzMsNCwiKGlfMCwgXFwgaV8xKSJdLFswLDEsIlggXFxwcm9kIHsoaV8wLCBcXCBpXzEpfSJdLFsxLDUsIlggXFxwcm9kIHAiXSxbNCwyLCJwIl0sWzAsNSwiXFxuYWJsYV9YIiwyLHsib2Zmc2V0IjoyLCJjdXJ2ZSI6Mn1dLFs2LDcsIiIsMCx7ImxldmVsIjoyfV1d
\begin{tikzcd}[ampersand replacement=\&]
	{\bullet \sqcup \bullet} \&\& I \&\& \bullet \& {(\star)} \\[-6pt]
	{X \coprod  X} \&\& {X \prod I} \&\& X \& {X \prod {(\star)}}
	\arrow["{(i_0, \ i_1)}", from=1-1, to=1-3]
	\arrow["p", from=1-3, to=1-5]
	\arrow[Rightarrow, from=1-6, to=2-6]
	\arrow["{X \prod {(i_0, \ i_1)}}", from=2-1, to=2-3]
	\arrow["{\nabla_X}"', shift right=2, curve={height=12pt}, from=2-1, to=2-5]
	\arrow["{X \prod p}", from=2-3, to=2-5]
\end{tikzcd}.
    \end{equation}
\end{definition}

\begin{lemma}
    $\Delta_X$ 是两组 $1_X$ 关于积泛性质诱导的态射; $\nabla_X$ 是两组 $1_X$ 关于余积泛性质诱导的态射.
    \begin{proof}
        以前者为例. 考虑单纯形范畴中一组复合为恒等的态射
        \begin{equation}
            \Delta[0] \xrightarrow{d^k} \Delta[1] \xrightarrow{s^0} \Delta[0] \quad (k = 0,1).
        \end{equation}
        经``实现''函子, 得 $p \circ i_k$ 是恒等, 从而 $X^{i_k}\circ X^p$ 是恒等的自然变换. $\{X^{i_k} :X^I \to X \prod X\}_{k  = 0,1}$ 由泛性质诱导的态射是 $X^{(i_0, i_1)} : X^I \to X$. 若复合 $X^p$, 则 $\{1_X = X^{i_k} \circ X^p :X \to X^I \to X \prod X\}_{k  = 0,1}$ 诱导的态射是 $\Delta_X$. 对 $\nabla _X$ 的证明类似.
    \end{proof} 
\end{lemma}

我们通过路对象与柱对象定义同伦关系, 以此描述态射类的某种等价关系. 柱同伦 (左同伦) 的拓扑学视角广为数学工作者所知.

\begin{definition}
    (拓扑学的同伦). 称拓扑空间间的一对连续映射 $f, g : X \to Y$ 是左同伦的, 若 $\operatorname{im}(f) \cup \operatorname{im} (g)\subseteq Y$ 所示的上下柱体能被连续地填补. 即存在连续映射 $H : X \times [0,1] \to Y$, 使得 $H(x,0) = f(x)$ 且 $H(x,1) = g(x)$ 对任意 $x \in X$ 成立. 称 $H$ 为从 $f$ 到 $g$ 的同伦.
    \\
    换言之, 下图交换 (左图与右图是等价的):
    \begin{equation}\label{eq: topological homotopy}
% https://q.uiver.app/#q=WzAsOCxbMSwyLCJYIl0sWzIsMSwiWSJdLFsxLDAsIlgiXSxbMCwxLCJYIFxccHJvZCBJIl0sWzQsMCwiWCBcXGNvcHJvZCBYIl0sWzQsMiwiWCJdLFs2LDAsIlkiXSxbNiwyLCJYIFxccHJvZCBJIl0sWzAsMywiWCBcXHByb2QgKGlfMCkiLDAseyJjb2xvdXIiOlsyMzUsMTAwLDYwXX0sWzIzNSwxMDAsNjAsMV1dLFsyLDMsIlggXFxwcm9kIChpXzEpIiwyLHsiY29sb3VyIjpbMjM1LDEwMCw2MF19LFsyMzUsMTAwLDYwLDFdXSxbMCwxLCJmIiwyLHsiY29sb3VyIjpbMzU2LDEwMCw2MF19LFszNTYsMTAwLDYwLDFdXSxbMiwxLCJnIiwwLHsiY29sb3VyIjpbMzU2LDEwMCw2MF19LFszNTYsMTAwLDYwLDFdXSxbMywxLCJIIiwwLHsiY29sb3VyIjpbMzU2LDEwMCw2MF0sInN0eWxlIjp7ImJvZHkiOnsibmFtZSI6ImRhc2hlZCJ9fX0sWzM1NiwxMDAsNjAsMV1dLFs0LDYsIihmLGcpIiwwLHsiY29sb3VyIjpbMzU2LDEwMCw2MF19LFszNTYsMTAwLDYwLDFdXSxbNyw2LCJIIiwyLHsiY29sb3VyIjpbMzU2LDEwMCw2MF0sInN0eWxlIjp7ImJvZHkiOnsibmFtZSI6ImRhc2hlZCJ9fX0sWzM1NiwxMDAsNjAsMV1dLFs0LDcsIihYIFxccHJvZCAoaV8wKSwgWCBcXHByb2QgKGlfMSkpIiwxLHsiY29sb3VyIjpbMjM1LDEwMCw2MF19LFsyMzUsMTAwLDYwLDFdXSxbNyw1LCJYIFxccHJvZCBwIl0sWzQsNSwiXFxuYWJsYV9YIiwyXV0=
\begin{tikzcd}
	& X &&& {X \coprod X} && Y \\
	{X \prod I} && Y \\
	& X &&& X && {X \prod I}
	\arrow["{X \prod (i_1)}"', color={rgb,255:red,51;green,68;blue,255}, from=1-2, to=2-1]
	\arrow["g", color={rgb,255:red,255;green,51;blue,65}, from=1-2, to=2-3]
	\arrow["{(f,g)}", color={rgb,255:red,255;green,51;blue,65}, from=1-5, to=1-7]
	\arrow["{\nabla_X}"', from=1-5, to=3-5]
	\arrow["{(X \prod (i_0), X \prod (i_1))}"{description}, color={rgb,255:red,51;green,68;blue,255}, from=1-5, to=3-7]
	\arrow["H", color={rgb,255:red,255;green,51;blue,65}, dashed, from=2-1, to=2-3]
	\arrow["{X \prod (i_0)}", color={rgb,255:red,51;green,68;blue,255}, from=3-2, to=2-1]
	\arrow["f"', color={rgb,255:red,255;green,51;blue,65}, from=3-2, to=2-3]
	\arrow["H"', color={rgb,255:red,255;green,51;blue,65}, dashed, from=3-7, to=1-7]
	\arrow["{X \prod p}", from=3-7, to=3-5]
\end{tikzcd}.
    \end{equation}
\end{definition}

以下是模型结构中左同伦的定义. 

\begin{definition}\label{item: finite limits and colimits}
    以下假定 $\mathcal{A}$ 是带有模型结构 $(\mathsf{Cofib}, \mathsf{Weq}, \mathsf{Fib})$ 的范畴, 满足以下三点:
\begin{enumerate}
    \item 范畴有有限积与有限余积;
    \item 给定任意平凡余纤维 $p : A \to C$ 与任意态射 $f : A \to B$, 则存在弱推出使得 $p' : B \to D$ 是平凡余纤维;
    \item 给定任意平凡纤维 $i : X \to Z$ 与任意态射 $g :  Y \to Z$, 则存在弱推出使得 $i' : W \to Y$ 是平凡纤维.
\end{enumerate}
\begin{equation}\label{eq: quillen theorem diagram}
        % https://q.uiver.app/#q=WzAsOCxbMSwwLCJYIl0sWzEsMSwiWiJdLFswLDEsIlkiXSxbMCwwLCJXIl0sWzQsMCwiQSJdLFs1LDAsIkMiXSxbNSwxLCJEIl0sWzQsMSwiQiJdLFsyLDEsImYiLDJdLFszLDIsIlxcbWF0aHNmeyhUKUZpYn0iLDIseyJzdHlsZSI6eyJib2R5Ijp7Im5hbWUiOiJkYXNoZWQifX19XSxbMywwLCJmJyIsMCx7InN0eWxlIjp7ImJvZHkiOnsibmFtZSI6ImRhc2hlZCJ9fX1dLFs0LDUsImciXSxbNSw2LCJcXG1hdGhzZnsoVClDb2ZpYn0iLDAseyJzdHlsZSI6eyJib2R5Ijp7Im5hbWUiOiJkYXNoZWQifX19XSxbNCw3LCJcXG1hdGhzZnsoVClDb2ZpYn0iLDJdLFs3LDYsImcnIiwyLHsic3R5bGUiOnsiYm9keSI6eyJuYW1lIjoiZGFzaGVkIn19fV0sWzAsMSwiXFxtYXRoc2Z7KFQpRmlifSJdLFswLDEsInAiLDIseyJzdHlsZSI6eyJib2R5Ijp7Im5hbWUiOiJub25lIn0sImhlYWQiOnsibmFtZSI6Im5vbmUifX19XSxbMywyLCJwJyIsMCx7InN0eWxlIjp7ImJvZHkiOnsibmFtZSI6Im5vbmUifSwiaGVhZCI6eyJuYW1lIjoibm9uZSJ9fX1dLFs0LDcsImkiLDAseyJzdHlsZSI6eyJib2R5Ijp7Im5hbWUiOiJub25lIn0sImhlYWQiOnsibmFtZSI6Im5vbmUifX19XSxbNSw2LCJpJyIsMix7InN0eWxlIjp7ImJvZHkiOnsibmFtZSI6Im5vbmUifSwiaGVhZCI6eyJuYW1lIjoibm9uZSJ9fX1dXQ==
\begin{tikzcd}[ampersand replacement=\&]
	W \& X \&\&\& A \& C \\
	Y \& Z \&\&\& B \& D
	\arrow["{f'}", dashed, from=1-1, to=1-2]
	\arrow["{\mathsf{Fib}}"', dashed, from=1-1, to=2-1]
	\arrow["{p'}", draw=none, from=1-1, to=2-1]
	\arrow["{\mathsf{Fib}}", from=1-2, to=2-2]
	\arrow["p"', draw=none, from=1-2, to=2-2]
	\arrow["g", from=1-5, to=1-6]
	\arrow["{\mathsf{Cofib}}"', from=1-5, to=2-5]
	\arrow["i", draw=none, from=1-5, to=2-5]
	\arrow["{\mathsf{Cofib}}", dashed, from=1-6, to=2-6]
	\arrow["{i'}"', draw=none, from=1-6, to=2-6]
	\arrow["f"', from=2-1, to=2-2]
	\arrow["{g'}"', dashed, from=2-5, to=2-6]
\end{tikzcd}.
\end{equation}

记 $\prod$ 与 $\coprod$ 分别为 $\mathcal{A}$ 中的二元积运算与二元余积运算; 零元积即始对象 $\bot$, 零元余积即终对象 $\top$. 出于习惯, 我们将 $X \coprod Y$ 与 $X \prod Y$ 视作列向量, 态射仍以矩阵形式表示.
\end{definition}

\begin{definition}\label{def: cylinder object and homotopy}
    (柱对象, 柱同伦). 如\Cref{eq: topological homotopy} 右图所示, 右上三角是柱同伦关系的的定义式, 左下角是余对角态射的定义 (\Cref{eq: cylinder object}). 由于单纯形范畴到通常范畴未必有``实现''函子, 我们难以定义 $X \prod I$; 一个解决方法是将 $X \prod I$ 换作\textbf{柱对象} $\mathrm{Cyl}(X)$:
    \begin{equation}
% https://q.uiver.app/#q=WzAsNyxbMCwwLCJYIFxcY29wcm9kIFgiXSxbMCwyLCJYIl0sWzIsMiwiWCBcXHByb2QgSSJdLFs0LDAsIlggXFxjb3Byb2QgWCJdLFs0LDIsIlgiXSxbNiwyLCJcXG1hdGhybXtDeWx9KFgpIl0sWzMsMSwiXFxpbXBsaWVzIl0sWzAsMiwiKFggXFxwcm9kIChpXzApLCBYIFxccHJvZCAoaV8xKSkiLDAseyJjb2xvdXIiOlsyMzUsMTAwLDYwXX0sWzIzNSwxMDAsNjAsMV1dLFsyLDEsIlggXFxwcm9kIHAiXSxbMCwxLCJcXG5hYmxhX1giLDJdLFs1LDQsIlxcbWF0aHNme1dlcX0iXSxbMyw0LCJcXGJpbm9tIDExIiwyXSxbMyw1LCJcXG1hdGhzZntDb2ZpYn0iLDAseyJjb2xvdXIiOlsyMzUsMTAwLDYwXX0sWzIzNSwxMDAsNjAsMV1dLFszLDUsIihcXHBhcnRpYWxfMCwgXFxwYXJ0aWFsIF8xKSIsMix7ImNvbG91ciI6WzIzNSwxMDAsNjBdLCJzdHlsZSI6eyJib2R5Ijp7Im5hbWUiOiJub25lIn0sImhlYWQiOnsibmFtZSI6Im5vbmUifX19LFsyMzUsMTAwLDYwLDFdXSxbNSw0LCJcXHNpZ21hICIsMix7InN0eWxlIjp7ImJvZHkiOnsibmFtZSI6Im5vbmUifSwiaGVhZCI6eyJuYW1lIjoibm9uZSJ9fX1dXQ==
\begin{tikzcd}
	{X \coprod X} &&&& {X \coprod X} \\
	&&& \implies \\
	X && {X \prod I} && X && {\mathrm{Cyl}(X)}
	\arrow["{\nabla_X}"', from=1-1, to=3-1]
	\arrow["{(X \prod (i_0), X \prod (i_1))}", color={rgb,255:red,51;green,68;blue,255}, from=1-1, to=3-3]
	\arrow["{\binom 11}"', from=1-5, to=3-5]
	\arrow["{\mathsf{Cofib}}", color={rgb,255:red,51;green,68;blue,255}, from=1-5, to=3-7]
	\arrow["{(\partial_0, \partial _1)}"', color={rgb,255:red,51;green,68;blue,255}, draw=none, from=1-5, to=3-7]
	\arrow["{X \prod p}", from=3-3, to=3-1]
	\arrow["{\mathsf{Weq}}", from=3-7, to=3-5]
	\arrow["{\sigma }"', draw=none, from=3-7, to=3-5]
\end{tikzcd}.
    \end{equation}
    称 $(\mathrm{Cyl}(X), \partial _0, \partial _1, \sigma, X)$ 是 $X$ 对应的柱对象, 若存在余纤维 $(\partial _0 , \partial _1) : X \coprod X \to \mathrm{Cyl}(X)$ 与弱等价 $\sigma : \mathrm{Cyl}(X) \to X$, 使得上图交换. 给定态射 $f, g : X \to Y$, 称 $f$ 与 $g$ 是\textbf{柱同伦}的, 若存在态射 $h : \mathrm{Cyl}(X) \to Y$, 使得下图交换:
    \begin{equation}\label{eq: cylinder homotopy}
        % https://q.uiver.app/#q=WzAsNCxbMCwwLCJYIFxcY29wcm9kIFgiXSxbMCwyLCJYIl0sWzIsMiwiXFxtYXRocm17Q3lsfShYKSJdLFsyLDAsIlkgIl0sWzIsMSwiXFxtYXRoc2Z7V2VxfSJdLFswLDEsIlxcYmlub20gMTEiLDJdLFswLDIsIlxcbWF0aHNme0NvZmlifSIsMCx7ImNvbG91ciI6WzIzNSwxMDAsNjBdfSxbMjM1LDEwMCw2MCwxXV0sWzAsMiwiKFxccGFydGlhbF8wLCBcXHBhcnRpYWwgXzEpIiwyLHsiY29sb3VyIjpbMjM1LDEwMCw2MF0sInN0eWxlIjp7ImJvZHkiOnsibmFtZSI6Im5vbmUifSwiaGVhZCI6eyJuYW1lIjoibm9uZSJ9fX0sWzIzNSwxMDAsNjAsMV1dLFsyLDEsIlxcc2lnbWEgIiwyLHsic3R5bGUiOnsiYm9keSI6eyJuYW1lIjoibm9uZSJ9LCJoZWFkIjp7Im5hbWUiOiJub25lIn19fV0sWzAsMywiKGYsZykiLDAseyJjb2xvdXIiOlszNTksMTAwLDYwXX0sWzM1OSwxMDAsNjAsMV1dLFsyLDMsImgiLDIseyJjb2xvdXIiOlszNTksMTAwLDYwXSwic3R5bGUiOnsiYm9keSI6eyJuYW1lIjoiZGFzaGVkIn19fSxbMzU5LDEwMCw2MCwxXV1d
\begin{tikzcd}
	{X \coprod X} && {Y } \\
	\\
	X && {\mathrm{Cyl}(X)}
	\arrow["{(f,g)}", color={rgb,255:red,255;green,51;blue,54}, from=1-1, to=1-3]
	\arrow["{\binom 11}"', from=1-1, to=3-1]
	\arrow["{\mathsf{Cofib}}", color={rgb,255:red,51;green,68;blue,255}, from=1-1, to=3-3]
	\arrow["{(\partial_0, \partial _1)}"', color={rgb,255:red,51;green,68;blue,255}, draw=none, from=1-1, to=3-3]
	\arrow["h"', color={rgb,255:red,255;green,51;blue,54}, dashed, from=3-3, to=1-3]
	\arrow["{\mathsf{Weq}}", from=3-3, to=3-1]
	\arrow["{\sigma }"', draw=none, from=3-3, to=3-1]
\end{tikzcd}.
    \end{equation}

\end{definition}

\begin{remark}\label{rmk: cofibration and weak equivalence}
    $(\partial _0, \partial _1)$ 是余纤维, CM1 说明 $\partial_0$ 与 $\partial_1$ 是弱等价. 通常无法断言 $(\partial _0, \partial _1)$ 是弱等价, 或 $\partial_k$ 是余纤维.
\end{remark}

\begin{remark}
    拓扑视角看, $f : X \to Y$ 建立了弱等价, 当且仅当其诱导的 $\pi_k(f)$ 群均是同构 (假定连通). 对 $\sigma$, 将``圆柱'' $\mathrm{Cyl}(X)$ 沿母线收缩至``任意横截面'' $X$, 其拓扑量 $\pi_\bullet$-群自然被保持.
\end{remark}

\begin{definition}
    (路对象, 路同伦). 可以类似定义路对象 $\mathrm{Path}(-)$ 与路同伦的定义与柱对象与柱同伦类似. 此处仅给出定义路同伦的交换图:
    \begin{equation}
        % https://q.uiver.app/#q=WzAsNCxbMiwyLCJZIFxccHJvZCBZIl0sWzIsMCwiWSJdLFswLDAsIlxcbWF0aHJte1BhdGh9KFkpIl0sWzAsMiwiWSAiXSxbMSwyLCJcXG1hdGhzZntXZXF9IiwyXSxbMSwwLCJcXGJpbm9tIDExIl0sWzIsMCwiXFxtYXRoc2Z7RmlifSIsMix7ImNvbG91ciI6WzIzNSwxMDAsNjBdfSxbMjM1LDEwMCw2MCwxXV0sWzIsMCwiXFxiaW5vbSB7ZF8wfXtkXzF9IiwwLHsiY29sb3VyIjpbMjM1LDEwMCw2MF0sInN0eWxlIjp7ImJvZHkiOnsibmFtZSI6Im5vbmUifSwiaGVhZCI6eyJuYW1lIjoibm9uZSJ9fX0sWzIzNSwxMDAsNjAsMV1dLFsxLDIsInMiLDAseyJzdHlsZSI6eyJib2R5Ijp7Im5hbWUiOiJub25lIn0sImhlYWQiOnsibmFtZSI6Im5vbmUifX19XSxbMywwLCJcXGJpbm9tIGZnIiwyLHsiY29sb3VyIjpbMzU5LDEwMCw2MF19LFszNTksMTAwLDYwLDFdXSxbMywyLCJrIiwwLHsiY29sb3VyIjpbMzU5LDEwMCw2MF0sInN0eWxlIjp7ImJvZHkiOnsibmFtZSI6ImRhc2hlZCJ9fX0sWzM1OSwxMDAsNjAsMV1dXQ==
\begin{tikzcd}
	{\mathrm{Path}(Y)} && Y \\
	\\
	{X } && {Y \prod Y}
	\arrow["{\mathsf{Fib}}"', color={rgb,255:red,51;green,68;blue,255}, from=1-1, to=3-3]
	\arrow["{\binom {d_0}{d_1}}", color={rgb,255:red,51;green,68;blue,255}, draw=none, from=1-1, to=3-3]
	\arrow["{\mathsf{Weq}}"', from=1-3, to=1-1]
	\arrow["s", draw=none, from=1-3, to=1-1]
	\arrow["{\binom 11}", from=1-3, to=3-3]
	\arrow["k", color={rgb,255:red,255;green,51;blue,54}, dashed, from=3-1, to=1-1]
	\arrow["{\binom fg}"', color={rgb,255:red,255;green,51;blue,54}, from=3-1, to=3-3]
\end{tikzcd}.
    \end{equation}
\end{definition}

\begin{remark}\label{rmk: fibration and weak equivalence}
    $\binom{d_0}{d_1}$ 是纤维, CM1 说明 $d_0$ 与 $d_1$ 是弱等价. 通常无法断言 $\binom{d_0}{d_1}$ 是弱等价, 或 $d_k$ 是纤维.
\end{remark}

\begin{remark}
    从拓扑视角看, 路对象 $Y^I$ 即``闭区间 $[0,1]$ 至 $Y$ 的所有态射''构成的空间, 配有紧-开拓扑. $s : Y \to Y^I$ 无非令所有道路 ($Y^I$ 中对象) 连续地坍缩到道路起点. 其拓扑量 $\pi_\bullet$-群自然被保持.
\end{remark}

\begin{remark}
    Steenrod 方便拓扑空间 (\cite{steenrodConvenientCategoryTopological1967}) 与 $\mathbf{Cat}$ 存在闭合幺半结构. 因此
    \begin{equation}
        (\mathrm{Cyl}(X), Y) = (X \prod I, Y) \cong (X, Y^I) = (X, \mathrm{Path}(Y)).
    \end{equation}
    假定伴随函子保持某些态射类, 则无需区分路同伦与柱同伦.
\end{remark}

\begin{example}\label{ex: chain homotopy}
    $\mathbf{Mod}_R$ 中链复形的一组同伦关系 $s : f \sim g : X \to Y$ ($f-g = sd+ds$) 对应态射 $(s,f,g) : \mathrm{Cyl}(X) \to Y$. 可以检验, $f$ 与 $g$ 是链映射且 $f - g = sd+ds$, 当且仅当下式成立:
    \begin{equation}
        (s,f,g) \circ \begin{pmatrix}
            -d_X & 0 & 0 \\
            1 & d_X & 0 \\
            -1 & 0 & d_X
        \end{pmatrix} = d_Y \circ (s,f,g).
    \end{equation}
    特别地, $\mathrm{Cyl}(R)$ 是单纯形 $\Delta[1]$ 在复形范畴中的实现, $\mathrm{Cyl}(X) = X \otimes \mathrm{Cyl}(R)$. 相应地,
    \begin{equation}
        \mathrm{Path}(Y) = \mathcal{HOM}(\mathrm{Cyl}(R), Y).
    \end{equation}
    链复形的同伦无需区分左右.
\end{example}

\begin{remark}
    从态射视角看, \Cref{eq: cylinder homotopy} 本质上将 $\binom 11$ 分解作余纤维和弱等价的复合, 使得 $(f,g)$ 能被余纤维态射分解. 我们并不关心 $\mathrm{Cyl}(X)$ 处的具体对象, 而是侧重 $\binom 11$ 的 $\mathsf{Weq} \circ \mathsf{Cofib}$-分解.
    \\
    通常范畴中, 未必存在具体的函子 $\mathrm{Cyl}$ 或 $\mathrm{Path}$. 往后将弃用 $\mathrm{Cyl}(-)$ 与 $\mathrm{Path}(-)$ 两个记号.
\end{remark}

\begin{lemma}\label{lem: equivalent definition of homotopy}
    (柱同伦的等价定义) 使用模型结构的性质, \Cref{def: cylinder object and homotopy} 中的题设 $(\partial_0, \partial _1) \in \mathsf{Cofib}$ 可以删去.
    \begin{proof}
        若将 $(\partial_0, \partial)$ 取作任意态射, 依照 CM4 将其分解作 $p \circ i \in \mathsf{TFib} \circ \mathsf{Cofib}$. 分别使用 $i$, $\sigma \circ p$ 与 $h \circ p$ 替换 $(\partial_0, \partial _1)$, $\sigma$ 与 $h$ 即可. 新的交换图满足\Cref{def: cylinder object and homotopy} 中假定.
    \end{proof}
\end{lemma}

\begin{lemma}\label{lem: equivalent definition of path homotopy}
    (路同伦的等价定义) 使用模型结构的性质, 定义路同伦时, 可以删去题设 $\binom{d_0}{d_1} \in \mathsf{Fib}$.
\end{lemma}

\begin{proposition}\label{prop: homotopy preserved by composition}
    向前复合 $- \circ \varphi$ 保持路同伦关系; 向后复合 $\psi \circ -$ 保持柱同伦关系.
    \begin{proof}
        以前者为例. 若 $(f,g)$ 通过 $h$ 实现柱同伦, 则 $(\psi \circ f, \psi \circ g)$ 通过 $\psi \circ h$ 实现柱同伦.
    \end{proof}
\end{proposition}

\begin{theorem}\label{thm: homotopy is equivalence relation}
    (余纤维对象出发的柱同伦). 选定 $\mathrm{Hom}(X,Y)$, 其中 $X \in \mathcal{C}$ (余纤维对象).
    \begin{enumerate}
        \item \Cref{def: cylinder object and homotopy} 中, 纤维态射 $(\partial_0, \partial _1)$ 的分量 $\partial_0$ 与 $\partial_1$ 均是平凡余纤维;
        \item 柱同伦是 $\mathrm{Hom}(X,Y)$ 上的等价关系.
    \end{enumerate}
    \begin{proof}
        (1). $\bot : \bot \to X$ 是余纤维, 推出所得的 $\bot \coprod X: \bot \sqcup X \to X \coprod X$ 也是余纤维. 复合得余纤维
        \begin{equation}
            % https://q.uiver.app/#q=WzAsNSxbMiwwLCJcXGJvdCBcXHNxY3VwIFgiXSxbNCwwLCJYIFxcc3FjdXAgWCJdLFsxLDAsIlgiXSxbNiwwLCJYJyJdLFswLDAsIlxccGFydGlhbCBfMCA6Il0sWzAsMSwiXFxib3QgXFxzcWN1cCBYIl0sWzIsMCwiZV8xIl0sWzEsMywiKFxccGFydGlhbCBfMCwgXFxwYXJ0aWFsIF8xKSJdXQ==
\begin{tikzcd}
	{\partial _k :} & X & {\bot \sqcup X} && {X \sqcup X} && {X'}
	\arrow["{e_k}", from=1-2, to=1-3]
	\arrow["{\bot \sqcup X}", from=1-3, to=1-5]
	\arrow["{(\partial _0, \partial _1)}", from=1-5, to=1-7]
\end{tikzcd}.
        \end{equation}
        \Cref{rmk: cofibration and weak equivalence} 说明 $\partial_k$ 是弱等价, 从而是平凡余纤维.
        \\
        (2). 自反性与对称性显然, 下证明传递性. 假定有以下柱同伦
        \begin{equation}\label{eq: transitivity of cylinder homotopy}
            % https://q.uiver.app/#q=WzAsOCxbMCwwLCJYIFxcY29wcm9kIFgiXSxbMSwwLCJZIl0sWzAsMSwiWCJdLFsxLDEsIlhfMSJdLFsyLDAsIlggXFxjb3Byb2QgWCJdLFsyLDEsIlgiXSxbMywxLCJYXzIiXSxbMywwLCJZIl0sWzAsMSwieyhnLGYpfSIsMCx7ImNvbG91ciI6Wy00LDEwMCw2MF19LFstNCwxMDAsNjAsMV1dLFswLDIsIntcXG5hYmxhX1h9IiwyXSxbMCwzLCIoYSxiKSIsMSx7ImNvbG91ciI6WzIzNSwxMDAsNjBdfSxbMjM1LDEwMCw2MCwxXV0sWzMsMSwiaCIsMix7ImNvbG91ciI6Wy00LDEwMCw2MF19LFstNCwxMDAsNjAsMV1dLFszLDIsIlxcc2lnbWEgXzEiXSxbNCw1LCJcXG5hYmxhIF9YIiwyXSxbNiw1LCJcXHNpZ21hIF8yIl0sWzQsNywiKGwsZikiLDAseyJjb2xvdXIiOlstNCwxMDAsNjBdfSxbLTQsMTAwLDYwLDFdXSxbNiw3LCJqIiwyLHsiY29sb3VyIjpbLTQsMTAwLDYwXX0sWy00LDEwMCw2MCwxXV0sWzQsNiwiKGMsZCkiLDEseyJjb2xvdXIiOlsyMzUsMTAwLDYwXX0sWzIzNSwxMDAsNjAsMV1dXQ==
\begin{tikzcd}
	{X \coprod X} & Y & {X \coprod X} & Y \\
	X & {X_1} & X & {X_2}
	\arrow["{{(g,f)}}", color={rgb,255:red,255;green,51;blue,65}, from=1-1, to=1-2]
	\arrow["{{\nabla_X}}"', from=1-1, to=2-1]
	\arrow["{(a,b)}"{description}, color={rgb,255:red,51;green,68;blue,255}, from=1-1, to=2-2]
	\arrow["{(l,f)}", color={rgb,255:red,255;green,51;blue,65}, from=1-3, to=1-4]
	\arrow["{\nabla _X}"', from=1-3, to=2-3]
	\arrow["{(c,d)}"{description}, color={rgb,255:red,51;green,68;blue,255}, from=1-3, to=2-4]
	\arrow["h"', color={rgb,255:red,255;green,51;blue,65}, from=2-2, to=1-2]
	\arrow["{\sigma _1}", from=2-2, to=2-1]
	\arrow["j"', color={rgb,255:red,255;green,51;blue,65}, from=2-4, to=1-4]
	\arrow["{\sigma _2}", from=2-4, to=2-3]
\end{tikzcd}.
        \end{equation}
        将\Cref{eq: transitivity of cylinder homotopy} 两图右上方的三角分别拆分作交换图, 再拼接作\Cref{eq: pushout for transitivity of cylinder homotopy} 左上方图. 依照\Cref{item: finite limits and colimits}, 作\Cref{eq: pushout for transitivity of cylinder homotopy} 左上方图中的弱推出方块 $\square$, 并诱导相应的态射 $\alpha$ 与 $\beta$ (\Cref{eq: pushout for transitivity of cylinder homotopy} 下行两图). 最终检验\Cref{eq: pushout for transitivity of cylinder homotopy} 右上方图的交换性即可.
        \begin{equation}\label{eq: pushout for transitivity of cylinder homotopy}
            % https://q.uiver.app/#q=WzAsMjEsWzEsMSwiWCJdLFsxLDAsIlkiXSxbMCwwLCJYIl0sWzAsMSwiWF8xIl0sWzMsMCwiWCBcXGNvcHJvZCBYIl0sWzMsMiwiWCJdLFs1LDAsIlkiXSxbNSwyLCJYJyJdLFsyLDEsIlhfMiJdLFsyLDAsIlgiXSxbMSwyLCJYJyJdLFswLDMsIlgiXSxbMSwzLCJYXzEiXSxbMCw0LCJYXzIiXSxbMSw0LCJYJyJdLFszLDMsIlgiXSxbNCwzLCJYXzEiXSxbMyw0LCJYXzIiXSxbNCw0LCJYJyJdLFsyLDUsIlkiXSxbNSw1LCJYIl0sWzAsMywiYiIsMCx7ImNvbG91ciI6WzIzNSwxMDAsNjBdfSxbMjM1LDEwMCw2MCwxXV0sWzIsMywiYSIsMix7ImNvbG91ciI6WzIzNSwxMDAsNjBdfSxbMjM1LDEwMCw2MCwxXV0sWzAsMSwiZiIsMCx7ImNvbG91ciI6WzM1NiwxMDAsNjBdfSxbMzU2LDEwMCw2MCwxXV0sWzIsMSwiZyIsMCx7ImNvbG91ciI6WzM1NiwxMDAsNjBdfSxbMzU2LDEwMCw2MCwxXV0sWzMsMSwiaCIsMCx7ImNvbG91ciI6WzM1NiwxMDAsNjBdfSxbMzU2LDEwMCw2MCwxXV0sWzQsNiwiKGcsbCkiLDAseyJjb2xvdXIiOlszNTYsMTAwLDYwXX0sWzM1NiwxMDAsNjAsMV1dLFs3LDYsIlxcYWxwaGEgIiwyLHsiY29sb3VyIjpbMzU2LDEwMCw2MF0sInN0eWxlIjp7ImJvZHkiOnsibmFtZSI6ImRhc2hlZCJ9fX0sWzM1NiwxMDAsNjAsMV1dLFs0LDcsIihjJ2EsIGInZCkiLDEseyJjb2xvdXIiOlsyMzUsMTAwLDYwXX0sWzIzNSwxMDAsNjAsMV1dLFs3LDUsIlxcYmV0YSIsMCx7InN0eWxlIjp7ImJvZHkiOnsibmFtZSI6ImRhc2hlZCJ9fX1dLFs0LDUsIlxcbmFibGFfWCIsMl0sWzksMSwibCIsMix7ImNvbG91ciI6WzM1NiwxMDAsNjBdfSxbMzU2LDEwMCw2MCwxXV0sWzAsOCwiYyIsMix7ImNvbG91ciI6WzIzNSwxMDAsNjBdfSxbMjM1LDEwMCw2MCwxXV0sWzksOCwiZCIsMCx7ImNvbG91ciI6WzIzNSwxMDAsNjBdfSxbMjM1LDEwMCw2MCwxXV0sWzgsMSwiaiIsMix7ImNvbG91ciI6WzM1NiwxMDAsNjBdfSxbMzU2LDEwMCw2MCwxXV0sWzMsMTAsImMnIiwyLHsiY3VydmUiOjIsInN0eWxlIjp7ImJvZHkiOnsibmFtZSI6ImRhc2hlZCJ9fX1dLFs4LDEwLCJiJyIsMCx7ImN1cnZlIjotMiwic3R5bGUiOnsiYm9keSI6eyJuYW1lIjoiZGFzaGVkIn19fV0sWzAsMTAsIlxcc3F1YXJlIiwxLHsic3R5bGUiOnsiYm9keSI6eyJuYW1lIjoibm9uZSJ9LCJoZWFkIjp7Im5hbWUiOiJub25lIn19fV0sWzExLDEzLCJjIiwyXSxbMTMsMTQsImInIiwyXSxbMTIsMTQsImMnIl0sWzE1LDE2LCJiIl0sWzE1LDE3LCJjIiwyXSxbMTcsMTgsImInIiwyXSxbMTYsMTgsImMnIl0sWzEyLDE5LCJoIiwwLHsiY3VydmUiOi0yfV0sWzEzLDE5LCJqIiwyLHsiY3VydmUiOjJ9XSxbMTQsMTksIlxcYWxwaGEiLDEseyJjb2xvdXIiOlszNTYsMTAwLDYwXSwic3R5bGUiOnsiYm9keSI6eyJuYW1lIjoiZGFzaGVkIn19fSxbMzU2LDEwMCw2MCwxXV0sWzExLDEyLCJiIl0sWzExLDE0LCJcXHNxdWFyZSIsMSx7InN0eWxlIjp7ImJvZHkiOnsibmFtZSI6Im5vbmUifSwiaGVhZCI6eyJuYW1lIjoibm9uZSJ9fX1dLFsxOCwyMCwiXFxiZXRhIiwxLHsic3R5bGUiOnsiYm9keSI6eyJuYW1lIjoiZGFzaGVkIn19fV0sWzE3LDIwLCJcXHNpZ21hIF8yIiwyLHsiY3VydmUiOjJ9XSxbMTYsMjAsIlxcc2lnbWEgXzEiLDAseyJjdXJ2ZSI6LTJ9XSxbMTUsMTgsIlxcc3F1YXJlIiwxLHsic3R5bGUiOnsiYm9keSI6eyJuYW1lIjoibm9uZSJ9LCJoZWFkIjp7Im5hbWUiOiJub25lIn19fV1d
\begin{tikzcd}
	X & Y & X & {X \coprod X} && Y \\
	{X_1} & X & {X_2} \\
	& {X'} && X && {X'} \\
	X & {X_1} && X & {X_1} \\
	{X_2} & {X'} && {X_2} & {X'} \\
	&& Y &&& X
	\arrow["g", color={rgb,255:red,255;green,51;blue,65}, from=1-1, to=1-2]
	\arrow["a"', color={rgb,255:red,51;green,68;blue,255}, from=1-1, to=2-1]
	\arrow["l"', color={rgb,255:red,255;green,51;blue,65}, from=1-3, to=1-2]
	\arrow["d", color={rgb,255:red,51;green,68;blue,255}, from=1-3, to=2-3]
	\arrow["{(g,l)}", color={rgb,255:red,255;green,51;blue,65}, from=1-4, to=1-6]
	\arrow["{\nabla_X}"', from=1-4, to=3-4]
	\arrow["{(c'a, b'd)}"{description}, color={rgb,255:red,51;green,68;blue,255}, from=1-4, to=3-6]
	\arrow["h", color={rgb,255:red,255;green,51;blue,65}, from=2-1, to=1-2]
	\arrow["{c'}"', curve={height=12pt}, dashed, from=2-1, to=3-2]
	\arrow["f", color={rgb,255:red,255;green,51;blue,65}, from=2-2, to=1-2]
	\arrow["b", color={rgb,255:red,51;green,68;blue,255}, from=2-2, to=2-1]
	\arrow["c"', color={rgb,255:red,51;green,68;blue,255}, from=2-2, to=2-3]
	\arrow["\square"{description}, draw=none, from=2-2, to=3-2]
	\arrow["j"', color={rgb,255:red,255;green,51;blue,65}, from=2-3, to=1-2]
	\arrow["{b'}", curve={height=-12pt}, dashed, from=2-3, to=3-2]
	\arrow["{\alpha }"', color={rgb,255:red,255;green,51;blue,65}, dashed, from=3-6, to=1-6]
	\arrow["\beta", dashed, from=3-6, to=3-4]
	\arrow["b", from=4-1, to=4-2]
	\arrow["c"', from=4-1, to=5-1]
	\arrow["\square"{description}, draw=none, from=4-1, to=5-2]
	\arrow["{c'}", from=4-2, to=5-2]
	\arrow["h", curve={height=-12pt}, from=4-2, to=6-3]
	\arrow["b", from=4-4, to=4-5]
	\arrow["c"', from=4-4, to=5-4]
	\arrow["\square"{description}, draw=none, from=4-4, to=5-5]
	\arrow["{c'}", from=4-5, to=5-5]
	\arrow["{\sigma _1}", curve={height=-12pt}, from=4-5, to=6-6]
	\arrow["{b'}"', from=5-1, to=5-2]
	\arrow["j"', curve={height=12pt}, from=5-1, to=6-3]
	\arrow["\alpha"{description}, color={rgb,255:red,255;green,51;blue,65}, dashed, from=5-2, to=6-3]
	\arrow["{b'}"', from=5-4, to=5-5]
	\arrow["{\sigma _2}"', curve={height=12pt}, from=5-4, to=6-6]
	\arrow["\beta"{description}, dashed, from=5-5, to=6-6]
\end{tikzcd}.
        \end{equation}
    \end{proof}
\end{theorem}

\begin{lemma}
    选定 $\mathrm{Hom}(X,Y)$, 其中 $X \in \mathcal{C}$ (余纤维对象). 此时, 柱同伦蕴含路同伦.
    \begin{proof}
        给定下图左侧所示的柱同伦, 下证明存在右式所示的路同伦:
\begin{equation}
    % https://q.uiver.app/#q=WzAsOCxbMCwwLCJYIFxcY29wcm9kIFgiXSxbMSwwLCJZIl0sWzMsMCwiWSciXSxbNCwwLCJZIl0sWzAsMSwiWCJdLFsxLDEsIlgnIl0sWzMsMSwiWCJdLFs0LDEsIllcXHByb2QgWSJdLFswLDEsInt7KGcsZil9fSIsMCx7ImNvbG91ciI6Wy00LDEwMCw2MF19LFstNCwxMDAsNjAsMV1dLFswLDQsInt7XFxuYWJsYV9YfX0iLDJdLFswLDUsInsoXFxwYXJ0aWFsIF8wLCBcXHBhcnRpYWwgXzEpfSIsMSx7ImNvbG91ciI6WzIzNSwxMDAsNjBdfSxbMjM1LDEwMCw2MCwxXV0sWzYsMiwiXFxsYW1iZGEiLDAseyJjb2xvdXIiOlstNCwxMDAsNjBdLCJzdHlsZSI6eyJib2R5Ijp7Im5hbWUiOiJkYXNoZWQifX19LFstNCwxMDAsNjAsMV1dLFsyLDcsIntcXGJpbm9te2RfMH17ZF8xfX0iLDEseyJjb2xvdXIiOlsyMzUsMTAwLDYwXX0sWzIzNSwxMDAsNjAsMV1dLFszLDIsInMiLDJdLFszLDcsIntcXERlbHRhIF9ZfSJdLFs1LDEsImgiLDIseyJjb2xvdXIiOlstNCwxMDAsNjBdfSxbLTQsMTAwLDYwLDFdXSxbNSw0LCJ7XFxzaWdtYSB9Il0sWzYsNywie1xcYmlub20gZ2Z9IiwyLHsiY29sb3VyIjpbLTQsMTAwLDYwXX0sWy00LDEwMCw2MCwxXV1d
\begin{tikzcd}
	{X \coprod X} & Y && {Y'} & Y \\
	X & {X'} && X & {Y\prod Y}
	\arrow["{{{(g,f)}}}", color={rgb,255:red,255;green,51;blue,65}, from=1-1, to=1-2]
	\arrow["{{{\nabla_X}}}"', from=1-1, to=2-1]
	\arrow["{{(\partial _0, \partial _1)}}"{description}, color={rgb,255:red,51;green,68;blue,255}, from=1-1, to=2-2]
	\arrow["{{\binom{d_0}{d_1}}}"{description}, color={rgb,255:red,51;green,68;blue,255}, from=1-4, to=2-5]
	\arrow["s"', from=1-5, to=1-4]
	\arrow["{{\Delta _Y}}", from=1-5, to=2-5]
	\arrow["h"', color={rgb,255:red,255;green,51;blue,65}, from=2-2, to=1-2]
	\arrow["{{\sigma }}", from=2-2, to=2-1]
	\arrow["\lambda", color={rgb,255:red,255;green,51;blue,65}, dashed, from=2-4, to=1-4]
	\arrow["{{\binom fg}}"', color={rgb,255:red,255;green,51;blue,65}, from=2-4, to=2-5]
\end{tikzcd}.
\end{equation}
        依照 CM4, 将 $\Delta_Y$ 分解作 $\mathsf{Fib} \circ \mathsf{TCofib}$ 的形式. 下只需求解 $\lambda$ 使得右图交换. 由\Cref{thm: homotopy is equivalence relation}, 不妨假定 $\partial_0$ 与 $\partial_1$ 为平凡纤维, CM3 给出态射提升的交换图
        \begin{equation}
            % https://q.uiver.app/#q=WzAsNyxbMCwwLCJYIl0sWzEsMCwiWSJdLFsyLDAsIlknIl0sWzAsMSwiWCciXSxbMiwxLCJZXFxwcm9kIFkiXSxbMywwLCJcXGJpbm9te2RfMH17ZF8xfSBcXGNpcmMgcyBcXGNpcmMgZyA9IFxcYmlub20xMSBcXGNpcmMgZyA9IFxcYmlub20gZ2ciXSxbMywxLCJcXGJpbm9te2h9e2cgXFxzaWdtYX1cXGNpcmMgXFxwYXJ0aWFsXzAgPSBcXGJpbm9tIHtoXFxjaXJjIFxccGFydGlhbCBfMH17Z1xcY2lyYyAoXFxzaWdtYSBcXGNpcmMgXFxwYXJ0aWFsIF8wKX0gPSBcXGJpbm9tIGdnIl0sWzAsMSwiZyIsMCx7ImNvbG91ciI6Wy00LDEwMCw2MF19LFstNCwxMDAsNjAsMV1dLFswLDMsIlxccGFydGlhbCBfMCIsMix7ImNvbG91ciI6WzIzNSwxMDAsNjBdfSxbMjM1LDEwMCw2MCwxXV0sWzAsMywie1xcbWF0aHNme1RDb2ZpYn19IiwwLHsiY29sb3VyIjpbMjM1LDEwMCw2MF0sInN0eWxlIjp7ImJvZHkiOnsibmFtZSI6Im5vbmUifSwiaGVhZCI6eyJuYW1lIjoibm9uZSJ9fX0sWzIzNSwxMDAsNjAsMV1dLFsxLDIsInMiXSxbMiw0LCJ7XFxiaW5vbXtkXzB9e2RfMX19IiwwLHsiY29sb3VyIjpbMjM1LDEwMCw2MF19LFsyMzUsMTAwLDYwLDFdXSxbMiw0LCJ7XFxtYXRoc2Z7RmlifX0iLDIseyJjb2xvdXIiOlsyMzUsMTAwLDYwXSwic3R5bGUiOnsiYm9keSI6eyJuYW1lIjoibm9uZSJ9LCJoZWFkIjp7Im5hbWUiOiJub25lIn19fSxbMjM1LDEwMCw2MCwxXV0sWzMsMiwiXFxtdSIsMSx7InN0eWxlIjp7ImJvZHkiOnsibmFtZSI6ImRhc2hlZCJ9fX1dLFszLDQsIlxcYmlub217aH17ZyBcXHNpZ21hfSIsMix7ImNvbG91ciI6Wy00LDEwMCw2MF19LFstNCwxMDAsNjAsMV1dXQ==
\begin{tikzcd}[ampersand replacement=\&]
	X \& Y \& {Y'} \& {\binom{d_0}{d_1} \circ s \circ g = \binom11 \circ g = \binom gg}, \\
	{X'} \&\& {Y\prod Y} \& {\binom{h}{g \sigma}\circ \partial_0 = \binom {h\circ \partial _0}{g\circ (\sigma \circ \partial _0)} = \binom gg}.
	\arrow["g", color={rgb,255:red,255;green,51;blue,65}, from=1-1, to=1-2]
	\arrow["{\partial _0}"', color={rgb,255:red,51;green,68;blue,255}, from=1-1, to=2-1]
	\arrow["{{\mathsf{TCofib}}}", color={rgb,255:red,51;green,68;blue,255}, draw=none, from=1-1, to=2-1]
	\arrow["s", from=1-2, to=1-3]
	\arrow["{{\binom{d_0}{d_1}}}", color={rgb,255:red,51;green,68;blue,255}, from=1-3, to=2-3]
	\arrow["{{\mathsf{Fib}}}"', color={rgb,255:red,51;green,68;blue,255}, draw=none, from=1-3, to=2-3]
	\arrow["\mu"{description}, dashed, from=2-1, to=1-3]
	\arrow["{\binom{h}{g \sigma}}"', color={rgb,255:red,255;green,51;blue,65}, from=2-1, to=2-3]
\end{tikzcd}
        \end{equation}
        取提升所得的态射 $\mu$, 记 $\lambda : = \mu \circ \partial _1$, 检验得交换图
        \begin{equation}
            % https://q.uiver.app/#q=WzAsNyxbMSwwLCJZJyJdLFsyLDAsIlkiXSxbMSwxLCJYIl0sWzIsMSwiWVxccHJvZCBZIl0sWzAsMSwiWCciXSxbMywwLCJkXzAgXFxjaXJjIFxcbGFtYmRhID0gZF8wIFxcY2lyYyBcXG11IFxcY2lyYyBcXHBhcnRpYWwgXzEgPSBoXFxjaXJjIFxccGFydGlhbCBfMSA9IGYiXSxbMywxLCJkXzEgXFxjaXJjIFxcbGFtYmRhID0gZF8xIFxcY2lyYyBcXG11IFxcY2lyYyBcXHBhcnRpYWwgXzEgPSBnIFxcY2lyYyBcXHNpZ21hIFxcY2lyYyBcXHBhcnRpYWwgXzEgPSAgZyJdLFswLDMsInt7XFxiaW5vbXtkXzB9e2RfMX19fSIsMSx7ImNvbG91ciI6WzIzNSwxMDAsNjBdfSxbMjM1LDEwMCw2MCwxXV0sWzEsMCwicyIsMl0sWzEsMywie3tcXERlbHRhIF9ZfX0iXSxbMiwzLCJ7e1xcYmlub20gZmd9fSIsMix7ImNvbG91ciI6Wy00LDEwMCw2MF19LFstNCwxMDAsNjAsMV1dLFsyLDAsIlxcbGFtYmRhIiwwLHsiY29sb3VyIjpbLTQsMTAwLDYwXSwic3R5bGUiOnsiYm9keSI6eyJuYW1lIjoiZGFzaGVkIn19fSxbLTQsMTAwLDYwLDFdXSxbNCwwLCJcXG11ICJdLFsyLDQsIlxccGFydGlhbCBfMSJdXQ==
\begin{tikzcd}[ampersand replacement=\&]
	\& {Y'} \& Y \& {d_0 \circ \lambda = d_0 \circ \mu \circ \partial _1 = h\circ \partial _1 = f} \\
	{X'} \& X \& {Y\prod Y} \& {d_1 \circ \lambda = d_1 \circ \mu \circ \partial _1 = g \circ \sigma \circ \partial _1 =  g}
	\arrow["{{{\binom{d_0}{d_1}}}}"{description}, color={rgb,255:red,51;green,68;blue,255}, from=1-2, to=2-3]
	\arrow["s"', from=1-3, to=1-2]
	\arrow["{{{\Delta _Y}}}", from=1-3, to=2-3]
	\arrow["{\mu }", from=2-1, to=1-2]
	\arrow["\lambda", color={rgb,255:red,255;green,51;blue,65}, dashed, from=2-2, to=1-2]
	\arrow["{\partial _1}", from=2-2, to=2-1]
	\arrow["{{{\binom fg}}}"', color={rgb,255:red,255;green,51;blue,65}, from=2-2, to=2-3]
\end{tikzcd}.
        \end{equation}
    \end{proof}
\end{lemma}

\begin{theorem}
    (余纤维对象出发的路同伦). 选定 $\mathrm{Hom}(X,Y)$, 其中 $X \in \mathcal{C}$ (余纤维对象). 对任意的路同伦的态射 $f, g : X \to Y$, 有以下事实
    \begin{enumerate}
        \item 对任意 $\psi : U \to X$, $f \circ \psi$ 与 $g \circ \psi$ 也是路同伦的态射;
        \item 对任意 $\varphi : Y \to W$, $\varphi \circ f$ 与 $\varphi \circ g$ 也是路同伦的态射.
    \end{enumerate}
    \begin{proof}
        (1). \Cref{prop: homotopy preserved by composition} 说明向前复合永远保持路同伦关系.
        \\
        (2). 给定路同伦交换图
        \begin{equation}\label{eq: path homotopy preserved by composition}
% https://q.uiver.app/#q=WzAsNCxbMCwwLCJZJyJdLFsyLDAsIlkiXSxbMCwxLCJYIl0sWzIsMSwiWVxccHJvZCBZIl0sWzAsMywie3t7XFxiaW5vbXtkXzB9e2RfMX19fX0iLDEseyJjb2xvdXIiOlsyMzUsMTAwLDYwXX0sWzIzNSwxMDAsNjAsMV1dLFsxLDAsInMiLDJdLFsxLDMsInt7e1xcRGVsdGEgX1l9fX0iXSxbMiwwLCJcXGxhbWJkYSIsMCx7ImNvbG91ciI6Wy00LDEwMCw2MF0sInN0eWxlIjp7ImJvZHkiOnsibmFtZSI6ImRhc2hlZCJ9fX0sWy00LDEwMCw2MCwxXV0sWzIsMywie3t7XFxiaW5vbSBmZ319fSIsMix7ImNvbG91ciI6Wy00LDEwMCw2MF19LFstNCwxMDAsNjAsMV1dXQ==
\begin{tikzcd}[ampersand replacement=\&]
	{Y'} \&\& Y \\
	X \&\& {Y\prod Y}
	\arrow["{{{{\binom{d_0}{d_1}}}}}"{description}, color={rgb,255:red,51;green,68;blue,255}, from=1-1, to=2-3]
	\arrow["s"', from=1-3, to=1-1]
	\arrow["{{{{\Delta _Y}}}}", from=1-3, to=2-3]
	\arrow["\lambda", color={rgb,255:red,255;green,51;blue,65}, dashed, from=2-1, to=1-1]
	\arrow["{{{{\binom fg}}}}"', color={rgb,255:red,255;green,51;blue,65}, from=2-1, to=2-3]
\end{tikzcd}.
        \end{equation}
        由\Cref{lem: equivalent definition of path homotopy}, 假定 $\binom{d_0}{d_1} \in \mathsf{Fib}$. 将 $s$ 分解作 $\mathsf{Fib} \circ \mathsf{TCofib}$, 则存在提升态射 $\lambda'$ (CM3):
        \begin{equation}
            % https://q.uiver.app/#q=WzAsNSxbMSwxLCJZJyJdLFsyLDAsIlkiXSxbMCwyLCJYIl0sWzIsMiwiWVxccHJvZCBZIl0sWzAsMCwiXFx3aWRldGlsZGUgWSJdLFswLDMsInt7e1xcYmlub217ZF8wfXtkXzF9fX19IiwxLHsiY29sb3VyIjpbMjM1LDEwMCw2MF19LFsyMzUsMTAwLDYwLDFdXSxbMSwzLCJ7e3tcXERlbHRhIF9ZfX19Il0sWzIsMCwiXFxsYW1iZGEiLDAseyJjb2xvdXIiOlstNCwxMDAsNjBdLCJzdHlsZSI6eyJib2R5Ijp7Im5hbWUiOiJkYXNoZWQifX19LFstNCwxMDAsNjAsMV1dLFsyLDMsInt7e1xcYmlub20gZmd9fX0iLDIseyJjb2xvdXIiOlstNCwxMDAsNjBdfSxbLTQsMTAwLDYwLDFdXSxbNCwwLCJzXzEiLDIseyJjb2xvdXIiOlsyMzUsMTAwLDYwXX0sWzIzNSwxMDAsNjAsMV1dLFsxLDQsInNfMiIsMl0sWzIsNCwiXFxsYW1iZGEgJyIsMCx7InN0eWxlIjp7ImJvZHkiOnsibmFtZSI6ImRhc2hlZCJ9fX1dLFsxLDQsIlxcbWF0aHNme1RDb2ZpYn0iLDAseyJsYWJlbF9wb3NpdGlvbiI6NDAsInN0eWxlIjp7ImJvZHkiOnsibmFtZSI6Im5vbmUifSwiaGVhZCI6eyJuYW1lIjoibm9uZSJ9fX1dLFs0LDAsIlxcbWF0aHNme1RGaWJ9IiwwLHsibGFiZWxfcG9zaXRpb24iOjYwLCJjb2xvdXIiOlsyMzUsMTAwLDYwXSwic3R5bGUiOnsiYm9keSI6eyJuYW1lIjoibm9uZSJ9LCJoZWFkIjp7Im5hbWUiOiJub25lIn19fSxbMjM1LDEwMCw2MCwxXV1d
\begin{tikzcd}[ampersand replacement=\&]
	{\widetilde Y} \&\& Y \\
	\& {Y'} \\
	X \&\& {Y\prod Y}
	\arrow["{s_1}"', color={rgb,255:red,51;green,68;blue,255}, from=1-1, to=2-2]
	\arrow["{\mathsf{TFib}}"{pos=0.6}, color={rgb,255:red,51;green,68;blue,255}, draw=none, from=1-1, to=2-2]
	\arrow["{s_2}"', from=1-3, to=1-1]
	\arrow["{\mathsf{TCofib}}"{pos=0.4}, draw=none, from=1-3, to=1-1]
	\arrow["{{{{\Delta _Y}}}}", from=1-3, to=3-3]
	\arrow["{{{{\binom{d_0}{d_1}}}}}"{description}, color={rgb,255:red,51;green,68;blue,255}, from=2-2, to=3-3]
	\arrow["{\lambda '}", dashed, from=3-1, to=1-1]
	\arrow["\lambda", color={rgb,255:red,255;green,51;blue,65}, dashed, from=3-1, to=2-2]
	\arrow["{{{{\binom fg}}}}"', color={rgb,255:red,255;green,51;blue,65}, from=3-1, to=3-3]
\end{tikzcd}.
        \end{equation}
        因此, 不妨假设\Cref{eq: path homotopy preserved by composition} 中 $s \in \mathsf{TCofib}$. 将 $\Delta_C$ 分解作 $\mathsf{Fib} \circ \mathsf{TCofib}$, 由 CM3 得提升态射 $t$:
\begin{equation}
    % https://q.uiver.app/#q=WzAsNyxbMCwwLCJZJyJdLFswLDIsIlgiXSxbNCwyLCJXIFxccHJvZCBXIl0sWzQsMCwiVyJdLFszLDEsIlcnIl0sWzIsMCwiWSJdLFsyLDIsIllcXHByb2QgWSJdLFsxLDAsIlxcbGFtYmRhIiwwLHsiY29sb3VyIjpbLTQsMTAwLDYwXSwic3R5bGUiOnsiYm9keSI6eyJuYW1lIjoiZGFzaGVkIn19fSxbLTQsMTAwLDYwLDFdXSxbMywyLCJcXERlbHRhX1ciXSxbNCwyLCJcXG1hdGhzZntGaWJ9Il0sWzMsNCwiXFxtYXRoc2Z7VENvZmlifSIsMix7ImxhYmVsX3Bvc2l0aW9uIjo3MH1dLFswLDQsInQiLDAseyJsYWJlbF9wb3NpdGlvbiI6NjAsImNvbG91ciI6Wy00LDEwMCw2MF0sInN0eWxlIjp7ImJvZHkiOnsibmFtZSI6ImRhc2hlZCJ9fX0sWy00LDEwMCw2MCwxXV0sWzUsMCwicyIsMl0sWzUsMCwiXFxtYXRoc2Z7VENvZmlifSIsMCx7InN0eWxlIjp7ImJvZHkiOnsibmFtZSI6Im5vbmUifSwiaGVhZCI6eyJuYW1lIjoibm9uZSJ9fX1dLFs1LDMsIlxcdmFycGhpICJdLFs1LDYsInt7e1xcRGVsdGEgX1l9fX0iXSxbMSw2LCJcXGJpbm9tIHtmfXsgZ30iLDIseyJjb2xvdXIiOlstNCwxMDAsNjBdfSxbLTQsMTAwLDYwLDFdXSxbNiwyLCJcXHZhcnBoaSBcXHByb2QgXFx2YXJwaGkgIiwyXSxbMCw2LCJcXGJpbm9te2RfMH17ZF8xfSIsMix7ImxhYmVsX3Bvc2l0aW9uIjo0MCwiY29sb3VyIjpbMjM1LDEwMCw2MF19LFsyMzUsMTAwLDYwLDFdXV0=
\begin{tikzcd}[ampersand replacement=\&]
	{Y'} \&\& Y \&\& W \\
	\&\&\& {W'} \\
	X \&\& {Y\prod Y} \&\& {W \prod W}
	\arrow["t"{pos=0.6}, color={rgb,255:red,255;green,51;blue,65}, dashed, from=1-1, to=2-4]
	\arrow["{\binom{d_0}{d_1}}"'{pos=0.4}, color={rgb,255:red,51;green,68;blue,255}, from=1-1, to=3-3]
	\arrow["s"', from=1-3, to=1-1]
	\arrow["{\mathsf{TCofib}}", draw=none, from=1-3, to=1-1]
	\arrow["{\varphi }", from=1-3, to=1-5]
	\arrow["{{{{\Delta _Y}}}}", from=1-3, to=3-3]
	\arrow["{\mathsf{TCofib}}"'{pos=0.7}, from=1-5, to=2-4]
	\arrow["{\Delta_W}", from=1-5, to=3-5]
	\arrow["{\mathsf{Fib}}", from=2-4, to=3-5]
	\arrow["\lambda", color={rgb,255:red,255;green,51;blue,65}, dashed, from=3-1, to=1-1]
	\arrow["{\binom {f}{ g}}"', color={rgb,255:red,255;green,51;blue,65}, from=3-1, to=3-3]
	\arrow["{\varphi \prod \varphi }"', from=3-3, to=3-5]
\end{tikzcd}.
\end{equation}
        删减与 $Y$ 相连的态射, 以及 $\binom{d_0}{d_1}$, 得到路同伦关系
\begin{equation}
    % https://q.uiver.app/#q=WzAsNixbMCwxLCJZJyJdLFswLDIsIlgiXSxbMiwyLCJXIFxccHJvZCBXIl0sWzIsMCwiVyJdLFswLDAsIlcnIl0sWzEsMiwiWVxccHJvZCBZIl0sWzEsMCwiXFxsYW1iZGEiLDAseyJjb2xvdXIiOlstNCwxMDAsNjBdLCJzdHlsZSI6eyJib2R5Ijp7Im5hbWUiOiJkYXNoZWQifX19LFstNCwxMDAsNjAsMV1dLFszLDIsIlxcRGVsdGFfVyJdLFs0LDIsIlxcbWF0aHNme0ZpYn0iXSxbMyw0LCJcXG1hdGhzZntUQ29maWJ9IiwyXSxbMCw0LCJ0IiwwLHsiY29sb3VyIjpbLTQsMTAwLDYwXSwic3R5bGUiOnsiYm9keSI6eyJuYW1lIjoiZGFzaGVkIn19fSxbLTQsMTAwLDYwLDFdXSxbMSw1LCJcXGJpbm9tIHtmfXsgZ30iLDIseyJjb2xvdXIiOlstNCwxMDAsNjBdfSxbLTQsMTAwLDYwLDFdXSxbNSwyLCJcXHZhcnBoaSBcXHByb2QgXFx2YXJwaGkgIiwyXV0=
\begin{tikzcd}[ampersand replacement=\&]
	{W'} \&\& W \\
	{Y'} \\
	X \& {Y\prod Y} \& {W \prod W}
	\arrow["{\mathsf{Fib}}", from=1-1, to=3-3]
	\arrow["{\mathsf{TCofib}}"', from=1-3, to=1-1]
	\arrow["{\Delta_W}", from=1-3, to=3-3]
	\arrow["t", color={rgb,255:red,255;green,51;blue,65}, dashed, from=2-1, to=1-1]
	\arrow["\lambda", color={rgb,255:red,255;green,51;blue,65}, dashed, from=3-1, to=2-1]
	\arrow["{\binom {f}{ g}}"', color={rgb,255:red,255;green,51;blue,65}, from=3-1, to=3-2]
	\arrow["{\varphi \prod \varphi }"', from=3-2, to=3-3]
\end{tikzcd}.
\end{equation}
    \end{proof}
\end{theorem}

\begin{corollary}\label{cor: homotopy is ideal}
    柱同伦是 $\mathcal{F}$ 上的等价关系理想; 路同伦是 $\mathcal{C}$ 上的等价关系理想.
\end{corollary}

我们暂不给出\Cref{thm: quillen theorem} 的证明. 

\begin{theorem}\label{thm: quillen theorem}
    (Quillen 定理). 在\Cref{item: finite limits and colimits} 的前提下, 以下是范畴等价:
    \begin{equation}\label{eq: quillen theorem equivalence}
        % https://q.uiver.app/#q=WzAsOSxbMCwwLCJcXHBpIFxcbWF0aHNjciBDIl0sWzAsMiwiXFxwaSBcXG1hdGhzY3IgRiJdLFswLDEsIlxccGkgKFxcbWF0aHNjciBDIFxcY2FwIFxcbWF0aHNjciBGICkiXSxbMiwwLCJcXG1hdGhzZntIb30oXFxtYXRoc2NyIEMpIl0sWzMsMCwiXFxtYXRoc2NyIENbIFxcbWF0aHNme1RDb2ZpYn1eey0xfV0iXSxbMiwyLCJcXG1hdGhzZntIb30oXFxtYXRoc2NyIEYpIl0sWzMsMiwiXFxtYXRoc2NyIEZbIFxcbWF0aHNme1RGaWJ9XnstMX1dIl0sWzMsMSwiXFxtYXRoc2NyIENbIFxcbWF0aHNme1dlcX1eey0xfV0iXSxbMiwxLCJcXG1hdGhzZntIb30oXFxtYXRoc2NyIEEpIl0sWzIsMCwiIiwwLHsic3R5bGUiOnsidGFpbCI6eyJuYW1lIjoibW9ubyJ9fX1dLFsyLDEsIiIsMix7InN0eWxlIjp7InRhaWwiOnsibmFtZSI6Im1vbm8ifX19XSxbMCwzLCJcXG92ZXJsaW5lIHtcXGdhbW1hIF9jfSJdLFszLDgsIlxcc2ltZXEiXSxbMSw1LCJcXG92ZXJsaW5lIHtcXGdhbW1hIF9mfSIsMl0sWzUsOCwiXFxzaW1lcSIsMl0sWzMsNCwiIiwyLHsibGV2ZWwiOjIsInN0eWxlIjp7ImhlYWQiOnsibmFtZSI6Im5vbmUifX19XSxbOCw3LCIiLDIseyJsZXZlbCI6Miwic3R5bGUiOnsiaGVhZCI6eyJuYW1lIjoibm9uZSJ9fX1dLFs1LDYsIiIsMix7ImxldmVsIjoyLCJzdHlsZSI6eyJoZWFkIjp7Im5hbWUiOiJub25lIn19fV0sWzIsOCwiXFxvdmVybGluZSBcXGdhbW1hICIsMl0sWzIsOCwiXFxzaW1lcSIsMCx7InN0eWxlIjp7ImJvZHkiOnsibmFtZSI6Im5vbmUifSwiaGVhZCI6eyJuYW1lIjoibm9uZSJ9fX1dLFszLDAsIiIsMix7Im9mZnNldCI6LTEsImN1cnZlIjotM31dLFs1LDEsIiIsMix7Im9mZnNldCI6MSwiY3VydmUiOjN9XSxbMTEsMjAsIlxcYm90IiwxLHsic2hvcnRlbiI6eyJzb3VyY2UiOjIwLCJ0YXJnZXQiOjIwfSwic3R5bGUiOnsiYm9keSI6eyJuYW1lIjoibm9uZSJ9LCJoZWFkIjp7Im5hbWUiOiJub25lIn19fV0sWzIxLDEzLCJcXGJvdCIsMSx7InNob3J0ZW4iOnsic291cmNlIjoyMCwidGFyZ2V0IjoyMH0sInN0eWxlIjp7ImJvZHkiOnsibmFtZSI6Im5vbmUifSwiaGVhZCI6eyJuYW1lIjoibm9uZSJ9fX1dXQ==
\begin{tikzcd}
	{\pi \mathcal C} && {\mathsf{Ho}(\mathcal C)} & {\mathcal C[ \mathsf{TCofib}^{-1}]} \\
	{\pi (\mathcal C \cap \mathcal F )} && {\mathsf{Ho}(\mathcal A)} & {\mathcal C[ \mathsf{Weq}^{-1}]} \\
	{\pi \mathcal F} && {\mathsf{Ho}(\mathcal F)} & {\mathcal F[ \mathsf{TFib}^{-1}]}
	\arrow[""{name=0, anchor=center, inner sep=0}, "{\overline {\gamma _c}}", from=1-1, to=1-3]
	\arrow[""{name=1, anchor=center, inner sep=0}, shift left, curve={height=-18pt}, from=1-3, to=1-1]
	\arrow[equals, from=1-3, to=1-4]
	\arrow["\simeq", from=1-3, to=2-3]
	\arrow[tail, from=2-1, to=1-1]
	\arrow["{\overline \gamma }"', from=2-1, to=2-3]
	\arrow["\simeq", draw=none, from=2-1, to=2-3]
	\arrow[tail, from=2-1, to=3-1]
	\arrow[equals, from=2-3, to=2-4]
	\arrow[""{name=2, anchor=center, inner sep=0}, "{\overline {\gamma _f}}"', from=3-1, to=3-3]
	\arrow["\simeq"', from=3-3, to=2-3]
	\arrow[""{name=3, anchor=center, inner sep=0}, shift right, curve={height=18pt}, from=3-3, to=3-1]
	\arrow[equals, from=3-3, to=3-4]
	\arrow["\bot"{description}, draw=none, from=0, to=1]
	\arrow["\bot"{description}, draw=none, from=3, to=2]
\end{tikzcd}.
    \end{equation}
    其中, $\pi(\mathcal{C})$ 是 $\mathcal{C}$ 关于路同伦这一等价关系 (\Cref{cor: homotopy is ideal}) 的商, $\pi(\mathcal{F})$ 是 $\mathcal{F}$ 关于柱同伦这一等价关系的商, $\pi(\mathcal{C} \cap \mathcal{F})$ 是 $\mathcal{C} \cap \mathcal{F}$ 关于柱同伦 (或路同伦) 这一等价关系的商.
\end{theorem}

\begin{example}
    对外三角范畴的相同闭模型结构, \Cref{thm:homotopy-pullback-1} 给出\Cref{eq: quillen theorem diagram} 的一种具体构造. \Cref{eq: quillen theorem equivalence} 中范畴等价成立.
\end{example}

\subsection{由 \texorpdfstring{$\mathcal{C} \cap \mathcal{F}$}{} 到同伦范畴的三种方式}\label{sec: three ways to homotopy category}

对相容闭模型结构, 本小节使用 $\mathcal{C} \cap \mathcal{F}$ 的三种``等价的商''描述同伦范畴. 往后选定外三角范畴 $\mathcal{A}$ 上的相容闭模型结构 (\Cref{def:compatible-model-structure}), 选定记号

\begin{definition}\label{def:compatible-model-structure-2}
    $\mathcal{A}$ 上的相容闭模型结构由以下资料描述:
    \begin{enumerate}
    \item $\mathsf{Cofib}$, $\mathsf{Fib}$ 与 $\mathsf{Weq}$ 分别为闭模型结构中的余纤维, 纤维与弱等价类;
    \item $\mathsf{TCofib} = \mathsf{Cofib} \cap \mathsf{Weq}$ 与 $\mathsf{TFib} = \mathsf{Fib} \cap \mathsf{Weq}$ 分别为平凡余纤维与平凡纤维;
    \item $\mathcal{C}$, $\mathcal{F}$ 与 $\mathcal{W}$ 分别为闭模型结构中的余纤维对象, 纤维对象与平凡对象;
    \item $\mathsf{T}\mathcal{F} = \mathcal{F} \cap \mathcal{W}$ 与 $\mathsf{T}\mathcal{C} = \mathcal{C} \cap \mathcal{W}$ 分别为平凡纤维对象与平凡余纤维对象.
    \item (Hovey). $(\mathcal{S}, \mathcal{S}^\perp; {}^\perp \mathcal{V}, \mathcal{V}) = (\mathsf{T}\mathcal{C}, \mathcal{F}, \mathcal{C}, \mathsf{T}\mathcal{C})$ 是其对应的 Hovey 孪生余挠对 (见\Cref{thm:model-to-hovey}, 逆命题即\Cref{sec:hovey-to-model}).
    \item (Hovey). 平凡对象类 $\mathcal{W} = \mathcal{N} = \mathrm{Cone}(\mathcal{V}, \mathcal{S}) = \mathrm{coCone}(\mathcal{V}, \mathcal{S})$.
\end{enumerate}
\end{definition}

后续证明三种构造局部化的等价方式:
\begin{equation}\label{eq: three ways to homotopy category}
    % https://q.uiver.app/#q=WzAsNCxbMSwxLCJcXGZyYWN7XFxtYXRoY2FsIEMgXFxjYXAgXFxtYXRoY2FsIEZ9e1xcbWF0aGNhbCBDIFxcY2FwIFxcbWF0aGNhbCBGIFxcY2FwIFxcbWF0aGNhbCBXfSJdLFsxLDAsIlxcbWF0aGNhbCBDIFxcY2FwIFxcbWF0aGNhbCBGIl0sWzAsMSwiKFxcbWF0aGNhbCBDIFxcY2FwIFxcbWF0aGNhbCBGKVsoXFxtYXRoc2Z7V2VxfV97XFxtYXRoY2FsIEMgXFxjYXAgXFxtYXRoY2FsIEZ9KV57LTF9XSJdLFsyLDEsIlxccGkgKFxcbWF0aGNhbCBDIFxcY2FwIFxcbWF0aGNhbCBGKSJdLFsxLDAsIlFfMiJdLFsxLDMsIlFfMyIsMCx7ImN1cnZlIjotMn1dLFsxLDIsIlFfMSIsMix7ImN1cnZlIjoyfV1d
\begin{tikzcd}[ampersand replacement=\&]
	\& {\mathcal C \cap \mathcal F} \\
	{(\mathcal C \cap \mathcal F)[(\mathsf{Weq}_{\mathcal C \cap \mathcal F})^{-1}]} \& {\frac{\mathcal C \cap \mathcal F}{\mathcal C \cap \mathcal F \cap \mathcal W}} \& {\pi (\mathcal C \cap \mathcal F)}
	\arrow["{Q_1}"', curve={height=12pt}, from=1-2, to=2-1]
	\arrow["{Q_2}", from=1-2, to=2-2]
	\arrow["{Q_3}", curve={height=-12pt}, from=1-2, to=2-3]
\end{tikzcd}.
\end{equation}

此处, $\mathsf{Weq}_{\mathcal{C} \cap \mathcal{F}} = \mathsf{Weq}\cap \mathsf{Mor}(\mathcal{C} \cap  \mathcal{F})$, $Q_1$ 是 GZ 局部化, $Q_2$ 是加法商, $Q_3$ 是同伦商 (稍后定义).

正式证明该命题前, 先看一则熟悉的例子.

\begin{proposition}
    假定 $\mathcal{R}$ 是一般的加法范畴, $C(\mathcal{R})$ 是复形范畴. 同伦范畴有以下三种等价描述.
    \begin{enumerate}
        \item (局部化). 称 $f : X \to Y$ 与 $g : Y \to X$ 同伦等价, 若 $fg - 1_Y, gf - 1_X$ 通过可裂无环复形分解. 记 $S$ 是同伦等价, 定义 $Q_1 : C(\mathcal{R}) \to C(\mathcal{R})[S^{-1}]$;
        \item (加法商). 记 $\mathcal{N}$ 为所有可裂无环复形构成的全子加法范畴, 定义 $Q_2 : C(\mathcal{R}) \to C(\mathcal{R}) / \mathcal{N}$.
        \item (同伦商). 称 $f, g : X \to Y$ 同伦, 若存在态射 $s : X \to Y$ 使得 $f - g = sd + ds$. 定义 $Q_3 : C(\mathcal{R}) \to \pi C(\mathcal{R})$ 为加法范畴 $C(\mathcal{R})$ 关于同伦关系的商范畴.
    \end{enumerate}
    以下证明过程无需\Cref{thm: when localization is additive}.
    \begin{proof}
        显然 $Q_2 (S)$ 是同构, 故 $Q_2$ 经 $Q_1$ 分解.
        \\
        再证明 $Q_1$ 经 $Q_3$ 分解, 即同伦等价的态射 $s : f \dim g : X \to Y$ 在 $C(\mathcal{R})[S^{-1}]$ 中同构. \Cref{ex: chain homotopy} 构造了态射 $(s,f,g) : \mathrm{Cyl}(X) \to Y$. 注意到
        \begin{equation}
            f = [X \xrightarrow {e_2}f = \mathrm{Cyl}(1_X) \xrightarrow {(s,f,g)} Y], \quad g = [X \xrightarrow {e_3} \mathrm{Cyl}(1_X) \xrightarrow {(s,f,g)} Y].
        \end{equation}
        $e_2$ 与 $e_3$ 在 $C(\mathcal{R})[S^{-1}]$ 中相同, 因为两者在复合同伦等价 $p_1 : \mathrm{Cyl} (X) \to \Sigma$ 后都是零态射.
        \\
        最后证明 $Q_3$ 经 $Q_2$ 分解. 显然 $Q_3(f) = Q_3 (g)$ 蕴含 $(f-g)$ 零伦, 即, 通过零对象分解.
    \end{proof}
\end{proposition}

\begin{lemma}
    $\mathsf{Weq}_{\mathcal{C} \cap \mathcal{F}}$ 对直和封闭, 从而 (\Cref{thm: when localization is additive}) $Q_1$ 是加法函子.
    \begin{proof}
        给定 $\mathcal{C} \cap \mathcal{F}$ 中弱等价 $f : X \to Y$. 由\Cref{lem:basic-model-structure} 第二条, $f$ 分解作 $\mathsf{TFib} \circ \mathsf{TCofib}$, 得
        \begin{equation}\label{eq: factorization of weq in C cap F}
            % https://q.uiver.app/#q=WzAsNSxbMCwwLCJYIl0sWzIsMCwiWSJdLFsxLDEsIkUiXSxbMiwyLCJTIl0sWzAsMiwiViJdLFswLDIsIlxcbWF0aHNme1RDb2ZpYn0iLDIseyJzdHlsZSI6eyJ0YWlsIjp7Im5hbWUiOiJtb25vIn19fV0sWzIsMSwiXFxtYXRoc2Z7VEZpYn0iLDIseyJzdHlsZSI6eyJoZWFkIjp7Im5hbWUiOiJlcGkifX19XSxbMCwxLCJmIl0sWzQsMiwiIiwyLHsic3R5bGUiOnsidGFpbCI6eyJuYW1lIjoibW9ubyJ9fX1dLFsyLDMsIiIsMix7InN0eWxlIjp7ImhlYWQiOnsibmFtZSI6ImVwaSJ9fX1dXQ==
\begin{tikzcd}[ampersand replacement=\&]
	X \&\& Y \\
	\& E \\
	V \&\& S
	\arrow["f", from=1-1, to=1-3]
	\arrow["{\mathsf{TCofib}}"', tail, from=1-1, to=2-2]
	\arrow["{\mathsf{TFib}}"', two heads, from=2-2, to=1-3]
	\arrow[two heads, from=2-2, to=3-3]
	\arrow[tail, from=3-1, to=2-2]
\end{tikzcd}.
        \end{equation}
        $\mathsf{TCofib}$ 所在的 conflation 表明 $E \in \mathcal{T}$, $\mathsf{TFib}$ 所在的 conflation 表明 $E \in \mathcal{U}$. 因此 $E \in \mathcal{U} \cap \mathcal{T} (=\mathcal{C} \cap \mathcal{F})$. 这说明 $w$ 经 $\mathcal{C} \cap \mathcal{F}$ 中对象分解. 容易验证 $\mathsf{Weq}_{\mathcal{C} \cap \mathcal{F}}$ 对直和封闭.
    \end{proof}
\end{lemma}

\begin{theorem}\label{thm: three ways to homotopy category}
    \Cref{eq: three ways to homotopy category} 中 $Q_1$ 与 $Q_2$ 互相分解, 诱导了\Cref{thm: quotient category is localization} 中所示的典范同构. 同\Cref{thm: quotient category is localization} 中记号, 我们暂时将 $Q_2(f)$ 记作 $[f]$.
    \begin{proof}
        ($Q_2$ 经 $Q_1$ 分解). 只需说明对任意 $f \in \mathsf{Weq}_{\mathcal{C} \cap \mathcal{W}}$, $[f]$ 是同构. 由态射分解\Cref{eq: factorization of weq in C cap F}, 只需证明 $X \rightarrowtail E$ 与 $E \twoheadrightarrow Y$ 是同构. 以前者为例, 将 $X \rightarrowtail E = X \oplus S$ 写作直和, 今断言 $(1 \ \ 0) : X \oplus S$ 是 $\frac{\mathcal{C} \cap \mathcal{F}}{\mathcal{C}\cap \mathcal{W} \cap \mathcal{F}}$ 中的逆映射. 往证 $\binom{0 \ \ 0}{0 \ \ 1} : X \oplus S \to X \oplus S$ 通过 $\mathcal{C} \cap \mathcal{W} \cap \mathcal{F}$ 中对象分解. 只需证明 $1_S$ 被 $\mathcal{C} \cap \mathcal{W} \cap \mathcal{F}$ 中对象分解. 考虑可裂 conflation $S \rightarrowtail S_V \twoheadrightarrow S_S$, 则 $S_V \in \mathcal{S} \cap \mathcal{V} = \mathcal{C} \cap \mathcal{F} \cap \mathcal{W}$. 得证.
        \\
        ($Q_1$ 经 $Q_2$ 分解). 依照\Cref{thm: quotient category is localization} 将加法商转写作 GZ-局部化. 只需说明对任意 $f : X \to Y$, 若 $[f]$ 是同构, 则 $f \in \mathsf{Weq}_{\mathcal{C} \cap \mathcal{F}}$. 将 $f$ 分解作 $p \circ i \in \mathsf{TFib} \circ \mathsf{Cofib}$, 则 $p$ 是可裂满态射, 记右逆元 $j$. 
        \begin{equation}
            % https://q.uiver.app/#q=WzAsNyxbMiwwLCJYIl0sWzQsMCwiWSJdLFszLDEsIkUiXSxbNCwyLCJVIl0sWzIsMiwiUyJdLFswLDAsIlxcLCJdLFs2LDAsIlxcLCJdLFswLDEsImYiXSxbMCwyLCJpIiwwLHsic3R5bGUiOnsidGFpbCI6eyJuYW1lIjoibW9ubyJ9fX1dLFsyLDEsInAiLDAseyJzdHlsZSI6eyJoZWFkIjp7Im5hbWUiOiJlcGkifX19XSxbMiwwLCJxIiwwLHsib2Zmc2V0IjotMywic3R5bGUiOnsiYm9keSI6eyJuYW1lIjoiZG90dGVkIn19fV0sWzEsMiwiaiIsMCx7Im9mZnNldCI6LTMsInN0eWxlIjp7InRhaWwiOnsibmFtZSI6Im1vbm8ifX19XSxbNCwyXSxbMiwzXV0=
\begin{tikzcd}
	{\,} && X && Y && {\,} \\
	&&& E \\
	&& V && U
	\arrow["f", from=1-3, to=1-5]
	\arrow["i", tail, from=1-3, to=2-4]
	\arrow["j", shift left=3, tail, from=1-5, to=2-4]
	\arrow["q", shift left=3, dotted, from=2-4, to=1-3]
	\arrow["p", two heads, from=2-4, to=1-5]
	\arrow[from=2-4, to=3-5]
	\arrow[from=3-3, to=2-4]
\end{tikzcd}.
        \end{equation}
        由上一步骤中的结论, $p \circ j$ 与 $1$ 相差一个 $1_S$, 从而 $[p]$ 与 $[j]$ 是同构. 因此 $[i] = [j] \circ [f]$ 也是同构. 任取 $q : E \to X$ 使得 $[q]$ 是 $[i]$ 的逆元. 下图是加法商范畴中交换:
        \begin{equation}
            % https://q.uiver.app/#q=WzAsOCxbMSwxLCJYIl0sWzMsMSwiRSJdLFs1LDEsIlUiXSxbMCwwLCJcXCwiXSxbNiwwLCJcXCwiXSxbMSwwLCJYIl0sWzUsMCwiVSJdLFszLDAsIlggXFxvcGx1cyBVIl0sWzAsMSwiaSIsMCx7InN0eWxlIjp7InRhaWwiOnsibmFtZSI6Im1vbm8ifX19XSxbMSwwLCJxIiwwLHsib2Zmc2V0IjotMywic3R5bGUiOnsiYm9keSI6eyJuYW1lIjoiZG90dGVkIn19fV0sWzEsMiwiXFxwaSJdLFs1LDcsIlxcYmlub20gMTAiXSxbNyw2LCIoMCBcXCBcXCAxKSJdLFsxLDcsIlxcYmlub20gcSBcXHBpICJdLFswLDUsIiIsMCx7ImxldmVsIjoyLCJzdHlsZSI6eyJoZWFkIjp7Im5hbWUiOiJub25lIn19fV0sWzIsNiwiIiwxLHsibGV2ZWwiOjIsInN0eWxlIjp7ImhlYWQiOnsibmFtZSI6Im5vbmUifX19XV0=
\begin{tikzcd}
	{\,} & X && {X \oplus U} && U & {\,} \\
	& X && E && U
	\arrow["{\binom 10}", from=1-2, to=1-4]
	\arrow["{(0 \ \ 1)}", from=1-4, to=1-6]
	\arrow[equals, from=2-2, to=1-2]
	\arrow["i", tail, from=2-2, to=2-4]
	\arrow["{\binom q \pi }", from=2-4, to=1-4]
	\arrow["q", shift left=3, dotted, from=2-4, to=2-2]
	\arrow["\pi", from=2-4, to=2-6]
	\arrow[equals, from=2-6, to=1-6]
\end{tikzcd}.
        \end{equation}
        因此 $[\binom 1 0] : X \to X \oplus U$ 是同构. 容易看出 $[1_U] = 0$. 由定义, $U$ 是 $\mathcal{W}$ 中对象的形变收缩, 故 $U \in \mathcal{U} \cap \mathcal{W} = \mathcal{S}$. 由定义, $i$ 是平凡纤维, 故 $f \in \mathsf{Weq}$.
    \end{proof}
\end{theorem}

\begin{remark}
    以上将加法商写作 GZ 局部化 (\Cref{thm: quotient category is localization}), 并证明了范畴关于 $S$ 与 $\mathsf{Weq}_{\mathcal{C} \cap \mathcal{F}}$ 两个态射类的局部化是同构的. 
\end{remark}

\begin{theorem}\label{thm: homotopy category as homotopy quotient}
    \Cref{eq: three ways to homotopy category} 中存在 $Q_3 :{\mathcal{C} \cap \mathcal{F}} \to \pi(\mathcal{C} \cap \mathcal{F})$ 诱导的范畴同构 $\frac{\mathcal{C} \cap \mathcal{F}}{{\mathcal{C} \cap \mathcal{W} \cap \mathcal{F}}} \cong \pi (\mathcal{C} \cap \mathcal{F})$.
    \begin{proof}
        为证明 $\frac{\mathcal{C} \cap \mathcal{F}}{{\mathcal{C} \cap \mathcal{W} \cap \mathcal{F}}} \cong \pi (\mathcal{C} \cap \mathcal{F})$ 良定义, 只需说明 $\mathcal{C} \cap \mathcal{F}$ 中弱等价被映至 $\pi (\mathcal{C} \cap \mathcal{F})$ 中同构 (\Cref{thm: three ways to homotopy category}). 将弱等价分解作 $\mathsf{TFib} \circ \mathsf{TCofib}$, 下通过路同伦证明 $\mathsf{TFib}$ 在 $Q_3$ 下是同构. 通过柱同伦, 可以对偶地证明 $\mathsf{TCofib}$ 在 $Q_3$ 下也是同构.
        \\
        任取平凡纤维 $p : X \to Y$, 下证明 $\pi (-, p) : \pi(-, X) \to \pi(-, Y )$ 是函子的同构.
        \begin{enumerate}
            \item (可裂满). 由 $(0 \to A) \pitchfork p$, 得 $(-, p)_{\mathcal{C} \cap \mathcal{F}}$ 是满自然变换. 由\Cref{lem:mono-epi-split} 知 $p$ 可裂满, 故 $\pi (-, p)$ 可裂满.
            \item (单). 即证: 若 $p\circ f, p\circ g: A \to X$ 通过 $h$ 实现柱同伦, 则 $f, g : A \to Y$ 也是柱同伦的. 考虑下图:
            \begin{equation}
                % https://q.uiver.app/#q=WzAsNyxbMSwwLCJBIFxcY29wcm9kIEEiXSxbMSwyLCJBIl0sWzQsMiwiQSciXSxbNCwwLCJZIl0sWzMsMCwiWCJdLFswLDEsIlxcLCJdLFs1LDEsIlxcLCJdLFsyLDEsIlxcc2lnbWEgIl0sWzAsNCwiKGYsZykiXSxbNCwzLCJwIl0sWzAsMiwiKFxccGFydGlhbCBfMCwgXFxwYXJ0aWFsIF8xKSIsMix7ImxhYmVsX3Bvc2l0aW9uIjoyMH1dLFswLDIsIlxcbWF0aHNme0NvZmlifSIsMCx7ImxhYmVsX3Bvc2l0aW9uIjo0MCwic3R5bGUiOnsiYm9keSI6eyJuYW1lIjoibm9uZSJ9LCJoZWFkIjp7Im5hbWUiOiJub25lIn19fV0sWzIsMywiaCIsMl0sWzQsMywiXFxtYXRoc2Z7VEZpYn0iLDIseyJzdHlsZSI6eyJib2R5Ijp7Im5hbWUiOiJub25lIn0sImhlYWQiOnsibmFtZSI6Im5vbmUifX19XSxbMiw0LCJzIiwwLHsic3R5bGUiOnsiYm9keSI6eyJuYW1lIjoiZGFzaGVkIn19fV0sWzIsMSwiXFxtYXRoc2Z7V2VxfSIsMix7InN0eWxlIjp7ImJvZHkiOnsibmFtZSI6Im5vbmUifSwiaGVhZCI6eyJuYW1lIjoibm9uZSJ9fX1dLFswLDEsIlxcbmFibGFfQSIsMl1d
\begin{tikzcd}
	& {A \coprod A} && X & Y \\
	{\,} &&&&& {\,} \\
	& A &&& {A'}
	\arrow["{(f,g)}", from=1-2, to=1-4]
	\arrow["{\nabla_A}"', from=1-2, to=3-2]
	\arrow["{(\partial _0, \partial _1)}"'{pos=0.2}, from=1-2, to=3-5]
	\arrow["{\mathsf{Cofib}}"{pos=0.4}, draw=none, from=1-2, to=3-5]
	\arrow["p", from=1-4, to=1-5]
	\arrow["{\mathsf{TFib}}"', draw=none, from=1-4, to=1-5]
	\arrow["s", dashed, from=3-5, to=1-4]
	\arrow["h"', from=3-5, to=1-5]
	\arrow["{\sigma }", from=3-5, to=3-2]
	\arrow["{\mathsf{Weq}}"', draw=none, from=3-5, to=3-2]
\end{tikzcd}.
            \end{equation}
            CM3 给出分解态射 $s$. 此时, $f$ 与 $g$ 通过 $s$ 实现柱同伦. 
        \end{enumerate}
    下证明 $Q_3 : \mathcal{C} \cap \mathcal{F} \to \pi (\mathcal{C} \cap \mathcal{F})$ 经 $Q_2$ 分解, $Q_3$ 是关于柱同伦这一等价关系理想的商. 若 $f$ 与 $g$ 路同伦, 则考虑交换图:
    \begin{equation}
        % https://q.uiver.app/#q=WzAsNixbMSwwLCJBIFxcY29wcm9kIEEiXSxbMSwyLCJBIl0sWzQsMiwiQSciXSxbNCwwLCJYIl0sWzAsMSwiXFwsIl0sWzUsMSwiXFwsIl0sWzIsMSwiXFxzaWdtYSAiXSxbMCwzLCIoZixnKSJdLFswLDIsIihcXHBhcnRpYWwgXzAsIFxccGFydGlhbCBfMSkiLDIseyJsYWJlbF9wb3NpdGlvbiI6MjB9XSxbMCwyLCJcXG1hdGhzZntDb2ZpYn0iLDAseyJsYWJlbF9wb3NpdGlvbiI6NDAsInN0eWxlIjp7ImJvZHkiOnsibmFtZSI6Im5vbmUifSwiaGVhZCI6eyJuYW1lIjoibm9uZSJ9fX1dLFsyLDMsInMiLDJdLFsyLDEsIlxcbWF0aHNme1dlcX0iLDIseyJzdHlsZSI6eyJib2R5Ijp7Im5hbWUiOiJub25lIn0sImhlYWQiOnsibmFtZSI6Im5vbmUifX19XSxbMCwxLCJcXG5hYmxhX0EiLDJdXQ==
\begin{tikzcd}
	& {A \coprod A} &&& X \\
	{\,} &&&&& {\,} \\
	& A &&& {A'}
	\arrow["{(f,g)}", from=1-2, to=1-5]
	\arrow["{\nabla_A}"', from=1-2, to=3-2]
	\arrow["{(\partial _0, \partial _1)}"'{pos=0.2}, from=1-2, to=3-5]
	\arrow["{\mathsf{Cofib}}"{pos=0.4}, draw=none, from=1-2, to=3-5]
	\arrow["s"', from=3-5, to=1-5]
	\arrow["{\sigma }", from=3-5, to=3-2]
	\arrow["{\mathsf{Weq}}"', draw=none, from=3-5, to=3-2]
\end{tikzcd}.
    \end{equation}
    特别地, $[\sigma] \circ [\partial_0 - \partial_1] = [0]$. 由 $\sigma \in \mathsf{Weq}$, 得 $[\partial_0 - \partial_1] = 0$. 因此 $[f] = [g]$.
    \end{proof}
\end{theorem}

\subsection{同伦范畴的三角结构}

本章节证明 $\mathsf{Ho}\mathcal{A}$ 是三角范畴. 试回顾相容闭模型结构的资料 (\Cref{def:compatible-model-structure-2}), 以及同伦范畴的等价描述:

\begin{enumerate}
    \item (Quillen 定理, \Cref{thm: quillen theorem}). $\pi(\mathcal{C} \cap \mathcal{F})$, $\pi (\mathcal{F})$, $\pi(\mathcal{C})$ 与 $\mathsf{Ho}\mathcal{A}$ 是互相等价的范畴. 等价函子由全子范畴的包含诱导.
    \item (双纤维对象逼近, \Cref{sec: three ways to homotopy category}). \Cref{eq: three ways to homotopy category} 中函子彼此分解, 诱导了范畴同构
    \begin{equation}
        (\mathcal{C} \cap \mathcal{F}) [(\mathsf{Weq}_{\mathcal{C}\cap\mathcal{F}})^{-1}] \cong \frac{\mathcal{C} \cap \mathcal{F}}{\mathcal{C} \cap \mathcal{W} \cap \mathcal{F}} \cong \pi (\mathcal{C} \cap \mathcal{F}).
    \end{equation}
\end{enumerate}

我们尽可能地在 $\mathsf{Ho}\mathcal{A}$ 中构造平移函子, 最后使用 $\pi (\mathcal{C} \cap \mathcal{F})$ 建立一类``标准好三角''.

以下关于``好三角位移''的构造类似\Cref{thm:extri-frobenius-happel}.

\begin{definition}
    记局部化函子为 $[\cdot] : \mathcal{A} \to \mathsf{Ho}\mathcal{A}$. 对任意 $X \in \mathcal{A}$, 选定 conflation
    \begin{equation}
% https://q.uiver.app/#q=WzAsNCxbMCwwLCJYIl0sWzEsMCwiWF9WIl0sWzIsMCwiWF9VIl0sWzMsMCwiXFwsIl0sWzAsMSwiaV9YIiwwLHsic3R5bGUiOnsidGFpbCI6eyJuYW1lIjoibW9ubyJ9fX1dLFsxLDIsInBfWCIsMCx7InN0eWxlIjp7ImhlYWQiOnsibmFtZSI6ImVwaSJ9fX1dLFsyLDMsIlxcZGVsdGFfWCIsMCx7InN0eWxlIjp7ImJvZHkiOnsibmFtZSI6ImRhc2hlZCJ9fX1dXQ==
\begin{tikzcd}
	X & {X_V} & {X_U} & {\,}
	\arrow["{i_X}", tail, from=1-1, to=1-2]
	\arrow["{p_X}", two heads, from=1-2, to=1-3]
	\arrow["{\delta_X}", dashed, from=1-3, to=1-4]
\end{tikzcd}.
    \end{equation}
\end{definition}

现选定 $\mathsf{Ho}\mathcal{A} \simeq \pi (\mathcal{C} \cap \mathcal{F})$ 的骨架 $\mathcal{K}$.

\begin{lemma}\label{lem:factor-through-w-1}
    对任意 $f : X \to Y$, 可选取 $f_V$ (由提升性) 与 $f_U$ (由 ET3) 使得下图是 conflation 间的态射:
    \begin{equation}
        % https://q.uiver.app/#q=WzAsOCxbMCwwLCJYIl0sWzEsMCwiWF9WIl0sWzIsMCwiWF9VIl0sWzMsMCwiXFwsIl0sWzAsMSwiWSJdLFsxLDEsIllfViJdLFsyLDEsIllfVSJdLFszLDEsIlxcLCJdLFswLDEsImlfWCIsMCx7InN0eWxlIjp7InRhaWwiOnsibmFtZSI6Im1vbm8ifX19XSxbMSwyLCJwX1giLDAseyJzdHlsZSI6eyJoZWFkIjp7Im5hbWUiOiJlcGkifX19XSxbMiwzLCJcXGRlbHRhX1giLDAseyJzdHlsZSI6eyJib2R5Ijp7Im5hbWUiOiJkYXNoZWQifX19XSxbMCw0LCJmIl0sWzEsNSwiZl9WIl0sWzIsNiwiZl9VIl0sWzQsNSwiaV9YIiwwLHsic3R5bGUiOnsidGFpbCI6eyJuYW1lIjoibW9ubyJ9fX1dLFs1LDYsInBfWSIsMCx7InN0eWxlIjp7ImhlYWQiOnsibmFtZSI6ImVwaSJ9fX1dLFs2LDcsIlxcZGVsdGFfWSIsMCx7InN0eWxlIjp7ImJvZHkiOnsibmFtZSI6ImRhc2hlZCJ9fX1dXQ==
\begin{tikzcd}
	X & {X_V} & {X_U} & {\,} \\
	Y & {Y_V} & {Y_U} & {\,}
	\arrow["{i_X}", tail, from=1-1, to=1-2]
	\arrow["f", from=1-1, to=2-1]
	\arrow["{p_X}", two heads, from=1-2, to=1-3]
	\arrow["{f_V}", from=1-2, to=2-2]
	\arrow["{\delta_X}", dashed, from=1-3, to=1-4]
	\arrow["{f_U}", from=1-3, to=2-3]
	\arrow["{i_Y}", tail, from=2-1, to=2-2]
	\arrow["{p_Y}", two heads, from=2-2, to=2-3]
	\arrow["{\delta_Y}", dashed, from=2-3, to=2-4]
\end{tikzcd}.
    \end{equation}
    特别地, $f_V$ 与 $f_U$ 在 $\mathcal{K}$ 中唯一.
    \begin{proof}
        态射 $f$ 是 $\mathsf{Ho}\mathcal{A}$ 中的零态射, 当且仅当 $f$ 通过 $\mathcal{W}$ 中对象分解, 亦当且仅当态射被某个 $U \to V$ 分解 (\Cref{lem:factor-through-w}).
        \\
        ($f_V$ 的唯一性). 对任意不同的 $f_V$ 与 $(f_V)'$ 使得上左侧方块图交换, 则 $(f_V - (f_V)') \circ i_X = 0$. 从而 $f_V - (f_V)'$ 被 $X_U \to Y_V$ 分解, 故为零态射.
        \\
        ($f_U$ 的唯一性). 今选定 $f_V$. 对任意不同的 $f_U$ 与 $(f_U)'$ 使得上图右侧方块交换, 则 $(\delta_Y)_\sharp (f_U - (f_U)') = 0$. 从而 $f_U - (f_U)'$ 被 $X_U \to Y_V$ 分解, 故为零态射.
    \end{proof}
\end{lemma}

\begin{corollary}\label{cor: functoriality of shift}
    $[(\cdot)_U] : \mathcal{A} \xrightarrow{(\cdot)_U} \mathcal{A} \xrightarrow{[\cdot]} \mathcal{K}$ 是加法函子.
    \begin{proof}
        由\Cref{lem:factor-through-w-1}, $X \mapsto [X_U]$ 与 $f \mapsto [X_f]$ 唯一决定. 后续验证步骤与\Cref{thm:extri-frobenius-happel} 完全相同.
    \end{proof}
\end{corollary}

再证明 $[(\cdot)_U]$ 通过 $\mathcal{A} \to \mathcal{K}$ 分解.

\begin{lemma}\label{lem: factor through w-2}
    若 $f \in \mathsf{TCofib}$ ($f \in \mathsf{TFib}$), 则 $U_f$ 是弱等价.
    \begin{proof}
        ($f \in\mathsf{TCofib}$). 考虑 inflation 的复合 $X \overset i \rightarrowtail Y \overset {i_Y}\rightarrowtail Y_V$. 由 ET4 得下图左, 进而得下图右
        \begin{equation}
            % https://q.uiver.app/#q=WzAsMjQsWzAsMCwiWCJdLFsxLDAsIlkiLFsyMzYsMTAwLDYwLDFdXSxbMSwxLCJZX1YiXSxbMCwxLCJYIixbMzUyLDEwMCw2MCwxXV0sWzIsMCwiUyJdLFsyLDEsIkUiLFszNTIsMTAwLDYwLDFdXSxbMSwyLCJZX1UiLFsyMzYsMTAwLDYwLDFdXSxbMiwyLCJZX1UiXSxbMSwzLCJcXCwiXSxbMywwLCJcXCwiXSxbMywxLCJcXCwiXSxbMiwzLCJcXCwiXSxbNCwxLCJYIixbMzUyLDEwMCw2MCwxXV0sWzQsMiwiWSIsWzIzNiwxMDAsNjAsMV1dLFs1LDEsIllfViJdLFs2LDEsIkUiLFszNTIsMTAwLDYwLDFdXSxbNSwyLCJZX1YiXSxbNiwyLCJZX1UiLFsyMzYsMTAwLDYwLDFdXSxbNywxLCJcXCwiXSxbNywyLCJcXCwiXSxbNCwwLCJYIl0sWzUsMCwiWF9WIl0sWzYsMCwiWF9VIl0sWzcsMCwiXFwsIl0sWzAsMSwiaSIsMl0sWzAsMywiIiwwLHsibGV2ZWwiOjIsInN0eWxlIjp7ImhlYWQiOnsibmFtZSI6Im5vbmUifX19XSxbMSw0LCIiLDAseyJzdHlsZSI6eyJoZWFkIjp7Im5hbWUiOiJlcGkifX19XSxbMiw1LCIiLDAseyJjb2xvdXIiOlszNTIsMTAwLDYwXSwic3R5bGUiOnsiaGVhZCI6eyJuYW1lIjoiZXBpIn19fV0sWzEsMiwiXFxtYXRoc2Z7Q29maWJ9IiwwLHsic3R5bGUiOnsidGFpbCI6eyJuYW1lIjoibW9ubyJ9fX1dLFsyLDYsIiIsMCx7ImNvbG91ciI6WzIzNiwxMDAsNjBdLCJzdHlsZSI6eyJoZWFkIjp7Im5hbWUiOiJlcGkifX19XSxbNiw3LCIiLDAseyJsZXZlbCI6Miwic3R5bGUiOnsiaGVhZCI6eyJuYW1lIjoibm9uZSJ9fX1dLFs0LDUsIiIsMSx7InN0eWxlIjp7InRhaWwiOnsibmFtZSI6Im1vbm8ifSwiYm9keSI6eyJuYW1lIjoiZGFzaGVkIn19fV0sWzUsNywiXFxsYW1iZGEiLDAseyJzdHlsZSI6eyJib2R5Ijp7Im5hbWUiOiJkYXNoZWQifSwiaGVhZCI6eyJuYW1lIjoiZXBpIn19fV0sWzAsMSwiXFxtYXRoc2Z7VENvZmlifSIsMCx7InN0eWxlIjp7InRhaWwiOnsibmFtZSI6Im1vbm8ifSwiYm9keSI6eyJuYW1lIjoibm9uZSJ9LCJoZWFkIjp7Im5hbWUiOiJub25lIn19fV0sWzQsOSwiIiwwLHsic3R5bGUiOnsiYm9keSI6eyJuYW1lIjoiZGFzaGVkIn19fV0sWzUsMTAsIlxcd2lkZXRpbGRlIHtcXGRlbHRhX1h9IiwwLHsiY29sb3VyIjpbMzUyLDEwMCw2MF0sInN0eWxlIjp7ImJvZHkiOnsibmFtZSI6ImRhc2hlZCJ9fX0sWzM1MiwxMDAsNjAsMV1dLFs3LDExLCIiLDAseyJzdHlsZSI6eyJib2R5Ijp7Im5hbWUiOiJkYXNoZWQifX19XSxbMTIsMTMsImkiLDAseyJzdHlsZSI6eyJ0YWlsIjp7Im5hbWUiOiJtb25vIn19fV0sWzEyLDE0LCIiLDIseyJjb2xvdXIiOlszNTIsMTAwLDYwXSwic3R5bGUiOnsidGFpbCI6eyJuYW1lIjoibW9ubyJ9fX1dLFsxNCwxNSwiIiwyLHsiY29sb3VyIjpbMzUyLDEwMCw2MF0sInN0eWxlIjp7ImhlYWQiOnsibmFtZSI6ImVwaSJ9fX1dLFsxMywxNiwiIiwwLHsiY29sb3VyIjpbMjM2LDEwMCw2MF0sInN0eWxlIjp7InRhaWwiOnsibmFtZSI6Im1vbm8ifX19XSxbMTYsMTcsIiIsMCx7ImNvbG91ciI6WzIzNiwxMDAsNjBdLCJzdHlsZSI6eyJoZWFkIjp7Im5hbWUiOiJlcGkifX19XSxbMTQsMTYsIjFfe1ZfWX0iLDAseyJsZXZlbCI6Miwic3R5bGUiOnsiaGVhZCI6eyJuYW1lIjoibm9uZSJ9fX1dLFsxNywxOSwiXFxkZWx0YSBfWSIsMCx7ImNvbG91ciI6WzIzNiwxMDAsNjBdLCJzdHlsZSI6eyJib2R5Ijp7Im5hbWUiOiJkYXNoZWQifX19LFsyMzYsMTAwLDYwLDFdXSxbMTUsMTgsIlxcd2lkZXRpbGRlIHtcXGRlbHRhX1h9IiwwLHsiY29sb3VyIjpbMzUyLDEwMCw2MF0sInN0eWxlIjp7ImJvZHkiOnsibmFtZSI6ImRhc2hlZCJ9fX0sWzM1MiwxMDAsNjAsMV1dLFsxNSwxNywiXFxsYW1iZGEiXSxbNiw4LCJcXGRlbHRhIF9ZIiwwLHsiY29sb3VyIjpbMjM2LDEwMCw2MF0sInN0eWxlIjp7ImJvZHkiOnsibmFtZSI6ImRhc2hlZCJ9fX0sWzIzNiwxMDAsNjAsMV1dLFszLDIsIiIsMCx7ImNvbG91ciI6WzM1MiwxMDAsNjBdLCJzdHlsZSI6eyJ0YWlsIjp7Im5hbWUiOiJtb25vIn19fV0sWzIwLDEyLCIiLDAseyJsZXZlbCI6Miwic3R5bGUiOnsiaGVhZCI6eyJuYW1lIjoibm9uZSJ9fX1dLFsyMCwyMV0sWzIxLDIyXSxbMjEsMTQsIiIsMSx7InN0eWxlIjp7ImJvZHkiOnsibmFtZSI6ImRhc2hlZCJ9fX1dLFsyMiwyMywiXFxkZWx0YV9YIiwwLHsic3R5bGUiOnsiYm9keSI6eyJuYW1lIjoiZGFzaGVkIn19fV0sWzIyLDE1LCJcXG11IiwwLHsic3R5bGUiOnsiYm9keSI6eyJuYW1lIjoiZGFzaGVkIn19fV1d
\begin{tikzcd}
	X & \textcolor{rgb,255:red,51;green,65;blue,255}{Y} & S & {\,} & X & {X_V} & {X_U} & {\,} \\
	\textcolor{rgb,255:red,255;green,51;blue,78}{X} & {Y_V} & \textcolor{rgb,255:red,255;green,51;blue,78}{E} & {\,} & \textcolor{rgb,255:red,255;green,51;blue,78}{X} & {Y_V} & \textcolor{rgb,255:red,255;green,51;blue,78}{E} & {\,} \\
	& \textcolor{rgb,255:red,51;green,65;blue,255}{{Y_U}} & {Y_U} && \textcolor{rgb,255:red,51;green,65;blue,255}{Y} & {Y_V} & \textcolor{rgb,255:red,51;green,65;blue,255}{{Y_U}} & {\,} \\
	& {\,} & {\,}
	\arrow["i"', from=1-1, to=1-2]
	\arrow["{\mathsf{TCofib}}", no body, tail, no head, from=1-1, to=1-2]
	\arrow[equals, from=1-1, to=2-1]
	\arrow[two heads, from=1-2, to=1-3]
	\arrow["{\mathsf{Cofib}}", tail, from=1-2, to=2-2]
	\arrow[dashed, from=1-3, to=1-4]
	\arrow[dashed, tail, from=1-3, to=2-3]
	\arrow[from=1-5, to=1-6]
	\arrow[equals, from=1-5, to=2-5]
	\arrow[from=1-6, to=1-7]
	\arrow[dashed, from=1-6, to=2-6]
	\arrow["{\delta_X}", dashed, from=1-7, to=1-8]
	\arrow["\mu", dashed, from=1-7, to=2-7]
	\arrow[draw={rgb,255:red,255;green,51;blue,78}, tail, from=2-1, to=2-2]
	\arrow[draw={rgb,255:red,255;green,51;blue,78}, two heads, from=2-2, to=2-3]
	\arrow[draw={rgb,255:red,51;green,65;blue,255}, two heads, from=2-2, to=3-2]
	\arrow["{\widetilde {\delta_X}}", color={rgb,255:red,255;green,51;blue,78}, dashed, from=2-3, to=2-4]
	\arrow["\lambda", dashed, two heads, from=2-3, to=3-3]
	\arrow[draw={rgb,255:red,255;green,51;blue,78}, tail, from=2-5, to=2-6]
	\arrow["i", tail, from=2-5, to=3-5]
	\arrow[draw={rgb,255:red,255;green,51;blue,78}, two heads, from=2-6, to=2-7]
	\arrow["{1_{V_Y}}", equals, from=2-6, to=3-6]
	\arrow["{\widetilde {\delta_X}}", color={rgb,255:red,255;green,51;blue,78}, dashed, from=2-7, to=2-8]
	\arrow["\lambda", from=2-7, to=3-7]
	\arrow[equals, from=3-2, to=3-3]
	\arrow["{\delta _Y}", color={rgb,255:red,51;green,65;blue,255}, dashed, from=3-2, to=4-2]
	\arrow[dashed, from=3-3, to=4-3]
	\arrow[draw={rgb,255:red,51;green,65;blue,255}, tail, from=3-5, to=3-6]
	\arrow[draw={rgb,255:red,51;green,65;blue,255}, two heads, from=3-6, to=3-7]
	\arrow["{\delta _Y}", color={rgb,255:red,51;green,65;blue,255}, dashed, from=3-7, to=3-8]
\end{tikzcd}.
        \end{equation}
        由\Cref{lem:factor-through-w-1}, $[\mu]$ 是同构. 再由 $S \in \mathcal{W}$, 得 $\lambda \in \mathsf{Weq}$. 从而 $[U_f] = [\lambda \circ \mu]$ 是同构.
        \\
        ($f \in\mathsf{TCofib}$). \Cref{thm:five-lemma-extri} 给出了交换图:
        \begin{equation}
            % https://q.uiver.app/#q=WzAsMTEsWzAsMSwiWCJdLFswLDIsIlkiXSxbMiwxLCJYX1YgXFxvcGx1cyBZX1YiXSxbNCwxLCJYX1UgXFxvcGx1cyBWX1kiXSxbMiwyLCJZX1YiXSxbNCwyLCJZX1UiXSxbNSwxLCJcXCwiXSxbNSwyLCJcXCwiXSxbMCwwLCJWIl0sWzIsMCwiWF9WIl0sWzQsMCwiTiJdLFswLDIsIlxcYmlub20ge2lfWH0wIiwwLHsic3R5bGUiOnsidGFpbCI6eyJuYW1lIjoibW9ubyJ9fX1dLFswLDEsImYiLDIseyJzdHlsZSI6eyJoZWFkIjp7Im5hbWUiOiJlcGkifX19XSxbMSw0LCJpX1kiLDAseyJzdHlsZSI6eyJ0YWlsIjp7Im5hbWUiOiJtb25vIn19fV0sWzIsNCwiKGZfViwgMSkiLDIseyJzdHlsZSI6eyJoZWFkIjp7Im5hbWUiOiJlcGkifX19XSxbNCw1LCJwX1kiLDAseyJzdHlsZSI6eyJoZWFkIjp7Im5hbWUiOiJlcGkifX19XSxbMiwzLCJwX1ggXFxvcGx1cyAxIiwwLHsic3R5bGUiOnsiaGVhZCI6eyJuYW1lIjoiZXBpIn19fV0sWzMsNiwiXFxkZWx0YV9YIFxcb3BsdXMgMCIsMCx7InN0eWxlIjp7ImJvZHkiOnsibmFtZSI6ImRhc2hlZCJ9fX1dLFs1LDcsIlxcZGVsdGFfWSIsMCx7InN0eWxlIjp7ImJvZHkiOnsibmFtZSI6ImRhc2hlZCJ9fX1dLFszLDUsIihmX1UscF9ZKSIsMix7InN0eWxlIjp7ImhlYWQiOnsibmFtZSI6ImVwaSJ9fX1dLFs5LDIsIiIsMCx7InN0eWxlIjp7InRhaWwiOnsibmFtZSI6Im1vbm8ifX19XSxbOCwwLCIiLDAseyJzdHlsZSI6eyJ0YWlsIjp7Im5hbWUiOiJtb25vIn19fV0sWzgsOSwiIiwxLHsic3R5bGUiOnsidGFpbCI6eyJuYW1lIjoibW9ubyJ9LCJib2R5Ijp7Im5hbWUiOiJkYXNoZWQifX19XSxbOSwxMCwiIiwxLHsic3R5bGUiOnsiYm9keSI6eyJuYW1lIjoiZGFzaGVkIn0sImhlYWQiOnsibmFtZSI6ImVwaSJ9fX1dLFsxMCwzLCIiLDAseyJzdHlsZSI6eyJ0YWlsIjp7Im5hbWUiOiJtb25vIn19fV1d
\begin{tikzcd}
	V && {X_V} && N \\
	X && {X_V \oplus Y_V} && {X_U \oplus V_Y} & {\,} \\
	Y && {Y_V} && {Y_U} & {\,}
	\arrow[dashed, tail, from=1-1, to=1-3]
	\arrow[tail, from=1-1, to=2-1]
	\arrow[dashed, two heads, from=1-3, to=1-5]
	\arrow[tail, from=1-3, to=2-3]
	\arrow[tail, from=1-5, to=2-5]
	\arrow["{\binom {i_X}0}", tail, from=2-1, to=2-3]
	\arrow["f"', two heads, from=2-1, to=3-1]
	\arrow["{p_X \oplus 1}", two heads, from=2-3, to=2-5]
	\arrow["{(f_V, 1)}"', two heads, from=2-3, to=3-3]
	\arrow["{\delta_X \oplus 0}", dashed, from=2-5, to=2-6]
	\arrow["{(f_U,p_Y)}"', two heads, from=2-5, to=3-5]
	\arrow["{i_Y}", tail, from=3-1, to=3-3]
	\arrow["{p_Y}", two heads, from=3-3, to=3-5]
	\arrow["{\delta_Y}", dashed, from=3-5, to=3-6]
\end{tikzcd}.
        \end{equation}
        由于 $\mathcal{W}$ 是厚子范畴 (\Cref{thm:hovey-triangle}), 得 $N \in \mathcal{W}$. 由\Cref{thm:weq-trivial-corr} 知 $(f_U, \ p_Y) \in \mathsf{Weq}$. 由于 $X_U \xrightarrow {\binom 10}X_U \oplus V_Y \to X_U$ 也是弱等价, 故 $U_f$ 是弱等价.
    \end{proof}
\end{lemma}

\begin{theorem}\label{thm: shift functor}
    存在加法双函子 $\Sigma: \mathcal{K} \to \mathcal{K}$, 使得 $\Sigma [\cdot] = [(\cdot )_U]$.
    \begin{proof}
        对\Cref{cor: functoriality of shift} 使用\Cref{lem: factor through w-2}, 得局部化诱导的函子 $\widetilde \Sigma : \mathsf{Ho}\mathcal{A} \to \mathcal{K}$. 最后复合 $\mathcal{K} \to \mathsf{Ho}\mathcal{A}$ 即可.
    \end{proof}
\end{theorem}

\begin{theorem}\label{thm:loop functor}
    存在加法双函子 $\Omega: \mathcal{K} \to \mathcal{K}$, 满足 $[(\cdot)^T] = \Omega [\cdot]$, 其中
    \begin{equation}
        % https://q.uiver.app/#q=WzAsNCxbMCwwLCJYXlQiXSxbMSwwLCJYXlMiXSxbMiwwLCJYIl0sWzMsMCwiXFwsIl0sWzAsMSwial9YIiwwLHsic3R5bGUiOnsidGFpbCI6eyJuYW1lIjoibW9ubyJ9fX1dLFsxLDIsInFfWCIsMCx7InN0eWxlIjp7ImhlYWQiOnsibmFtZSI6ImVwaSJ9fX1dLFsyLDMsIlxcdmFyZXBzaWxvbiBfWCIsMCx7InN0eWxlIjp7ImJvZHkiOnsibmFtZSI6ImRhc2hlZCJ9fX1dXQ==
\begin{tikzcd}[ampersand replacement=\&]
	{X^T} \& {X^S} \& X \& {\,}
	\arrow["{j_X}", tail, from=1-1, to=1-2]
	\arrow["{q_X}", two heads, from=1-2, to=1-3]
	\arrow["{\varepsilon _X}", dashed, from=1-3, to=1-4]
\end{tikzcd}.
    \end{equation}
\end{theorem}

在证明 $\Sigma$ 与 $\Omega$ 互逆前, 我们需要将扩张元对应作局部化范畴中的态射. 不同于\Cref{thm:extri-frobenius-happel}, 此处涉及局部化范畴中的``分式''计算.

\begin{definition}\label{def:lesh-transform}
    ($\ell$-变换). 对 conflation $X \overset f \rightarrowtail Y \overset g \twoheadrightarrow Z \overset \eta \dashrightarrow$, 定义
    \begin{equation}
        \ell : \mathbb E(Z,X) \to (Z, \Sigma X)_{\mathcal{K}},\quad \eta \mapsto [a] \circ [s]^{-1}.
    \end{equation}
    其中, $a$ 与 $s$ 来自\Cref{thm:bi-pullback} 给出的交换图:
    \begin{equation}
        % https://q.uiver.app/#q=WzAsMTQsWzMsMCwiWiJdLFsxLDAsIlgiXSxbMiwwLCJZIl0sWzQsMCwiXFwsIl0sWzEsMiwiWF9VIl0sWzEsMywiXFwsIl0sWzEsMSwiWF9WIl0sWzIsMiwiWF9VIl0sWzIsMSwiRSJdLFszLDEsIloiXSxbNCwxLCJcXCwiXSxbMiwzLCJcXCwiXSxbMCwxLCJcXCwiXSxbNSwxLCJcXCwiXSxbMSwyLCJmIiwwLHsic3R5bGUiOnsidGFpbCI6eyJuYW1lIjoibW9ubyJ9fX1dLFsyLDAsImciLDAseyJzdHlsZSI6eyJoZWFkIjp7Im5hbWUiOiJlcGkifX19XSxbMCwzLCJcXGV0YSIsMCx7InN0eWxlIjp7ImJvZHkiOnsibmFtZSI6ImRhc2hlZCJ9fX1dLFs0LDUsIlxcZGVsdGFfWCIsMCx7InN0eWxlIjp7ImJvZHkiOnsibmFtZSI6ImRhc2hlZCJ9fX1dLFs2LDQsInBfWCIsMCx7InN0eWxlIjp7ImhlYWQiOnsibmFtZSI6ImVwaSJ9fX1dLFsxLDYsImlfWCIsMCx7InN0eWxlIjp7InRhaWwiOnsibmFtZSI6Im1vbm8ifX19XSxbOSwwLCIiLDAseyJsZXZlbCI6Miwic3R5bGUiOnsiaGVhZCI6eyJuYW1lIjoibm9uZSJ9fX1dLFs0LDcsIiIsMCx7ImxldmVsIjoyLCJzdHlsZSI6eyJoZWFkIjp7Im5hbWUiOiJub25lIn19fV0sWzcsMTEsIiIsMCx7InN0eWxlIjp7ImJvZHkiOnsibmFtZSI6ImRhc2hlZCJ9fX1dLFs5LDEwLCIiLDAseyJzdHlsZSI6eyJib2R5Ijp7Im5hbWUiOiJkYXNoZWQifX19XSxbNiw4LCIiLDAseyJzdHlsZSI6eyJ0YWlsIjp7Im5hbWUiOiJtb25vIn19fV0sWzgsOSwiXFxtYXRoc2Z7VEZpYn0iLDAseyJjb2xvdXIiOlsyMzYsMTAwLDYwXSwic3R5bGUiOnsiaGVhZCI6eyJuYW1lIjoiZXBpIn19fSxbMjM2LDEwMCw2MCwxXV0sWzIsOCwiIiwwLHsic3R5bGUiOnsidGFpbCI6eyJuYW1lIjoibW9ubyJ9fX1dLFs4LDcsIi1hIiwwLHsiY29sb3VyIjpbMCwxMDAsNjBdLCJzdHlsZSI6eyJoZWFkIjp7Im5hbWUiOiJlcGkifX19LFswLDEwMCw2MCwxXV0sWzgsOSwicyIsMix7ImNvbG91ciI6WzIzNiwxMDAsNjBdLCJzdHlsZSI6eyJib2R5Ijp7Im5hbWUiOiJub25lIn0sImhlYWQiOnsibmFtZSI6Im5vbmUifX19LFsyMzYsMTAwLDYwLDFdXV0=
\begin{tikzcd}
	& X & Y & Z & {\,} \\
	{\,} & {X_V} & E & Z & {\,} & {\,} \\
	& {X_U} & {X_U} \\
	& {\,} & {\,}
	\arrow["f", tail, from=1-2, to=1-3]
	\arrow["{i_X}", tail, from=1-2, to=2-2]
	\arrow["g", two heads, from=1-3, to=1-4]
	\arrow[tail, from=1-3, to=2-3]
	\arrow["\eta", dashed, from=1-4, to=1-5]
	\arrow[tail, from=2-2, to=2-3]
	\arrow["{p_X}", two heads, from=2-2, to=3-2]
	\arrow["{\mathsf{TFib}}", color={rgb,255:red,51;green,65;blue,255}, two heads, from=2-3, to=2-4]
	\arrow["s"', color={rgb,255:red,51;green,65;blue,255}, draw=none, from=2-3, to=2-4]
	\arrow["{-a}", color={rgb,255:red,255;green,51;blue,51}, two heads, from=2-3, to=3-3]
	\arrow[equals, from=2-4, to=1-4]
	\arrow[dashed, from=2-4, to=2-5]
	\arrow[equals, from=3-2, to=3-3]
	\arrow["{\delta_X}", dashed, from=3-2, to=4-2]
	\arrow[dashed, from=3-3, to=4-3]
\end{tikzcd}.
    \end{equation}
    需要检验, 这是良定义的.
\end{definition}

\begin{remark}
    \Cref{thm:bi-pullback} 的等式表明 $s^\ast \eta = a^\ast \delta_X$. 定义表明 $\ell(\delta_X) = 1$. 我们希望 $\ell (\eta) = [a] \circ [s]^{-1}$.
\end{remark}

\begin{definition}
    (分式). 若 $s \in \mathsf{TFib}$, 我们将一对态射 $A \overset s\gets B \overset f\to C$ 称作一个分式. 往后记作 $(s,f)$. 分式仅看作 $\mathsf{Mor} \sqcup \mathsf{TFib}$ 中的一类字, $(s,f)$ 在局部化范畴中的取值是 $[f] \circ [s]^{-1}$.
\end{definition}

\begin{remark}
    \Cref{def:lesh-transform} 也可通过 $\mathsf{TCofib}$ 对偶地定义, 从而得到反向的分式. 我们不使用这种分式, 因此无需考虑左分式与右分式之别.
\end{remark}

\begin{proposition}\label{prop:lesh-well-defined}
    沿用\Cref{def:lesh-transform} 中的记号. 假定存在另一组分式 $(t,b)$, 满足
    \begin{equation}
        s^\ast \eta = a^\ast \delta _X,\quad t^\ast \eta = b^\ast \delta _X.
    \end{equation}
    此时, $[a]\circ [s]^{-1} = [b] \circ [t]^{-1}$.
    \begin{proof}
        将 $(s,a)$ 与 $(t,b)$ 置于下图:
        \begin{equation}
            % https://q.uiver.app/#q=WzAsMTAsWzIsMCwiWiJdLFswLDAsIlgiXSxbMSwwLCJZIl0sWzMsMCwiXFwsIl0sWzIsMiwiWF9VIl0sWzMsMiwiXFwsIl0sWzEsMiwiWF9WIl0sWzAsMiwiWCJdLFsxLDEsIk0iLFsyLDEwMCw2MCwxXV0sWzMsMSwiTiIsWzIzNywxMDAsNjAsMV1dLFsxLDIsImYiLDAseyJzdHlsZSI6eyJ0YWlsIjp7Im5hbWUiOiJtb25vIn19fV0sWzIsMCwiZyIsMCx7InN0eWxlIjp7ImhlYWQiOnsibmFtZSI6ImVwaSJ9fX1dLFswLDMsIlxcZXRhIiwwLHsic3R5bGUiOnsiYm9keSI6eyJuYW1lIjoiZGFzaGVkIn19fV0sWzQsNSwiXFxkZWx0YV9YIiwyLHsic3R5bGUiOnsiYm9keSI6eyJuYW1lIjoiZGFzaGVkIn19fV0sWzYsNCwicF9YIiwyLHsic3R5bGUiOnsiaGVhZCI6eyJuYW1lIjoiZXBpIn19fV0sWzEsNywiIiwyLHsibGV2ZWwiOjIsInN0eWxlIjp7ImhlYWQiOnsibmFtZSI6Im5vbmUifX19XSxbOCwwLCJzIiwyLHsiY29sb3VyIjpbMiwxMDAsNjBdfSxbMiwxMDAsNjAsMV1dLFs4LDQsImEiLDAseyJjb2xvdXIiOlsyLDEwMCw2MF19LFsyLDEwMCw2MCwxXV0sWzksMCwidCIsMCx7ImNvbG91ciI6WzIzNywxMDAsNjBdfSxbMjM3LDEwMCw2MCwxXV0sWzksNCwiYiIsMix7ImNvbG91ciI6WzIzNywxMDAsNjBdfSxbMjM3LDEwMCw2MCwxXV0sWzksMCwiXFxtYXRoc2Z7VEZpYn0iLDIseyJjb2xvdXIiOlsyMzcsMTAwLDYwXSwic3R5bGUiOnsiYm9keSI6eyJuYW1lIjoibm9uZSJ9LCJoZWFkIjp7Im5hbWUiOiJlcGkifX19LFsyMzcsMTAwLDYwLDFdXSxbOCwwLCJcXG1hdGhzZntURmlifSIsMCx7ImNvbG91ciI6WzIsMTAwLDYwXSwic3R5bGUiOnsiYm9keSI6eyJuYW1lIjoibm9uZSJ9LCJoZWFkIjp7Im5hbWUiOiJlcGkifX19LFsyLDEwMCw2MCwxXV0sWzcsNiwiaV9YIiwyLHsic3R5bGUiOnsidGFpbCI6eyJuYW1lIjoibW9ubyJ9fX1dXQ==
\begin{tikzcd}
	X & Y & Z & {\,} \\
	& \textcolor{rgb,255:red,255;green,58;blue,51}{M} && \textcolor{rgb,255:red,51;green,61;blue,255}{N} \\
	X & {X_V} & {X_U} & {\,}
	\arrow["f", tail, from=1-1, to=1-2]
	\arrow[equals, from=1-1, to=3-1]
	\arrow["g", two heads, from=1-2, to=1-3]
	\arrow["\eta", dashed, from=1-3, to=1-4]
	\arrow["s"', color={rgb,255:red,255;green,58;blue,51}, from=2-2, to=1-3]
	\arrow["{\mathsf{TFib}}", color={rgb,255:red,255;green,58;blue,51}, no body, two heads, from=2-2, to=1-3]
	\arrow["a", color={rgb,255:red,255;green,58;blue,51}, from=2-2, to=3-3]
	\arrow["t", color={rgb,255:red,51;green,61;blue,255}, from=2-4, to=1-3]
	\arrow["{\mathsf{TFib}}"', color={rgb,255:red,51;green,61;blue,255}, no body, two heads, from=2-4, to=1-3]
	\arrow["b"', color={rgb,255:red,51;green,61;blue,255}, from=2-4, to=3-3]
	\arrow["{i_X}"', tail, from=3-1, to=3-2]
	\arrow["{p_X}"', two heads, from=3-2, to=3-3]
	\arrow["{\delta_X}"', dashed, from=3-3, to=3-4]
\end{tikzcd}.
        \end{equation}
        使用\Cref{thm:bi-pullback} 作 $s'$ 与 $t'$, 记合成态射 $st' = c = ts'$, 得下图:
        \begin{equation}\label{eq:lesh-transform-well-defined}
% https://q.uiver.app/#q=WzAsMTEsWzIsMCwiWiJdLFswLDAsIlgiXSxbMSwwLCJZIl0sWzMsMCwiXFwsIl0sWzIsMiwiWF9VIl0sWzMsMiwiXFwsIl0sWzEsMiwiWF9WIl0sWzAsMiwiWCJdLFsxLDEsIk0iLFsyLDEwMCw2MCwxXV0sWzMsMSwiTiIsWzIzNywxMDAsNjAsMV1dLFsyLDEsIkYiXSxbMSwyLCJmIiwwLHsic3R5bGUiOnsidGFpbCI6eyJuYW1lIjoibW9ubyJ9fX1dLFsyLDAsImciLDAseyJzdHlsZSI6eyJoZWFkIjp7Im5hbWUiOiJlcGkifX19XSxbMCwzLCJcXGV0YSIsMCx7InN0eWxlIjp7ImJvZHkiOnsibmFtZSI6ImRhc2hlZCJ9fX1dLFs0LDUsIlxcZGVsdGFfWCIsMix7InN0eWxlIjp7ImJvZHkiOnsibmFtZSI6ImRhc2hlZCJ9fX1dLFs2LDQsInBfWCIsMix7InN0eWxlIjp7ImhlYWQiOnsibmFtZSI6ImVwaSJ9fX1dLFsxLDcsIiIsMix7ImxldmVsIjoyLCJzdHlsZSI6eyJoZWFkIjp7Im5hbWUiOiJub25lIn19fV0sWzcsNiwiaV9YIiwyLHsic3R5bGUiOnsidGFpbCI6eyJuYW1lIjoibW9ubyJ9fX1dLFs4LDAsInMiLDAseyJjb2xvdXIiOlsyLDEwMCw2MF0sInN0eWxlIjp7ImhlYWQiOnsibmFtZSI6ImVwaSJ9fX0sWzIsMTAwLDYwLDFdXSxbOCw0LCJhIiwyLHsiY29sb3VyIjpbMiwxMDAsNjBdfSxbMiwxMDAsNjAsMV1dLFs5LDAsInQiLDIseyJjb2xvdXIiOlsyMzcsMTAwLDYwXSwic3R5bGUiOnsiaGVhZCI6eyJuYW1lIjoiZXBpIn19fSxbMjM3LDEwMCw2MCwxXV0sWzksNCwiYiIsMCx7ImNvbG91ciI6WzIzNywxMDAsNjBdfSxbMjM3LDEwMCw2MCwxXV0sWzEwLDgsInQnIiwyLHsiY29sb3VyIjpbMiwxMDAsNjBdLCJzdHlsZSI6eyJoZWFkIjp7Im5hbWUiOiJlcGkifX19LFsyLDEwMCw2MCwxXV0sWzEwLDksInMnIiwwLHsiY29sb3VyIjpbMjM3LDEwMCw2MF0sInN0eWxlIjp7ImhlYWQiOnsibmFtZSI6ImVwaSJ9fX0sWzIzNywxMDAsNjAsMV1dLFsxMCwwLCJjIiwwLHsiY29sb3VyIjpbMjczLDEwMCw2MF19LFsyNzMsMTAwLDYwLDFdXSxbMTAsNCwiXFxib3hlZCA/IiwxLHsic3R5bGUiOnsiYm9keSI6eyJuYW1lIjoibm9uZSJ9LCJoZWFkIjp7Im5hbWUiOiJub25lIn19fV1d
\begin{tikzcd}
	X & Y & Z & {\,} \\
	& \textcolor{rgb,255:red,255;green,58;blue,51}{M} & F & \textcolor{rgb,255:red,51;green,61;blue,255}{N} \\
	X & {X_V} & {X_U} & {\,}
	\arrow["f", tail, from=1-1, to=1-2]
	\arrow[equals, from=1-1, to=3-1]
	\arrow["g", two heads, from=1-2, to=1-3]
	\arrow["\eta", dashed, from=1-3, to=1-4]
	\arrow["s", color={rgb,255:red,255;green,58;blue,51}, two heads, from=2-2, to=1-3]
	\arrow["a"', color={rgb,255:red,255;green,58;blue,51}, from=2-2, to=3-3]
	\arrow["c", color={rgb,255:red,163;green,51;blue,255}, from=2-3, to=1-3]
	\arrow["{t'}"', color={rgb,255:red,255;green,58;blue,51}, two heads, from=2-3, to=2-2]
	\arrow["{s'}", color={rgb,255:red,51;green,61;blue,255}, two heads, from=2-3, to=2-4]
	\arrow["{\boxed ?}"{description}, draw=none, from=2-3, to=3-3]
	\arrow["t"', color={rgb,255:red,51;green,61;blue,255}, two heads, from=2-4, to=1-3]
	\arrow["b", color={rgb,255:red,51;green,61;blue,255}, from=2-4, to=3-3]
	\arrow["{i_X}"', tail, from=3-1, to=3-2]
	\arrow["{p_X}"', two heads, from=3-2, to=3-3]
	\arrow["{\delta_X}"', dashed, from=3-3, to=3-4]
\end{tikzcd}.
        \end{equation}
        由 $[a] \circ [s]^{-1} = [at'] \circ  [c]^{-1}$ 与 $[b] \circ [t] = [bs'] \circ [c]$, 下只需证明 $[at'] = [bs']$. 注意到
        \begin{equation}
            (at' - bs')^\ast \delta_X = (st' - ts')^\ast \eta = 0^\ast \eta = 0,
        \end{equation}
        因此, $(at' - bs')$ 经 $p_X$ 分解. 由 $X_V \in \mathcal{W}$, 得 $[at' - bs'] = 0$.
    \end{proof}
\end{proposition}

\begin{proposition}
    ($\ell$ 保持加法). 对任意 $\eta, \eta' \in \mathbb E(Z,X)$, 总有 $\ell(\eta + \eta ' ) = \ell (\eta) + \ell (\eta')$.
    \begin{proof}
        \Cref{prop:lesh-well-defined} 说明 $\ell(\cdot)$ 无关分式选取, \Cref{eq:lesh-transform-well-defined} 说明任意两个分式可以``通分''. 因此
        \begin{equation}
            \ell(\eta + \eta') = [a+a'] \circ [s]^{-1} = [a] \circ [s]^{-1} + [a'] \circ [s]^{-1} = \ell(\eta) + \ell(\eta').
        \end{equation}
    \end{proof}
\end{proposition}

\begin{theorem}
    ($\ell$ 是自然变换). 假定下图左侧是 conflation 间的态射, 则下图右侧交换:
    \begin{equation}
        % https://q.uiver.app/#q=WzAsMTIsWzAsMCwiWCJdLFsxLDAsIlkiXSxbMiwwLCJaIl0sWzAsMSwiWCciXSxbMSwxLCJZJyJdLFsyLDEsIlonIl0sWzMsMCwiXFwsIl0sWzMsMSwiXFwsIl0sWzQsMCwiW1pdIl0sWzUsMCwiW1hfVV0iXSxbNSwxLCJbWCdfVV0iXSxbNCwxLCJbWiddIl0sWzAsMSwiIiwwLHsic3R5bGUiOnsidGFpbCI6eyJuYW1lIjoibW9ubyJ9fX1dLFsxLDIsIiIsMCx7InN0eWxlIjp7ImhlYWQiOnsibmFtZSI6ImVwaSJ9fX1dLFszLDQsIiIsMCx7InN0eWxlIjp7InRhaWwiOnsibmFtZSI6Im1vbm8ifX19XSxbNCw1LCIiLDAseyJzdHlsZSI6eyJoZWFkIjp7Im5hbWUiOiJlcGkifX19XSxbMCwzLCJmIl0sWzIsNSwiZyJdLFsyLDYsIlxcZXRhIiwwLHsic3R5bGUiOnsiYm9keSI6eyJuYW1lIjoiZGFzaGVkIn19fV0sWzUsNywiXFxldGEnIiwwLHsic3R5bGUiOnsiYm9keSI6eyJuYW1lIjoiZGFzaGVkIn19fV0sWzgsOSwiXFxlbGwgKFxcZXRhKSJdLFs4LDExLCJbZ10iLDJdLFs5LDEwLCJbZl9VXSJdLFsxLDRdLFsxMSwxMCwiXFxlbGwgKFxcZXRhJykiLDJdXQ==
\begin{tikzcd}
	X & Y & Z & {\,} & {[Z]} & {[X_U]} \\
	{X'} & {Y'} & {Z'} & {\,} & {[Z']} & {[X'_U]}
	\arrow[tail, from=1-1, to=1-2]
	\arrow["f", from=1-1, to=2-1]
	\arrow[two heads, from=1-2, to=1-3]
	\arrow[from=1-2, to=2-2]
	\arrow["\eta", dashed, from=1-3, to=1-4]
	\arrow["g", from=1-3, to=2-3]
	\arrow["{\ell (\eta)}", from=1-5, to=1-6]
	\arrow["{[g]}"', from=1-5, to=2-5]
	\arrow["{[f_U]}", from=1-6, to=2-6]
	\arrow[tail, from=2-1, to=2-2]
	\arrow[two heads, from=2-2, to=2-3]
	\arrow["{\eta'}", dashed, from=2-3, to=2-4]
	\arrow["{\ell (\eta')}"', from=2-5, to=2-6]
\end{tikzcd}.
    \end{equation}
    \begin{proof}
        记 $(s,a)$ 与 $(t,b)$ 分别是 $\eta$ 与 $\eta'$ 对应的分式, 只需验证下图在局部化范畴中交换:
        \begin{equation}
            % https://q.uiver.app/#q=WzAsNixbMiwwLCJYX1UiXSxbMCwxLCJaJyJdLFsxLDAsIk0iXSxbMSwxLCJOIl0sWzIsMSwiWCdfVSJdLFswLDAsIloiXSxbMywxLCJ0IiwwLHsiY29sb3VyIjpbMzU5LDEwMCw1OV0sInN0eWxlIjp7ImhlYWQiOnsibmFtZSI6ImVwaSJ9fX0sWzM1OSwxMDAsNTksMV1dLFsyLDAsImEiXSxbMCw0LCJmX1UiXSxbNSwxLCJnIiwyXSxbMiw1LCJzIiwyLHsiY29sb3VyIjpbMzU5LDEwMCw1OV0sInN0eWxlIjp7ImhlYWQiOnsibmFtZSI6ImVwaSJ9fX0sWzM1OSwxMDAsNTksMV1dLFszLDQsImIiLDJdXQ==
\begin{tikzcd}
	Z & M & {X_U} \\
	{Z'} & N & {X'_U}
	\arrow["g"', from=1-1, to=2-1]
	\arrow["s"', color={rgb,255:red,255;green,46;blue,49}, two heads, from=1-2, to=1-1]
	\arrow["a", from=1-2, to=1-3]
	\arrow["{f_U}", from=1-3, to=2-3]
	\arrow["t", color={rgb,255:red,255;green,46;blue,49}, two heads, from=2-2, to=2-1]
	\arrow["b"', from=2-2, to=2-3]
\end{tikzcd}.
        \end{equation}
        由\Cref{thm:homotopy-pullback-2} 构造同伦的推出拉回方块 $1$, 得 $t' \in \mathsf{TCofib}$ 与 $p'$. 定义 $b' = b\circ p$, 得交换方块 $2$. 由\Cref{thm:bi-pullback} 作同伦的推出拉回方块 $3$, 以及 $\overline s, \overline {t'} \in \mathsf{TCofib}$. 最后只需证明 $\boxed ?$ 在局部化范畴中交换:
        \begin{equation}\label{eq:lesh-naturality-goal}
            % https://q.uiver.app/#q=WzAsMTEsWzAsMCwiWiJdLFs0LDAsIlhfVSJdLFs0LDMsIlhfVSciXSxbMCwzLCJaJyJdLFsyLDAsIk0iXSxbMiwzLCJOIl0sWzIsMiwiQyJdLFs0LDIsIlgnX1UiXSxbMCwyLCJaIl0sWzIsMSwiRSJdLFszLDEsIlxcYm94ZWQgPyJdLFs0LDAsInMiLDIseyJjb2xvdXIiOlszNTksMTAwLDU5XSwic3R5bGUiOnsiaGVhZCI6eyJuYW1lIjoiZXBpIn19fSxbMzU5LDEwMCw1OSwxXV0sWzUsMywidCIsMCx7ImNvbG91ciI6WzM1OSwxMDAsNTldLCJzdHlsZSI6eyJoZWFkIjp7Im5hbWUiOiJlcGkifX19LFszNTksMTAwLDU5LDFdXSxbNCwxLCJhIl0sWzUsMiwiYiIsMl0sWzEsNywiZl9VIl0sWzcsMiwiIiwwLHsibGV2ZWwiOjIsInN0eWxlIjp7ImhlYWQiOnsibmFtZSI6Im5vbmUifX19XSxbNiw3LCJiJyIsMCx7InN0eWxlIjp7ImJvZHkiOnsibmFtZSI6ImRhc2hlZCJ9fX1dLFs4LDMsImciLDJdLFs2LDgsInQnIiwyLHsiY29sb3VyIjpbMzU5LDEwMCw1OV0sInN0eWxlIjp7ImJvZHkiOnsibmFtZSI6ImRhc2hlZCJ9LCJoZWFkIjp7Im5hbWUiOiJlcGkifX19LFszNTksMTAwLDU5LDFdXSxbMCw4LCIiLDEseyJsZXZlbCI6Miwic3R5bGUiOnsiaGVhZCI6eyJuYW1lIjoibm9uZSJ9fX1dLFs5LDQsIlxcb3ZlcmxpbmUgcyIsMCx7ImNvbG91ciI6WzM1OSwxMDAsNTldLCJzdHlsZSI6eyJoZWFkIjp7Im5hbWUiOiJlcGkifX19LFszNTksMTAwLDU5LDFdXSxbOSw2LCJcXG92ZXJsaW5lIHt0J30iLDIseyJjb2xvdXIiOlszNTksMTAwLDU5XSwic3R5bGUiOnsiaGVhZCI6eyJuYW1lIjoiZXBpIn19fSxbMzU5LDEwMCw1OSwxXV0sWzYsNSwicCIsMCx7InN0eWxlIjp7ImJvZHkiOnsibmFtZSI6ImRhc2hlZCJ9fX1dLFs4LDUsIjEiLDEseyJzdHlsZSI6eyJib2R5Ijp7Im5hbWUiOiJub25lIn0sImhlYWQiOnsibmFtZSI6Im5vbmUifX19XSxbNiwyLCIyIiwxLHsic3R5bGUiOnsiYm9keSI6eyJuYW1lIjoibm9uZSJ9LCJoZWFkIjp7Im5hbWUiOiJub25lIn19fV0sWzAsNiwiMyIsMSx7InN0eWxlIjp7ImJvZHkiOnsibmFtZSI6Im5vbmUifSwiaGVhZCI6eyJuYW1lIjoibm9uZSJ9fX1dXQ==
\begin{tikzcd}
	Z && M && {X_U} \\
	&& E & {\boxed ?} \\
	Z && C && {X'_U} \\
	{Z'} && N && {X_U'}
	\arrow[equals, from=1-1, to=3-1]
	\arrow["3"{description}, draw=none, from=1-1, to=3-3]
	\arrow["s"', color={rgb,255:red,255;green,46;blue,49}, two heads, from=1-3, to=1-1]
	\arrow["a", from=1-3, to=1-5]
	\arrow["{f_U}", from=1-5, to=3-5]
	\arrow["{\overline s}", color={rgb,255:red,255;green,46;blue,49}, two heads, from=2-3, to=1-3]
	\arrow["{\overline {t'}}"', color={rgb,255:red,255;green,46;blue,49}, two heads, from=2-3, to=3-3]
	\arrow["g"', from=3-1, to=4-1]
	\arrow["1"{description}, draw=none, from=3-1, to=4-3]
	\arrow["{t'}"', color={rgb,255:red,255;green,46;blue,49}, dashed, two heads, from=3-3, to=3-1]
	\arrow["{b'}", dashed, from=3-3, to=3-5]
	\arrow["p", dashed, from=3-3, to=4-3]
	\arrow["2"{description}, draw=none, from=3-3, to=4-5]
	\arrow[equals, from=3-5, to=4-5]
	\arrow["t", color={rgb,255:red,255;green,46;blue,49}, two heads, from=4-3, to=4-1]
	\arrow["b"', from=4-3, to=4-5]
\end{tikzcd}.
        \end{equation}
        我们希望 $(b'\circ \overline {t'} - f_U \circ a \circ \overline s)$ 通过 $p_{X'} : X'_V \twoheadrightarrow X'_U$ 分解, 即证 $(b'\circ \overline {t'} - f_U \circ a \circ \overline s)^\ast \delta _{X'} = 0$. 计算得
\begin{align}
& \quad \ (b'\circ \overline {t'} - f_U \circ a \circ \overline s)^\ast \delta _{X'} &&= (b\circ p \circ \overline {t'})^\ast \delta_{X'} - (f_U \circ a \circ \overline s)^\ast \delta _{X'} \\
&= (p \circ \overline {t'})^\ast (b^\ast \delta_{X'}) - (a \circ \overline s)^\ast ((f_U)^\ast \delta _{X'}) &&= (p \circ \overline {t'})^\ast (t^\ast \eta') - (a \circ \overline s)^\ast (f_\ast \delta _{X}) \\
&= (t\circ p \circ \overline {t'})^\ast \eta' - f_\ast \overline s^\ast (a^\ast \delta _{X}) &&= (g\circ s \circ \overline {s})^\ast \eta' - f_\ast \overline s^\ast (s^\ast \eta) \\
&= \overline s^\ast s^\ast (g^\ast \eta' -f_\ast \eta) &&= 0.
\end{align}
    \end{proof}
\end{theorem}

在说明 $\ell$ 是自然变换的前提下, 我们证明 \Cref{thm: shift functor} 与\Cref{thm:loop functor} 是局部化范畴的自等价.

\begin{theorem}\label{thm:shift-auto}
    $\Sigma (\Omega (\cdot)) : \mathcal{K} \to \mathcal{K}$ 是自同构.
    \begin{proof}
        由\Cref{thm:bi-pullback} 作交换图:
        \begin{equation}
            % https://q.uiver.app/#q=WzAsMTAsWzAsMCwiWF5UIl0sWzEsMCwiWF5TIl0sWzIsMCwiWCJdLFswLDEsIihYXlQpX1YiXSxbMCwyLCIoWF5UKV9VIl0sWzEsMiwiKFheVClfVSJdLFsyLDEsIlgiXSxbMSwxLCJGWCJdLFszLDAsIlxcLCJdLFswLDMsIlxcLCJdLFswLDEsIiIsMCx7InN0eWxlIjp7InRhaWwiOnsibmFtZSI6Im1vbm8ifX19XSxbMSwyLCIiLDAseyJzdHlsZSI6eyJoZWFkIjp7Im5hbWUiOiJlcGkifX19XSxbMCwzLCIiLDAseyJzdHlsZSI6eyJ0YWlsIjp7Im5hbWUiOiJtb25vIn19fV0sWzMsNCwiIiwwLHsic3R5bGUiOnsiaGVhZCI6eyJuYW1lIjoiZXBpIn19fV0sWzMsNywiIiwwLHsic3R5bGUiOnsidGFpbCI6eyJuYW1lIjoibW9ubyJ9LCJib2R5Ijp7Im5hbWUiOiJkYXNoZWQifX19XSxbNyw2LCJzX1giLDAseyJzdHlsZSI6eyJib2R5Ijp7Im5hbWUiOiJkYXNoZWQifSwiaGVhZCI6eyJuYW1lIjoiZXBpIn19fV0sWzEsNywiIiwwLHsic3R5bGUiOnsidGFpbCI6eyJuYW1lIjoibW9ubyJ9LCJib2R5Ijp7Im5hbWUiOiJkYXNoZWQifX19XSxbNyw1LCItdF9YIiwwLHsic3R5bGUiOnsiYm9keSI6eyJuYW1lIjoiZGFzaGVkIn0sImhlYWQiOnsibmFtZSI6ImVwaSJ9fX1dLFs0LDUsIiIsMCx7ImxldmVsIjoyLCJzdHlsZSI6eyJoZWFkIjp7Im5hbWUiOiJub25lIn19fV0sWzIsNiwiIiwwLHsibGV2ZWwiOjIsInN0eWxlIjp7ImhlYWQiOnsibmFtZSI6Im5vbmUifX19XSxbMiw4LCJcXHZhcmVwc2lsb24gX1giLDAseyJzdHlsZSI6eyJib2R5Ijp7Im5hbWUiOiJkYXNoZWQifX19XSxbNCw5LCJcXGRlbHRhX3tYXlR9IiwwLHsic3R5bGUiOnsiYm9keSI6eyJuYW1lIjoiZGFzaGVkIn19fV1d
\begin{tikzcd}[ampersand replacement=\&]
	{X^T} \& {X^S} \& X \& {\,} \\
	{(X^T)_V} \& FX \& X \\
	{(X^T)_U} \& {(X^T)_U} \\
	{\,}
	\arrow[tail, from=1-1, to=1-2]
	\arrow[tail, from=1-1, to=2-1]
	\arrow[two heads, from=1-2, to=1-3]
	\arrow[dashed, tail, from=1-2, to=2-2]
	\arrow["{\varepsilon _X}", dashed, from=1-3, to=1-4]
	\arrow[equals, from=1-3, to=2-3]
	\arrow[dashed, tail, from=2-1, to=2-2]
	\arrow[two heads, from=2-1, to=3-1]
	\arrow["{s_X}", dashed, two heads, from=2-2, to=2-3]
	\arrow["{-t_X}", dashed, two heads, from=2-2, to=3-2]
	\arrow[equals, from=3-1, to=3-2]
	\arrow["{\delta_{X^T}}", dashed, from=3-1, to=4-1]
\end{tikzcd}.
        \end{equation}
        由定义, $\ell(\varepsilon_X) = [t_X] \circ [s_X]^{-1}$. 因此 $\ell(\varepsilon_X) : X \to \Sigma\Omega X$ 是同构. 继而证明 $\ell (\varepsilon_\cdot)$ 是自然同构. 对任意 $f : X \to Y$, 则有诱导的 $S^f$, $f^T$ 与 $(f^T)_V$. 依照与\Cref{thm:extri-frobenius-happel} 中相同的论证, 取诱导的 $Ff : FX \to FY$, 以及 $\overline{f} : X \to Y$ 与 $\overline{(f^T)_U} : (X^T)_U \to (Y^T)_U$. 特别地, $[(f^T)_U] = [\overline {(f^T)_U}]$ 且 $[f] = [\overline f]$. 对扩张元间的态射 $(f^T, \overline f) : \varepsilon_X \to \varepsilon_Y$, 由 $\ell$ 是自然变换, 得
        \begin{equation}
            \ell(\varepsilon _Y) \circ [\overline f] = \Sigma[f^T] \circ \ell(\varepsilon _X) = \Sigma \Omega[f] \circ \ell(\varepsilon _X).
        \end{equation}
        这说明 $\ell(\varepsilon_X) : X \to \Sigma \Omega X$ 是一族自然同构.
    \end{proof}
\end{theorem}

\begin{theorem}
    $(\mathcal{K}, \Sigma, \mathcal{E})$ 是三角范畴. 其中,
    \begin{equation}
        \mathcal{E} = \{[X] \xrightarrow {[f]}[Y] \xrightarrow {[g]}[Z] \xrightarrow {\ell (\eta)}\Sigma [X] \mid X \overset f \rightarrowtail Y \overset g \twoheadrightarrow Z \overset \eta \dashrightarrow \ \text{是 conflation}\}.
    \end{equation}
\begin{proof}
    由\Cref{thm: quillen theorem} 与\Cref{cor: homotopy is ideal}, $\mathcal{K}$ 即 $\frac{\mathcal{C} \cap \mathcal{F}}{\mathcal{C} \cap \mathcal{W} \cap \mathcal{F}}$ 的骨架. 由 $\mathcal{C} \cap \mathcal{F}$ 是外三角范畴 (\Cref{thm: extension-closed is extri}), $\mathcal{C} \cap \mathcal{W} \cap \mathcal{F}$ 包含投射且内射的对象, 因此 $\frac{\mathcal{C} \cap \mathcal{F}}{\mathcal{C} \cap \mathcal{W} \cap \mathcal{F}}$ 是外三角范畴 (\Cref{thm:extri-quotient}). 由 $\Sigma$ 是自同构 (\Cref{thm:shift-auto}), 只需证明\Cref{def:lesh-transform} 诱导的自然变换 $\mathbb E([Z], [X]) \to ([Z], \Sigma [X])$ 是同构.
    \begin{enumerate}
        \item (满). 对任意 $[f] : [Z] \to \Sigma [X]$, 有原像 $f^\ast \delta_X$.
        \item (单). 记 $X \overset i \rightarrowtail Y \overset p \twoheadrightarrow Z \overset \eta \dashrightarrow$. 若 $\ell (\eta) = 0$, 则存在分式 $(s,a)$ 使得 $f^\ast \delta _X= s^\ast \eta = 0$, 其中 $s \in \mathsf{TFib}$ 通过 $p$ 分解. 由 $\mathbb E(Z, \mathcal{V}) = 0$, 则存在 $V \in \mathcal{V}$ 使得 $s = [M \simeq V \oplus Z \xrightarrow {(0 \ \ 1)}Z]$. 因此 $(0 \ \ 1)$ 被 $p$ 分解, 从而 $p$ 是可裂满, 即 $\eta = 0$. 
    \end{enumerate}
    结合\Cref{thm:extri-with-sigma-is-triangulated}, 以上是三角范畴.
\end{proof}  
\end{theorem}

以下特殊情形可避免使用\Cref{thm:shift-auto}, 即证明 $\Sigma$ 是子等价.

\begin{example}
    假定 Hovey 孪生余挠对是遗传的, 则 $\mathcal{C} \cap \mathcal{F}$ 是 Forbenius 外三角范畴. 其投射内射对象是 $\mathcal{C} \cap \mathcal{W} \cap \mathcal{F}$.
    \begin{proof}
        先说明 $\mathcal{C} \cap \mathcal{W} \cap \mathcal{F}$ 恰好包含 $\mathcal{C} \cap \mathcal{F}$ 中的投射内射对象 (未必足够多). 下仅证明投射情形.
        \begin{enumerate}
            \item 由 $\mathbb E(\mathcal{C}\cap \mathcal{W} \cap \mathcal{F}, \mathcal{C} \cap \mathcal{F}) = 0$, 故 $\mathcal{C}\cap \mathcal{W} \cap \mathcal{F}$ 中对象必投射.
            \item 对任意投射对象 $P$, 取可裂 conflation $P^T \rightarrowtail P^S \twoheadrightarrow P$. 由 $P^S \in \mathcal{W}$, 得 $P \in \mathcal{W}$. 从而 $P \in \mathcal{C} \cap \mathcal{W} \cap \mathcal{F}$.
        \end{enumerate}
        若余挠对遗传, 则 $P^T \rightarrowtail P^S \twoheadrightarrow P$ 中 $P^T \in \mathrm{coCone}(\mathcal{U},\mathcal{U}) = \mathcal{U} = \mathcal{C}$. 因此 $P^S \in \mathcal{C} \cap \mathcal{S} \subseteq \mathcal{C} \cap \mathcal{F}$. 由扩张闭, 上述 conflation 是 $\mathcal{C} \cap \mathcal{F}$ 中取值的.
    \end{proof}
\end{example}









